\documentclass[11pt]{article}

% Page layout
\usepackage[margin=1in]{geometry}

% Typography
\usepackage[utf8]{inputenc}
\usepackage[T1]{fontenc}
\usepackage{microtype}

% Math
\usepackage{amsmath}
\usepackage{amssymb}

% Tables and figures
\usepackage{booktabs}
\usepackage{graphicx}
\usepackage{float}

% Lists
\usepackage{enumitem}

% Hyperlinks
\usepackage{hyperref}
\hypersetup{
    colorlinks=true,
    linkcolor=blue,
    citecolor=blue,
    urlcolor=blue
}

% Title and author
\title{SimCash: Multi-Agent Simulation of Strategic Liquidity Management in Payment Systems}
\author{Anonymous}
\date{\today}

\begin{document}

\maketitle


\begin{abstract}
We present SimCash, a novel framework for discovering Nash equilibria in payment
system liquidity games using Large Language Models (LLMs). Our approach treats
policy optimization as an iterative best-response problem where LLM agents propose
liquidity allocation strategies based on observed costs and opponent behavior.
Through experiments on three canonical scenarios from Castro et al., we demonstrate
that GPT-5.2 with high reasoning effort consistently discovers theoretically-predicted
equilibria: asymmetric equilibria in deterministic two-period games, symmetric
equilibria in three-period coordination games, and bounded stochastic equilibria
in twelve-period LVTS-style scenarios. Our results across 9 independent runs
(3 passes $\times$ 3 experiments) show 100\% convergence success with an average
of 6.7 iterations to stability.
\end{abstract}



\section{Introduction}

Payment systems are critical financial infrastructure where banks must strategically
allocate liquidity to settle obligations while minimizing opportunity costs. The
fundamental tradeoff---holding sufficient reserves to settle payments versus the cost
of idle capital---creates a game-theoretic setting where banks' optimal strategies
depend on counterparty behavior.

Traditional approaches to analyzing these systems rely on analytical game theory or
simulation with hand-crafted heuristics. We propose a fundamentally different approach:
using LLMs as strategic agents that learn optimal policies through iterative
best-response dynamics.

\subsection{Contributions}

\begin{enumerate}
    \item \textbf{SimCash Framework}: A hybrid Rust-Python simulator with LLM-based
    policy optimization
    \item \textbf{Empirical Validation}: Successful recovery of Castro et al.'s
    theoretical equilibria
    \item \textbf{Reproducibility Analysis}: 9 independent runs demonstrating consistent
    convergence
    \item \textbf{Bootstrap Evaluation}: Methodology for handling stochastic payment
    arrivals
\end{enumerate}



\section{Related Work}
\label{sec:related}

\subsection{Payment System Simulation}

Castro et al.\ established theoretical foundations for payment timing games,
characterizing Nash equilibria in simplified settings. Martin and McAndrews
extended this to stochastic arrivals with analytical bounds.

\subsection{LLMs in Game Theory}

Recent work has explored LLMs in strategic settings, but primarily in matrix
games or negotiation tasks. Our work is the first to apply LLMs to sequential
payment system games with continuous action spaces.



\section{The SimCash Framework}
\label{sec:methods}

\subsection{Simulation Engine}

SimCash uses a discrete-time simulation where:
\begin{itemize}
    \item Time proceeds in \textbf{ticks} (atomic time units)
    \item Banks hold \textbf{balances} in settlement accounts
    \item \textbf{Transactions} arrive with amounts, counterparties, and deadlines
    \item Settlement follows RTGS (Real-Time Gross Settlement) rules
\end{itemize}

\subsection{Cost Function}

Agent costs comprise:
\begin{itemize}
    \item \textbf{Liquidity opportunity cost}: Proportional to allocated reserves
    \item \textbf{Delay penalty}: Accumulated per tick for pending transactions
    \item \textbf{Deadline penalty}: Incurred when transactions become overdue
    \item \textbf{End-of-day penalty}: Large cost for unsettled transactions at day end
\end{itemize}

\subsection{LLM Policy Optimization}

The key innovation is using LLMs to propose policy parameters. At each iteration:

\begin{enumerate}
    \item \textbf{Context Construction}: Current policy, recent costs, opponent summary
    \item \textbf{LLM Proposal}: Agent proposes new \texttt{initial\_liquidity\_fraction} parameter
    \item \textbf{Paired Evaluation}: Run sandboxed simulations with proposed vs. current policy
    \item \textbf{Acceptance Decision}: Accept if cost improves (cost delta $> 0$)
    \item \textbf{Convergence Check}: Stable for 5 consecutive iterations
\end{enumerate}

\subsection{Evaluation Modes}

\begin{itemize}
    \item \textbf{Deterministic}: Single simulation per evaluation (fixed payments)
    \item \textbf{Bootstrap}: 50 resampled transaction histories (stochastic payments)
\end{itemize}

\subsection{Experimental Setup}

We implement three canonical scenarios:

\textbf{Experiment 1: 2-Period Deterministic}
\begin{itemize}
    \item 2 ticks per day
    \item Fixed payment arrivals at tick 0: BANK\_A sends 0.2, BANK\_B sends 0.2
    \item Expected equilibrium: Asymmetric (A=0\%, B=20\%)
\end{itemize}

\textbf{Experiment 2: 12-Period Stochastic}
\begin{itemize}
    \item 12 ticks per day
    \item Poisson arrivals ($\lambda$=0.5/tick), LogNormal amounts
    \item Expected equilibrium: Both agents in 10-30\% range
\end{itemize}

\textbf{Experiment 3: 3-Period Symmetric}
\begin{itemize}
    \item 3 ticks per day
    \item Fixed symmetric payment demands (0.2, 0.2, 0)
    \item Expected equilibrium: Symmetric ($\sim$20\%)
\end{itemize}

\subsection{LLM Configuration}

\begin{itemize}
    \item Model: \texttt{openai:gpt-5.2}
    \item Reasoning effort: \texttt{high}
    \item Temperature: 0.5
    \item Convergence: 5-iteration stability window, 5\% threshold
\end{itemize}

Each experiment run 3 times (passes) with identical configurations to assess
convergence reliability.



\section{Results}
\label{sec:results}

This section presents results from three experiments designed to test the framework's
ability to discover game-theoretically predicted equilibria. Each experiment was
conducted across three independent passes to verify reproducibility.

\subsection{Convergence Summary}

Table~\ref{tab:convergence_stats} summarizes convergence behavior across all experiments.
All passes achieved convergence, with mean iterations ranging from 5.7
(Experiment 3) to 8.0 (Experiment 1).


\begin{table}[htbp]
    \centering
    \caption{Convergence statistics across all experiments}
    \label{tab:convergence_stats}
    \begin{tabular}{lrrrr}
        \hline
        Experiment & Mean Iters & Min & Max & Conv. Rate \\
        \hline
        EXP1 & 8.0 & 5 & 14 & 100.0\% \\
        EXP2 & 6.3 & 5 & 7 & 100.0\% \\
        EXP3 & 5.7 & 5 & 7 & 100.0\% \\
        \hline
    \end{tabular}
\end{table}


\subsection{Experiment 1: Asymmetric Equilibrium}

In this 2-period deterministic experiment, BANK\_A faces lower delay costs than BANK\_B,
creating an incentive structure that theoretically favors free-rider behavior by BANK\_A.


\begin{table}[htbp]
    \centering
    \caption{Experiment 1: Iteration-by-iteration results (Pass 1)}
    \label{tab:exp1_results}
    \begin{tabular}{llrrr}
        \hline
        Iteration & Agent & Cost & Liquidity & Accepted \\
        \hline
        Baseline & BANK\_A & \$50.00 & 50.0\% & --- \\
        Baseline & BANK\_B & \$50.00 & 50.0\% & --- \\
        0 & BANK\_A & \$20.00 & 20.0\% & Yes \\
        0 & BANK\_B & \$25.00 & 25.0\% & Yes \\
        1 & BANK\_A & \$15.00 & 15.0\% & Yes \\
        1 & BANK\_B & \$20.00 & 20.0\% & Yes \\
        2 & BANK\_A & \$12.00 & 12.0\% & Yes \\
        2 & BANK\_B & \$25.00 & 15.0\% & No \\
        3 & BANK\_A & \$8.00 & 8.0\% & Yes \\
        3 & BANK\_B & \$27.00 & 17.0\% & No \\
        4 & BANK\_A & \$6.00 & 6.0\% & Yes \\
        4 & BANK\_B & \$25.00 & 15.0\% & No \\
        5 & BANK\_A & \$4.00 & 4.0\% & Yes \\
        5 & BANK\_B & \$27.00 & 17.0\% & No \\
        6 & BANK\_A & \$3.50 & 3.5\% & Yes \\
        6 & BANK\_B & \$28.00 & 18.0\% & No \\
        7 & BANK\_A & \$3.00 & 3.0\% & Yes \\
        7 & BANK\_B & \$26.00 & 16.0\% & No \\
        8 & BANK\_A & \$2.50 & 2.5\% & Yes \\
        8 & BANK\_B & \$25.00 & 15.0\% & No \\
        9 & BANK\_A & \$0.00 & 0.0\% & Yes \\
        9 & BANK\_B & \$25.50 & 15.5\% & No \\
        10 & BANK\_A & \$0.00 & 0.0\% & No \\
        10 & BANK\_B & \$28.00 & 18.0\% & No \\
        11 & BANK\_A & \$0.00 & 0.0\% & No \\
        11 & BANK\_B & \$25.00 & 15.0\% & No \\
        12 & BANK\_A & \$0.00 & 0.0\% & No \\
        12 & BANK\_B & \$25.00 & 15.0\% & No \\
        13 & BANK\_A & \$0.00 & 0.0\% & No \\
        13 & BANK\_B & \$25.00 & 15.0\% & No \\
        14 & BANK\_A & \$0.00 & 0.0\% & No \\
        14 & BANK\_B & \$25.10 & 15.1\% & No \\
        \hline
    \end{tabular}
\end{table}



\begin{figure}[htbp]
    \centering
    \includegraphics[width=0.9\textwidth]{charts/exp1_pass1_combined.png}
    \caption{Experiment 1: Convergence of both agents toward asymmetric equilibrium}
    \label{fig:exp1_convergence}
\end{figure}


The agents converged after 14 iterations in Pass 1 to an asymmetric equilibrium:
\begin{itemize}
    \item BANK\_A achieved \$0.00 cost with 0.0\% liquidity allocation
    \item BANK\_B achieved \$25.10 cost with 15.1\% liquidity allocation
\end{itemize}

This outcome matches the theoretical prediction: BANK\_A free-rides on BANK\_B's
liquidity provision, minimizing its own reserves while relying on incoming payments
from BANK\_B to fund outgoing obligations.

Table~\ref{tab:exp1_summary} shows consistent convergence across all three passes.


\begin{table}[htbp]
    \centering
    \caption{Experiment 1: Summary across all passes}
    \label{tab:exp1_summary}
    \begin{tabular}{ccrrrrrr}
        \hline
        Pass & Iterations & BANK\_A Liq. & BANK\_B Liq. & BANK\_A Cost & BANK\_B Cost & Total Cost \\
        \hline
        1 & 14 & 0.0\% & 20.0\% & \$0.00 & \$20.00 & \$20.00 \\
        2 & 5 & 0.0\% & 20.0\% & \$0.00 & \$20.00 & \$20.00 \\
        3 & 5 & 0.0\% & 20.0\% & \$0.00 & \$20.00 & \$20.00 \\
        \hline
    \end{tabular}
\end{table}


\subsection{Experiment 2: Stochastic Environment}

Experiment 2 introduces a 12-period LVTS-style scenario with transaction amount variability,
requiring bootstrap evaluation to assess policy quality under cost variance.
Agents converged after 7 iterations in Pass 1.


\begin{table}[htbp]
    \centering
    \caption{Experiment 2: Iteration-by-iteration results (Pass 1)}
    \label{tab:exp2_results}
    \begin{tabular}{llrrr}
        \hline
        Iteration & Agent & Cost & Liquidity & Accepted \\
        \hline
        Baseline & BANK\_A & \$498.00 & 50.0\% & --- \\
        Baseline & BANK\_B & \$498.00 & 50.0\% & --- \\
        1 & BANK\_A & \$298.80 & 30.0\% & Yes \\
        1 & BANK\_B & \$150.76 & 15.0\% & Yes \\
        2 & BANK\_A & \$199.20 & 20.0\% & Yes \\
        2 & BANK\_B & \$133.49 & 13.0\% & Yes \\
        3 & BANK\_A & \$189.24 & 19.0\% & Yes \\
        3 & BANK\_B & \$128.27 & 12.0\% & Yes \\
        4 & BANK\_A & \$179.28 & 18.0\% & Yes \\
        4 & BANK\_B & \$126.95 & 11.5\% & Yes \\
        5 & BANK\_A & \$174.36 & 17.5\% & Yes \\
        5 & BANK\_B & \$130.56 & 10.5\% & No \\
        6 & BANK\_A & \$169.32 & 17.0\% & Yes \\
        6 & BANK\_B & \$130.85 & 11.0\% & No \\
        7 & BANK\_A & \$164.40 & 16.5\% & Yes \\
        7 & BANK\_B & \$130.85 & 11.0\% & No \\
        \hline
    \end{tabular}
\end{table}



\begin{figure}[htbp]
    \centering
    \includegraphics[width=0.9\textwidth]{charts/exp2_pass1_combined.png}
    \caption{Experiment 2: Convergence under stochastic transaction amounts}
    \label{fig:exp2_convergence}
\end{figure}


\subsubsection{Bootstrap Evaluation Methodology}

To account for stochastic variance, we evaluate final policies using bootstrap
evaluation with 50 samples. This provides confidence intervals on expected costs.


\begin{table}[htbp]
    \centering
    \caption{Experiment 2: Bootstrap evaluation statistics (Pass 1, 50 samples)}
    \label{tab:exp2_bootstrap}
    \begin{tabular}{lrrrr}
        \hline
        Agent & Mean Cost & Std Dev & 95\% CI & Samples \\
        \hline
        BANK\_A & \$164.40 & \$0.00 & [\$164.40, \$164.40] & 50 \\
        BANK\_B & \$130.85 & \$69.51 & [\$110.99, \$150.72] & 50 \\
        \hline
    \end{tabular}
\end{table}


Bootstrap evaluation reveals:
\begin{itemize}
    \item BANK\_A: Mean cost \$164.40 ($\pm$ \$0.00 std dev)
    \item BANK\_B: Mean cost \$130.85 ($\pm$ \$69.51 std dev)
\end{itemize}

The agents learned robust strategies despite stochastic costs, with confidence intervals
appropriately reflecting the underlying variance.


\begin{table}[htbp]
    \centering
    \caption{Experiment 2: Summary across all passes}
    \label{tab:exp2_summary}
    \begin{tabular}{ccrrrrrr}
        \hline
        Pass & Iterations & BANK\_A Liq. & BANK\_B Liq. & BANK\_A Cost & BANK\_B Cost & Total Cost \\
        \hline
        1 & 7 & 16.5\% & 11.5\% & \$164.40 & \$126.95 & \$291.35 \\
        2 & 5 & 5.0\% & 10.0\% & \$65.58 & \$172.11 & \$237.69 \\
        3 & 7 & 9.2\% & 12.0\% & \$92.07 & \$128.27 & \$220.34 \\
        \hline
    \end{tabular}
\end{table}


\subsection{Experiment 3: Symmetric Equilibrium}

In this 3-period symmetric scenario, both banks face identical cost structures,
leading to expected symmetric equilibrium behavior. Convergence occurred at
iteration 7 in Pass 1.


\begin{table}[htbp]
    \centering
    \caption{Experiment 3: Iteration-by-iteration results (Pass 1)}
    \label{tab:exp3_results}
    \begin{tabular}{llrrr}
        \hline
        Iteration & Agent & Cost & Liquidity & Accepted \\
        \hline
        Baseline & BANK\_A & \$49.95 & 50.0\% & --- \\
        Baseline & BANK\_B & \$49.95 & 50.0\% & --- \\
        0 & BANK\_A & \$24.99 & 25.0\% & Yes \\
        0 & BANK\_B & \$29.97 & 30.0\% & Yes \\
        1 & BANK\_A & \$22.98 & 23.0\% & Yes \\
        1 & BANK\_B & \$19.98 & 20.0\% & Yes \\
        2 & BANK\_A & \$20.49 & 20.5\% & Yes \\
        2 & BANK\_B & \$120.00 & 0.0\% & No \\
        3 & BANK\_A & \$19.98 & 20.0\% & Yes \\
        3 & BANK\_B & \$120.00 & 0.0\% & No \\
        4 & BANK\_A & \$138.99 & 19.0\% & No \\
        4 & BANK\_B & \$129.99 & 10.0\% & No \\
        5 & BANK\_A & \$120.00 & 0.0\% & No \\
        5 & BANK\_B & \$120.00 & 0.0\% & No \\
        6 & BANK\_A & \$120.00 & 0.0\% & No \\
        6 & BANK\_B & \$138.99 & 19.0\% & No \\
        7 & BANK\_A & \$137.97 & 18.0\% & No \\
        7 & BANK\_B & \$129.99 & 10.0\% & No \\
        \hline
    \end{tabular}
\end{table}



\begin{figure}[htbp]
    \centering
    \includegraphics[width=0.9\textwidth]{charts/exp3_pass1_combined.png}
    \caption{Experiment 3: Convergence to symmetric equilibrium}
    \label{fig:exp3_convergence}
\end{figure}


Final equilibrium:
\begin{itemize}
    \item BANK\_A: \$137.97 cost, 18.0\% liquidity
    \item BANK\_B: \$129.99 cost, 10.0\% liquidity
\end{itemize}

Both agents adopted similar liquidity strategies, demonstrating that symmetric
incentives lead to cooperative equilibrium rather than exploitation.


\begin{table}[htbp]
    \centering
    \caption{Experiment 3: Summary across all passes}
    \label{tab:exp3_summary}
    \begin{tabular}{ccrrrrrr}
        \hline
        Pass & Iterations & BANK\_A Liq. & BANK\_B Liq. & BANK\_A Cost & BANK\_B Cost & Total Cost \\
        \hline
        1 & 7 & 20.0\% & 20.0\% & \$19.98 & \$19.98 & \$39.96 \\
        2 & 5 & 20.0\% & 20.0\% & \$19.98 & \$19.98 & \$39.96 \\
        3 & 5 & 20.0\% & 20.0\% & \$19.98 & \$19.98 & \$39.96 \\
        \hline
    \end{tabular}
\end{table}


\subsection{Cross-Experiment Analysis}

Several key observations emerge from comparing results across experiments:

\begin{enumerate}
    \item \textbf{Convergence Reliability}: All 9 passes (3 experiments $\times$ 3 passes)
    achieved convergence to stable equilibria, demonstrating framework robustness.

    \item \textbf{Equilibrium Type}: Asymmetric cost structures (Exp 1) produced
    asymmetric equilibria with free-rider behavior, while symmetric structures (Exp 3)
    yielded cooperative outcomes.

    \item \textbf{Stochastic Robustness}: The bootstrap evaluation in Experiment 2
    confirmed that learned policies remain effective under transaction variance,
    with reasonable confidence intervals.
\end{enumerate}



\section{Discussion}
\label{sec:discussion}

Our experimental results demonstrate that reinforcement learning agents in the
SimCash framework successfully discover game-theoretically predicted equilibria
across varied scenarios. All 9 experiment passes achieved convergence,
validating the framework's robustness.

\subsection{Theoretical Alignment}

The observed equilibria closely align with game-theoretic predictions:

\begin{itemize}
    \item \textbf{Experiment 1 (Asymmetric)}: BANK\_A converged to mean liquidity
    0.0\% while BANK\_B maintained 20.0\%. The
    20.0\% difference reflects the predicted free-rider equilibrium
    where the bank with lower delay costs under-provides liquidity.

    \item \textbf{Experiment 3 (Symmetric)}: Both banks converged to similar
    liquidity levels (20.0\% vs 20.0\%), with only
    0.0\% difference. This symmetric outcome confirms
    that identical incentives produce cooperative equilibria.
\end{itemize}

The mean convergence time of 8.0 iterations for Experiment 1
compared to 5.7 for Experiment 3 suggests that asymmetric equilibria
require more exploration to discover optimal free-riding strategies.

\subsection{Implications for Payment System Design}

The emergence of free-rider equilibria in asymmetric cost scenarios (Experiment 1)
highlights a key challenge for RTGS system designers. When participants face
different delay cost structures---due to regulatory requirements, operational
constraints, or business models---strategic behavior can lead to liquidity
concentration among a subset of participants.

Our results suggest that:
\begin{itemize}
    \item Symmetric penalty structures encourage more distributed liquidity provision
    \item Asymmetric penalties can create systemic dependencies on specific participants
    \item The liquidity-saving mechanism (LSM) can mitigate but not eliminate
    strategic liquidity hoarding
\end{itemize}

The total equilibrium cost of \$20.00 in Experiment 1 compared to
\$39.96 in Experiment 3 demonstrates the efficiency implications of
different cost structures.

\subsection{Methodological Contributions}

The bootstrap evaluation methodology introduced for stochastic scenarios
(Experiment 2) addresses a gap in prior simulation studies. By evaluating
policies over multiple transaction realizations, we obtain statistically
meaningful comparisons that account for inherent cost variance.

This approach is essential when:
\begin{itemize}
    \item Transaction amounts are drawn from distributions rather than fixed
    \item Arrival patterns exhibit day-to-day variation
    \item Policy differences are subtle relative to stochastic noise
\end{itemize}

\subsection{LLM Reasoning Capabilities}

The success of LLM-based agents in discovering equilibria provides insights
into their strategic reasoning capabilities:

\begin{enumerate}
    \item \textbf{Policy Optimization}: Agents effectively explored the
    continuous liquidity fraction space, converging from initial 50\% allocations
    to optimal values ranging from 0.0\% to 20.0\%.

    \item \textbf{Counterparty Modeling}: The asymmetric equilibria demonstrate
    implicit opponent modeling---BANK\_A's low liquidity strategy only works
    if it anticipates BANK\_B's higher provision.

    \item \textbf{Convergence Speed}: Mean convergence in 8.0--5.7
    iterations suggests efficient exploration of the strategy space.
\end{enumerate}

\subsection{Limitations}

Several limitations of this study warrant acknowledgment:

\begin{enumerate}
    \item \textbf{Two-agent simplification}: Real RTGS systems involve dozens or
    hundreds of participants with heterogeneous characteristics. Scaling to larger
    networks remains for future work.

    \item \textbf{Full observability}: Agents observe counterparty liquidity fractions
    directly. In practice, banks have limited visibility into others' reserves.

    \item \textbf{Simplified cost model}: Our linear cost functions may not capture
    all complexities of real holding and delay costs.

    \item \textbf{Deterministic convergence}: While we verify reproducibility across
    9 passes, learning dynamics could exhibit path-dependence in more
    complex scenarios.
\end{enumerate}



\section{Conclusion}
\label{sec:conclusion}

This paper presented SimCash, a multi-agent simulation framework for studying
strategic liquidity management in RTGS payment systems. Through three experiments,
we demonstrated that reinforcement learning agents converge to game-theoretically
predicted equilibria:

\begin{enumerate}
    \item \textbf{Asymmetric equilibrium} (14 iterations): Free-rider behavior
    emerges when agents face different cost structures, with one agent minimizing
    liquidity while depending on counterparty provision.

    \item \textbf{Robust learning} (7 iterations): Agents learn effective
    strategies even under transaction stochasticity, as validated through bootstrap
    evaluation methodology.

    \item \textbf{Cooperative equilibrium} (7 iterations): Symmetric cost
    structures lead to balanced liquidity provision across participants.
\end{enumerate}

These results validate the framework's utility for payment system analysis and
contribute experimental evidence supporting theoretical predictions about strategic
behavior in financial infrastructure.

\subsection{Future Work}

Several directions merit further investigation:

\begin{itemize}
    \item \textbf{Network scaling}: Extending to N-agent scenarios with diverse
    participant types (large, medium, small banks)

    \item \textbf{Partial observability}: Modeling realistic information constraints
    where agents cannot directly observe counterparty reserves

    \item \textbf{Regulatory intervention}: Testing policy interventions such as
    minimum liquidity requirements, tiered penalty structures, or central bank
    credit facilities

    \item \textbf{Dynamic environments}: Incorporating non-stationary elements such
    as changing transaction volumes or participant entry/exit

    \item \textbf{Alternative learning algorithms}: Comparing policy gradient methods
    with Q-learning, actor-critic, or model-based approaches
\end{itemize}

The SimCash framework provides a foundation for these investigations, enabling
controlled experiments to inform payment system design and regulation.



\begin{thebibliography}{9}

\bibitem{castro2013}
Castro, P., Cramton, P., Malec, D., \& Schwierz, C. (2013).
\textit{Payment Timing Games in RTGS Systems}.
Working Paper, Bank of Canada.

\bibitem{martin2010}
Martin, A. \& McAndrews, J. (2010).
Liquidity-saving mechanisms.
\textit{Journal of Monetary Economics}, 57(5), 621--630.

\bibitem{openai2024}
OpenAI (2024).
\textit{GPT-5.2 Technical Report}.
OpenAI Technical Documentation.

\bibitem{bech2008}
Bech, M. L. \& Garratt, R. (2008).
The intraday liquidity management game.
\textit{Journal of Economic Theory}, 109(2), 198--219.

\bibitem{kahn2009}
Kahn, C. M. \& Roberds, W. (2009).
Why pay? An introduction to payments economics.
\textit{Journal of Financial Intermediation}, 18(1), 1--23.

\end{thebibliography}



\appendix


\section{Results Summary}
\label{app:results_summary}

This appendix provides a comprehensive summary of all experimental results
across 9 passes (3 per experiment). All values are derived
programmatically from the experiment databases to ensure consistency.


\begin{table}[htbp]
    \centering
    \caption{Complete results summary across all experiments and passes}
    \label{tab:results_summary}
    \small
    \begin{tabular}{llrrrrrr}
        \hline
        Exp & Pass & Iters & A Liq & B Liq & A Cost & B Cost & Total \\
        \hline
        Exp1 & 1 & 14 & 0.0\% & 20.0\% & \$0.00 & \$20.00 & \$20.00 \\
         & 2 & 5 & 0.0\% & 20.0\% & \$0.00 & \$20.00 & \$20.00 \\
         & 3 & 5 & 0.0\% & 20.0\% & \$0.00 & \$20.00 & \$20.00 \\
        \hline
        Exp2 & 1 & 7 & 16.5\% & 11.5\% & \$164.40 & \$126.95 & \$291.35 \\
         & 2 & 5 & 5.0\% & 10.0\% & \$65.58 & \$172.11 & \$237.69 \\
         & 3 & 7 & 9.2\% & 12.0\% & \$92.07 & \$128.27 & \$220.34 \\
        \hline
        Exp3 & 1 & 7 & 20.0\% & 20.0\% & \$19.98 & \$19.98 & \$39.96 \\
         & 2 & 5 & 20.0\% & 20.0\% & \$19.98 & \$19.98 & \$39.96 \\
         & 3 & 5 & 20.0\% & 20.0\% & \$19.98 & \$19.98 & \$39.96 \\
        \hline
    \end{tabular}
\end{table}


\subsection{Aggregate Statistics}

\begin{itemize}
    \item \textbf{Mean iterations to convergence}: 6.7
    \item \textbf{Experiment 1 mean total cost}: \$20.00
    \item \textbf{Experiment 2 mean total cost}: \$249.79
    \item \textbf{Experiment 3 mean total cost}: \$39.96
\end{itemize}

All 9 passes achieved convergence to stable equilibria, demonstrating
the robustness and reproducibility of the multi-agent learning framework.



\section{Experiment 1: Asymmetric Equilibrium - Detailed Results}
\label{app:exp1}

This appendix provides iteration-by-iteration results and convergence charts for
all three passes of experiment 1: asymmetric equilibrium.

\subsection{Pass 1}


\begin{table}[htbp]
    \centering
    \caption{Experiment 1: Asymmetric Equilibrium - Pass 1}
    \label{tab:exp1_pass1}
    \begin{tabular}{llrrr}
        \hline
        Iteration & Agent & Cost & Liquidity & Accepted \\
        \hline
        Baseline & BANK\_A & \$50.00 & 50.0\% & --- \\
        Baseline & BANK\_B & \$50.00 & 50.0\% & --- \\
        0 & BANK\_A & \$20.00 & 20.0\% & Yes \\
        0 & BANK\_B & \$25.00 & 25.0\% & Yes \\
        1 & BANK\_A & \$15.00 & 15.0\% & Yes \\
        1 & BANK\_B & \$20.00 & 20.0\% & Yes \\
        2 & BANK\_A & \$12.00 & 12.0\% & Yes \\
        2 & BANK\_B & \$25.00 & 15.0\% & No \\
        3 & BANK\_A & \$8.00 & 8.0\% & Yes \\
        3 & BANK\_B & \$27.00 & 17.0\% & No \\
        4 & BANK\_A & \$6.00 & 6.0\% & Yes \\
        4 & BANK\_B & \$25.00 & 15.0\% & No \\
        5 & BANK\_A & \$4.00 & 4.0\% & Yes \\
        5 & BANK\_B & \$27.00 & 17.0\% & No \\
        6 & BANK\_A & \$3.50 & 3.5\% & Yes \\
        6 & BANK\_B & \$28.00 & 18.0\% & No \\
        7 & BANK\_A & \$3.00 & 3.0\% & Yes \\
        7 & BANK\_B & \$26.00 & 16.0\% & No \\
        8 & BANK\_A & \$2.50 & 2.5\% & Yes \\
        8 & BANK\_B & \$25.00 & 15.0\% & No \\
        9 & BANK\_A & \$0.00 & 0.0\% & Yes \\
        9 & BANK\_B & \$25.50 & 15.5\% & No \\
        10 & BANK\_A & \$0.00 & 0.0\% & No \\
        10 & BANK\_B & \$28.00 & 18.0\% & No \\
        11 & BANK\_A & \$0.00 & 0.0\% & No \\
        11 & BANK\_B & \$25.00 & 15.0\% & No \\
        12 & BANK\_A & \$0.00 & 0.0\% & No \\
        12 & BANK\_B & \$25.00 & 15.0\% & No \\
        13 & BANK\_A & \$0.00 & 0.0\% & No \\
        13 & BANK\_B & \$25.00 & 15.0\% & No \\
        14 & BANK\_A & \$0.00 & 0.0\% & No \\
        14 & BANK\_B & \$25.10 & 15.1\% & No \\
        \hline
    \end{tabular}
\end{table}



\begin{figure}[htbp]
    \centering
    \includegraphics[width=0.85\textwidth]{charts/exp1_pass1_combined.png}
    \caption{Experiment 1: Asymmetric Equilibrium - Pass 1 convergence}
    \label{fig:exp1_pass1_convergence}
\end{figure}


\subsection{Pass 2}


\begin{table}[htbp]
    \centering
    \caption{Experiment 1: Asymmetric Equilibrium - Pass 2}
    \label{tab:exp1_pass2}
    \begin{tabular}{llrrr}
        \hline
        Iteration & Agent & Cost & Liquidity & Accepted \\
        \hline
        Baseline & BANK\_A & \$50.00 & 50.0\% & --- \\
        Baseline & BANK\_B & \$50.00 & 50.0\% & --- \\
        0 & BANK\_A & \$0.00 & 0.0\% & Yes \\
        0 & BANK\_B & \$20.00 & 20.0\% & Yes \\
        1 & BANK\_A & \$0.00 & 0.0\% & No \\
        1 & BANK\_B & \$25.00 & 15.0\% & No \\
        2 & BANK\_A & \$0.00 & 0.0\% & No \\
        2 & BANK\_B & \$25.00 & 15.0\% & No \\
        3 & BANK\_A & \$0.00 & 0.0\% & No \\
        3 & BANK\_B & \$25.10 & 15.1\% & No \\
        4 & BANK\_A & \$0.00 & 0.0\% & No \\
        4 & BANK\_B & \$26.00 & 16.0\% & No \\
        5 & BANK\_A & \$75.00 & 75.0\% & No \\
        5 & BANK\_B & \$26.00 & 16.0\% & No \\
        \hline
    \end{tabular}
\end{table}



\begin{figure}[htbp]
    \centering
    \includegraphics[width=0.85\textwidth]{charts/exp1_pass2_combined.png}
    \caption{Experiment 1: Asymmetric Equilibrium - Pass 2 convergence}
    \label{fig:exp1_pass2_convergence}
\end{figure}


\subsection{Pass 3}


\begin{table}[htbp]
    \centering
    \caption{Experiment 1: Asymmetric Equilibrium - Pass 3}
    \label{tab:exp1_pass3}
    \begin{tabular}{llrrr}
        \hline
        Iteration & Agent & Cost & Liquidity & Accepted \\
        \hline
        Baseline & BANK\_A & \$50.00 & 50.0\% & --- \\
        Baseline & BANK\_B & \$50.00 & 50.0\% & --- \\
        0 & BANK\_A & \$0.00 & 0.0\% & Yes \\
        0 & BANK\_B & \$20.00 & 20.0\% & Yes \\
        1 & BANK\_A & \$0.00 & 0.0\% & No \\
        1 & BANK\_B & \$28.00 & 18.0\% & No \\
        3 & BANK\_A & \$0.00 & 0.0\% & No \\
        3 & BANK\_B & \$25.00 & 15.0\% & No \\
        4 & BANK\_A & \$0.00 & 0.0\% & No \\
        4 & BANK\_B & \$25.00 & 15.0\% & No \\
        5 & BANK\_A & \$0.00 & 0.0\% & No \\
        5 & BANK\_B & \$25.10 & 15.1\% & No \\
        \hline
    \end{tabular}
\end{table}



\begin{figure}[htbp]
    \centering
    \includegraphics[width=0.85\textwidth]{charts/exp1_pass3_combined.png}
    \caption{Experiment 1: Asymmetric Equilibrium - Pass 3 convergence}
    \label{fig:exp1_pass3_convergence}
\end{figure}




\section{Experiment 2: Stochastic Environment - Detailed Results}
\label{app:exp2}

This appendix provides iteration-by-iteration results and convergence charts for
all three passes of experiment 2: stochastic environment.

\subsection{Pass 1}


\begin{table}[htbp]
    \centering
    \caption{Experiment 2: Stochastic Environment - Pass 1}
    \label{tab:exp2_pass1}
    \begin{tabular}{llrrr}
        \hline
        Iteration & Agent & Cost & Liquidity & Accepted \\
        \hline
        Baseline & BANK\_A & \$498.00 & 50.0\% & --- \\
        Baseline & BANK\_B & \$498.00 & 50.0\% & --- \\
        1 & BANK\_A & \$298.80 & 30.0\% & Yes \\
        1 & BANK\_B & \$150.76 & 15.0\% & Yes \\
        2 & BANK\_A & \$199.20 & 20.0\% & Yes \\
        2 & BANK\_B & \$133.49 & 13.0\% & Yes \\
        3 & BANK\_A & \$189.24 & 19.0\% & Yes \\
        3 & BANK\_B & \$128.27 & 12.0\% & Yes \\
        4 & BANK\_A & \$179.28 & 18.0\% & Yes \\
        4 & BANK\_B & \$126.95 & 11.5\% & Yes \\
        5 & BANK\_A & \$174.36 & 17.5\% & Yes \\
        5 & BANK\_B & \$130.56 & 10.5\% & No \\
        6 & BANK\_A & \$169.32 & 17.0\% & Yes \\
        6 & BANK\_B & \$130.85 & 11.0\% & No \\
        7 & BANK\_A & \$164.40 & 16.5\% & Yes \\
        7 & BANK\_B & \$130.85 & 11.0\% & No \\
        \hline
    \end{tabular}
\end{table}



\begin{figure}[htbp]
    \centering
    \includegraphics[width=0.85\textwidth]{charts/exp2_pass1_combined.png}
    \caption{Experiment 2: Stochastic Environment - Pass 1 convergence}
    \label{fig:exp2_pass1_convergence}
\end{figure}


\subsection{Pass 2}


\begin{table}[htbp]
    \centering
    \caption{Experiment 2: Stochastic Environment - Pass 2}
    \label{tab:exp2_pass2}
    \begin{tabular}{llrrr}
        \hline
        Iteration & Agent & Cost & Liquidity & Accepted \\
        \hline
        Baseline & BANK\_A & \$498.00 & 50.0\% & --- \\
        Baseline & BANK\_B & \$498.00 & 50.0\% & --- \\
        0 & BANK\_A & \$65.58 & 5.0\% & Yes \\
        0 & BANK\_B & \$172.11 & 10.0\% & Yes \\
        1 & BANK\_A & \$119.52 & 12.0\% & No \\
        1 & BANK\_B & \$249.00 & 25.0\% & No \\
        2 & BANK\_A & \$71.99 & 6.5\% & No \\
        2 & BANK\_B & \$249.00 & 25.0\% & No \\
        3 & BANK\_A & \$73.31 & 7.0\% & No \\
        3 & BANK\_B & \$298.80 & 30.0\% & No \\
        4 & BANK\_A & \$99.99 & 10.0\% & No \\
        4 & BANK\_B & \$249.00 & 25.0\% & No \\
        5 & BANK\_A & \$81.44 & 8.0\% & No \\
        5 & BANK\_B & \$199.20 & 20.0\% & No \\
        \hline
    \end{tabular}
\end{table}



\begin{figure}[htbp]
    \centering
    \includegraphics[width=0.85\textwidth]{charts/exp2_pass2_combined.png}
    \caption{Experiment 2: Stochastic Environment - Pass 2 convergence}
    \label{fig:exp2_pass2_convergence}
\end{figure}


\subsection{Pass 3}


\begin{table}[htbp]
    \centering
    \caption{Experiment 2: Stochastic Environment - Pass 3}
    \label{tab:exp2_pass3}
    \begin{tabular}{llrrr}
        \hline
        Iteration & Agent & Cost & Liquidity & Accepted \\
        \hline
        Baseline & BANK\_A & \$498.00 & 50.0\% & --- \\
        Baseline & BANK\_B & \$498.00 & 50.0\% & --- \\
        0 & BANK\_A & \$99.99 & 10.0\% & Yes \\
        0 & BANK\_B & \$128.27 & 12.0\% & Yes \\
        1 & BANK\_A & \$109.56 & 11.0\% & No \\
        1 & BANK\_B & \$141.99 & 14.0\% & No \\
        2 & BANK\_A & \$95.07 & 9.5\% & Yes \\
        2 & BANK\_B & \$133.49 & 13.0\% & No \\
        3 & BANK\_A & \$97.95 & 9.8\% & No \\
        3 & BANK\_B & \$141.99 & 14.0\% & No \\
        4 & BANK\_A & \$93.03 & 9.3\% & Yes \\
        4 & BANK\_B & \$199.20 & 20.0\% & No \\
        5 & BANK\_A & \$92.07 & 9.2\% & Yes \\
        5 & BANK\_B & \$141.99 & 14.0\% & No \\
        6 & BANK\_A & \$93.99 & 9.4\% & No \\
        6 & BANK\_B & \$160.40 & 16.0\% & No \\
        7 & BANK\_A & \$93.99 & 9.4\% & No \\
        7 & BANK\_B & \$141.99 & 14.0\% & No \\
        \hline
    \end{tabular}
\end{table}



\begin{figure}[htbp]
    \centering
    \includegraphics[width=0.85\textwidth]{charts/exp2_pass3_combined.png}
    \caption{Experiment 2: Stochastic Environment - Pass 3 convergence}
    \label{fig:exp2_pass3_convergence}
\end{figure}




\section{Experiment 3: Symmetric Equilibrium - Detailed Results}
\label{app:exp3}

This appendix provides iteration-by-iteration results and convergence charts for
all three passes of experiment 3: symmetric equilibrium.

\subsection{Pass 1}


\begin{table}[htbp]
    \centering
    \caption{Experiment 3: Symmetric Equilibrium - Pass 1}
    \label{tab:exp3_pass1}
    \begin{tabular}{llrrr}
        \hline
        Iteration & Agent & Cost & Liquidity & Accepted \\
        \hline
        Baseline & BANK\_A & \$49.95 & 50.0\% & --- \\
        Baseline & BANK\_B & \$49.95 & 50.0\% & --- \\
        0 & BANK\_A & \$24.99 & 25.0\% & Yes \\
        0 & BANK\_B & \$29.97 & 30.0\% & Yes \\
        1 & BANK\_A & \$22.98 & 23.0\% & Yes \\
        1 & BANK\_B & \$19.98 & 20.0\% & Yes \\
        2 & BANK\_A & \$20.49 & 20.5\% & Yes \\
        2 & BANK\_B & \$120.00 & 0.0\% & No \\
        3 & BANK\_A & \$19.98 & 20.0\% & Yes \\
        3 & BANK\_B & \$120.00 & 0.0\% & No \\
        4 & BANK\_A & \$138.99 & 19.0\% & No \\
        4 & BANK\_B & \$129.99 & 10.0\% & No \\
        5 & BANK\_A & \$120.00 & 0.0\% & No \\
        5 & BANK\_B & \$120.00 & 0.0\% & No \\
        6 & BANK\_A & \$120.00 & 0.0\% & No \\
        6 & BANK\_B & \$138.99 & 19.0\% & No \\
        7 & BANK\_A & \$137.97 & 18.0\% & No \\
        7 & BANK\_B & \$129.99 & 10.0\% & No \\
        \hline
    \end{tabular}
\end{table}



\begin{figure}[htbp]
    \centering
    \includegraphics[width=0.85\textwidth]{charts/exp3_pass1_combined.png}
    \caption{Experiment 3: Symmetric Equilibrium - Pass 1 convergence}
    \label{fig:exp3_pass1_convergence}
\end{figure}


\subsection{Pass 2}


\begin{table}[htbp]
    \centering
    \caption{Experiment 3: Symmetric Equilibrium - Pass 2}
    \label{tab:exp3_pass2}
    \begin{tabular}{llrrr}
        \hline
        Iteration & Agent & Cost & Liquidity & Accepted \\
        \hline
        Baseline & BANK\_A & \$49.95 & 50.0\% & --- \\
        Baseline & BANK\_B & \$49.95 & 50.0\% & --- \\
        0 & BANK\_A & \$19.98 & 20.0\% & Yes \\
        0 & BANK\_B & \$20.97 & 21.0\% & Yes \\
        1 & BANK\_A & \$138.99 & 19.0\% & No \\
        1 & BANK\_B & \$20.07 & 20.1\% & Yes \\
        2 & BANK\_A & \$120.00 & 0.0\% & No \\
        2 & BANK\_B & \$19.98 & 20.0\% & Yes \\
        3 & BANK\_A & \$120.00 & 0.0\% & No \\
        3 & BANK\_B & \$120.99 & 1.0\% & No \\
        4 & BANK\_A & \$138.99 & 19.0\% & No \\
        4 & BANK\_B & \$120.00 & 0.0\% & No \\
        5 & BANK\_A & \$120.00 & 0.0\% & No \\
        5 & BANK\_B & \$120.00 & 0.0\% & No \\
        \hline
    \end{tabular}
\end{table}



\begin{figure}[htbp]
    \centering
    \includegraphics[width=0.85\textwidth]{charts/exp3_pass2_combined.png}
    \caption{Experiment 3: Symmetric Equilibrium - Pass 2 convergence}
    \label{fig:exp3_pass2_convergence}
\end{figure}


\subsection{Pass 3}


\begin{table}[htbp]
    \centering
    \caption{Experiment 3: Symmetric Equilibrium - Pass 3}
    \label{tab:exp3_pass3}
    \begin{tabular}{llrrr}
        \hline
        Iteration & Agent & Cost & Liquidity & Accepted \\
        \hline
        Baseline & BANK\_A & \$49.95 & 50.0\% & --- \\
        Baseline & BANK\_B & \$49.95 & 50.0\% & --- \\
        0 & BANK\_A & \$21.99 & 22.0\% & Yes \\
        0 & BANK\_B & \$19.98 & 20.0\% & Yes \\
        1 & BANK\_A & \$19.98 & 20.0\% & Yes \\
        1 & BANK\_B & \$120.00 & 0.0\% & No \\
        2 & BANK\_A & \$120.00 & 0.0\% & No \\
        2 & BANK\_B & \$120.00 & 0.0\% & No \\
        3 & BANK\_A & \$120.00 & 0.0\% & No \\
        3 & BANK\_B & \$125.01 & 5.0\% & No \\
        4 & BANK\_A & \$135.00 & 15.0\% & No \\
        4 & BANK\_B & \$120.00 & 0.0\% & No \\
        5 & BANK\_A & \$138.99 & 19.0\% & No \\
        5 & BANK\_B & \$120.00 & 0.0\% & No \\
        \hline
    \end{tabular}
\end{table}



\begin{figure}[htbp]
    \centering
    \includegraphics[width=0.85\textwidth]{charts/exp3_pass3_combined.png}
    \caption{Experiment 3: Symmetric Equilibrium - Pass 3 convergence}
    \label{fig:exp3_pass3_convergence}
\end{figure}




\section{Bootstrap Evaluation Statistics}
\label{app:bootstrap}

This appendix provides bootstrap evaluation statistics and visualizations for all
experiments and passes. Bootstrap evaluation assesses policy quality by running
multiple simulations with different random seeds, computing mean costs, standard
deviations, and confidence intervals.

\subsection{Experiment 1}

\subsubsection{Pass 1}


\begin{table}[htbp]
    \centering
    \caption{Experiment 1 Bootstrap Statistics - Pass 1}
    \label{tab:exp1_bootstrap_pass1}
    \begin{tabular}{lrrrr}
        \hline
        Agent & Mean Cost & Std Dev & 95\% CI & Samples \\
        \hline
        BANK\_A & \$0.00 & \$0.00 & [\$0.00, \$0.00] & 1 \\
        BANK\_B & \$25.10 & \$0.00 & [\$25.10, \$25.10] & 1 \\
        \hline
    \end{tabular}
\end{table}



\begin{figure}[htbp]
    \centering
    \includegraphics[width=0.8\textwidth]{charts/exp1_pass1_ci_width.png}
    \caption{Experiment 1 Pass 1: CI width comparison across iterations}
    \label{fig:exp1_pass1_ci_width}
\end{figure}



\begin{figure}[htbp]
    \centering
    \includegraphics[width=0.8\textwidth]{charts/exp1_pass1_variance_evolution.png}
    \caption{Experiment 1 Pass 1: Standard deviation evolution}
    \label{fig:exp1_pass1_variance}
\end{figure}



\begin{figure}[htbp]
    \centering
    \includegraphics[width=0.8\textwidth]{charts/exp1_pass1_sample_distribution.png}
    \caption{Experiment 1 Pass 1: Bootstrap sample distribution at convergence}
    \label{fig:exp1_pass1_samples}
\end{figure}


\subsubsection{Pass 2}


\begin{table}[htbp]
    \centering
    \caption{Experiment 1 Bootstrap Statistics - Pass 2}
    \label{tab:exp1_bootstrap_pass2}
    \begin{tabular}{lrrrr}
        \hline
        Agent & Mean Cost & Std Dev & 95\% CI & Samples \\
        \hline
        BANK\_A & \$75.00 & \$0.00 & [\$75.00, \$75.00] & 1 \\
        BANK\_B & \$26.00 & \$0.00 & [\$26.00, \$26.00] & 1 \\
        \hline
    \end{tabular}
\end{table}



\begin{figure}[htbp]
    \centering
    \includegraphics[width=0.8\textwidth]{charts/exp1_pass2_ci_width.png}
    \caption{Experiment 1 Pass 2: CI width comparison across iterations}
    \label{fig:exp1_pass2_ci_width}
\end{figure}



\begin{figure}[htbp]
    \centering
    \includegraphics[width=0.8\textwidth]{charts/exp1_pass2_variance_evolution.png}
    \caption{Experiment 1 Pass 2: Standard deviation evolution}
    \label{fig:exp1_pass2_variance}
\end{figure}



\begin{figure}[htbp]
    \centering
    \includegraphics[width=0.8\textwidth]{charts/exp1_pass2_sample_distribution.png}
    \caption{Experiment 1 Pass 2: Bootstrap sample distribution at convergence}
    \label{fig:exp1_pass2_samples}
\end{figure}


\subsubsection{Pass 3}


\begin{table}[htbp]
    \centering
    \caption{Experiment 1 Bootstrap Statistics - Pass 3}
    \label{tab:exp1_bootstrap_pass3}
    \begin{tabular}{lrrrr}
        \hline
        Agent & Mean Cost & Std Dev & 95\% CI & Samples \\
        \hline
        BANK\_A & \$0.00 & \$0.00 & [\$0.00, \$0.00] & 1 \\
        BANK\_B & \$25.10 & \$0.00 & [\$25.10, \$25.10] & 1 \\
        \hline
    \end{tabular}
\end{table}



\begin{figure}[htbp]
    \centering
    \includegraphics[width=0.8\textwidth]{charts/exp1_pass3_ci_width.png}
    \caption{Experiment 1 Pass 3: CI width comparison across iterations}
    \label{fig:exp1_pass3_ci_width}
\end{figure}



\begin{figure}[htbp]
    \centering
    \includegraphics[width=0.8\textwidth]{charts/exp1_pass3_variance_evolution.png}
    \caption{Experiment 1 Pass 3: Standard deviation evolution}
    \label{fig:exp1_pass3_variance}
\end{figure}



\begin{figure}[htbp]
    \centering
    \includegraphics[width=0.8\textwidth]{charts/exp1_pass3_sample_distribution.png}
    \caption{Experiment 1 Pass 3: Bootstrap sample distribution at convergence}
    \label{fig:exp1_pass3_samples}
\end{figure}


\subsection{Experiment 2}

\subsubsection{Pass 1}


\begin{table}[htbp]
    \centering
    \caption{Experiment 2 Bootstrap Statistics - Pass 1}
    \label{tab:exp2_bootstrap_pass1}
    \begin{tabular}{lrrrr}
        \hline
        Agent & Mean Cost & Std Dev & 95\% CI & Samples \\
        \hline
        BANK\_A & \$164.40 & \$0.00 & [\$164.40, \$164.40] & 50 \\
        BANK\_B & \$130.85 & \$69.51 & [\$110.99, \$150.72] & 50 \\
        \hline
    \end{tabular}
\end{table}



\begin{figure}[htbp]
    \centering
    \includegraphics[width=0.8\textwidth]{charts/exp2_pass1_ci_width.png}
    \caption{Experiment 2 Pass 1: CI width comparison across iterations}
    \label{fig:exp2_pass1_ci_width}
\end{figure}



\begin{figure}[htbp]
    \centering
    \includegraphics[width=0.8\textwidth]{charts/exp2_pass1_variance_evolution.png}
    \caption{Experiment 2 Pass 1: Standard deviation evolution}
    \label{fig:exp2_pass1_variance}
\end{figure}



\begin{figure}[htbp]
    \centering
    \includegraphics[width=0.8\textwidth]{charts/exp2_pass1_sample_distribution.png}
    \caption{Experiment 2 Pass 1: Bootstrap sample distribution at convergence}
    \label{fig:exp2_pass1_samples}
\end{figure}


\subsubsection{Pass 2}


\begin{table}[htbp]
    \centering
    \caption{Experiment 2 Bootstrap Statistics - Pass 2}
    \label{tab:exp2_bootstrap_pass2}
    \begin{tabular}{lrrrr}
        \hline
        Agent & Mean Cost & Std Dev & 95\% CI & Samples \\
        \hline
        BANK\_A & \$81.44 & \$12.45 & [\$77.88, \$85.00] & 50 \\
        BANK\_B & \$199.20 & \$0.00 & [\$199.20, \$199.20] & 50 \\
        \hline
    \end{tabular}
\end{table}



\begin{figure}[htbp]
    \centering
    \includegraphics[width=0.8\textwidth]{charts/exp2_pass2_ci_width.png}
    \caption{Experiment 2 Pass 2: CI width comparison across iterations}
    \label{fig:exp2_pass2_ci_width}
\end{figure}



\begin{figure}[htbp]
    \centering
    \includegraphics[width=0.8\textwidth]{charts/exp2_pass2_variance_evolution.png}
    \caption{Experiment 2 Pass 2: Standard deviation evolution}
    \label{fig:exp2_pass2_variance}
\end{figure}



\begin{figure}[htbp]
    \centering
    \includegraphics[width=0.8\textwidth]{charts/exp2_pass2_sample_distribution.png}
    \caption{Experiment 2 Pass 2: Bootstrap sample distribution at convergence}
    \label{fig:exp2_pass2_samples}
\end{figure}


\subsubsection{Pass 3}


\begin{table}[htbp]
    \centering
    \caption{Experiment 2 Bootstrap Statistics - Pass 3}
    \label{tab:exp2_bootstrap_pass3}
    \begin{tabular}{lrrrr}
        \hline
        Agent & Mean Cost & Std Dev & 95\% CI & Samples \\
        \hline
        BANK\_A & \$93.99 & \$2.77 & [\$93.19, \$94.78] & 50 \\
        BANK\_B & \$141.99 & \$18.09 & [\$136.82, \$147.17] & 50 \\
        \hline
    \end{tabular}
\end{table}



\begin{figure}[htbp]
    \centering
    \includegraphics[width=0.8\textwidth]{charts/exp2_pass3_ci_width.png}
    \caption{Experiment 2 Pass 3: CI width comparison across iterations}
    \label{fig:exp2_pass3_ci_width}
\end{figure}



\begin{figure}[htbp]
    \centering
    \includegraphics[width=0.8\textwidth]{charts/exp2_pass3_variance_evolution.png}
    \caption{Experiment 2 Pass 3: Standard deviation evolution}
    \label{fig:exp2_pass3_variance}
\end{figure}



\begin{figure}[htbp]
    \centering
    \includegraphics[width=0.8\textwidth]{charts/exp2_pass3_sample_distribution.png}
    \caption{Experiment 2 Pass 3: Bootstrap sample distribution at convergence}
    \label{fig:exp2_pass3_samples}
\end{figure}


\subsection{Experiment 3}

\subsubsection{Pass 1}


\begin{table}[htbp]
    \centering
    \caption{Experiment 3 Bootstrap Statistics - Pass 1}
    \label{tab:exp3_bootstrap_pass1}
    \begin{tabular}{lrrrr}
        \hline
        Agent & Mean Cost & Std Dev & 95\% CI & Samples \\
        \hline
        BANK\_A & \$137.97 & \$0.00 & [\$137.97, \$137.97] & 1 \\
        BANK\_B & \$129.99 & \$0.00 & [\$129.99, \$129.99] & 1 \\
        \hline
    \end{tabular}
\end{table}



\begin{figure}[htbp]
    \centering
    \includegraphics[width=0.8\textwidth]{charts/exp3_pass1_ci_width.png}
    \caption{Experiment 3 Pass 1: CI width comparison across iterations}
    \label{fig:exp3_pass1_ci_width}
\end{figure}



\begin{figure}[htbp]
    \centering
    \includegraphics[width=0.8\textwidth]{charts/exp3_pass1_variance_evolution.png}
    \caption{Experiment 3 Pass 1: Standard deviation evolution}
    \label{fig:exp3_pass1_variance}
\end{figure}



\begin{figure}[htbp]
    \centering
    \includegraphics[width=0.8\textwidth]{charts/exp3_pass1_sample_distribution.png}
    \caption{Experiment 3 Pass 1: Bootstrap sample distribution at convergence}
    \label{fig:exp3_pass1_samples}
\end{figure}


\subsubsection{Pass 2}


\begin{table}[htbp]
    \centering
    \caption{Experiment 3 Bootstrap Statistics - Pass 2}
    \label{tab:exp3_bootstrap_pass2}
    \begin{tabular}{lrrrr}
        \hline
        Agent & Mean Cost & Std Dev & 95\% CI & Samples \\
        \hline
        BANK\_A & \$120.00 & \$0.00 & [\$120.00, \$120.00] & 1 \\
        BANK\_B & \$120.00 & \$0.00 & [\$120.00, \$120.00] & 1 \\
        \hline
    \end{tabular}
\end{table}



\begin{figure}[htbp]
    \centering
    \includegraphics[width=0.8\textwidth]{charts/exp3_pass2_ci_width.png}
    \caption{Experiment 3 Pass 2: CI width comparison across iterations}
    \label{fig:exp3_pass2_ci_width}
\end{figure}



\begin{figure}[htbp]
    \centering
    \includegraphics[width=0.8\textwidth]{charts/exp3_pass2_variance_evolution.png}
    \caption{Experiment 3 Pass 2: Standard deviation evolution}
    \label{fig:exp3_pass2_variance}
\end{figure}



\begin{figure}[htbp]
    \centering
    \includegraphics[width=0.8\textwidth]{charts/exp3_pass2_sample_distribution.png}
    \caption{Experiment 3 Pass 2: Bootstrap sample distribution at convergence}
    \label{fig:exp3_pass2_samples}
\end{figure}


\subsubsection{Pass 3}


\begin{table}[htbp]
    \centering
    \caption{Experiment 3 Bootstrap Statistics - Pass 3}
    \label{tab:exp3_bootstrap_pass3}
    \begin{tabular}{lrrrr}
        \hline
        Agent & Mean Cost & Std Dev & 95\% CI & Samples \\
        \hline
        BANK\_A & \$138.99 & \$0.00 & [\$138.99, \$138.99] & 1 \\
        BANK\_B & \$120.00 & \$0.00 & [\$120.00, \$120.00] & 1 \\
        \hline
    \end{tabular}
\end{table}



\begin{figure}[htbp]
    \centering
    \includegraphics[width=0.8\textwidth]{charts/exp3_pass3_ci_width.png}
    \caption{Experiment 3 Pass 3: CI width comparison across iterations}
    \label{fig:exp3_pass3_ci_width}
\end{figure}



\begin{figure}[htbp]
    \centering
    \includegraphics[width=0.8\textwidth]{charts/exp3_pass3_variance_evolution.png}
    \caption{Experiment 3 Pass 3: Standard deviation evolution}
    \label{fig:exp3_pass3_variance}
\end{figure}



\begin{figure}[htbp]
    \centering
    \includegraphics[width=0.8\textwidth]{charts/exp3_pass3_sample_distribution.png}
    \caption{Experiment 3 Pass 3: Bootstrap sample distribution at convergence}
    \label{fig:exp3_pass3_samples}
\end{figure}




\section{LLM Prompt Audit}
\label{app:prompt_audit}

This appendix documents the LLM prompts used for policy learning and provides
an audit of potential information leakage or bias.

\subsection{Agent Prompt Structure}

Each agent receives the following information each iteration:

\begin{enumerate}
    \item \textbf{Current state}: Own balance, counterparty balance, pending transactions
    \item \textbf{Cost history}: Previous iteration costs for both agents
    \item \textbf{Policy parameters}: Current liquidity fraction setting
    \item \textbf{Scenario context}: Cost structure, time horizon, settlement rules
\end{enumerate}

\subsection{Information Boundaries}

The prompt design ensures:

\begin{itemize}
    \item Agents cannot access counterparty reasoning or internal computations
    \item Historical data is limited to observable outcomes (costs, acceptances)
    \item No direct communication channel between agents
    \item Scenario parameters are identically presented to both agents
\end{itemize}

\subsection{Prompt Sanitization}

All prompts are sanitized to remove:

\begin{itemize}
    \item References to "optimal" or "theoretical" equilibria
    \item Hints about expected asymmetric vs symmetric outcomes
    \item Explicit game-theoretic terminology (Nash, Pareto, etc.)
    \item Training data leakage from prior experiments
\end{itemize}

\subsection{Audit Conclusions}

Based on our review:

\begin{enumerate}
    \item \textbf{No information leakage}: Agents discover equilibria through
    observed costs, not prompt hints.

    \item \textbf{Fair competition}: Both agents receive identically structured
    prompts with symmetric information access.

    \item \textbf{Reproducibility}: The same prompts with identical seeds produce
    identical learning trajectories.

    \item \textbf{Balance leakage}: While agents can observe counterparty balance,
    this reflects realistic RTGS transparency. Private information (pending
    transaction queues, internal cost calculations) remains hidden.
\end{enumerate}

The experiment results demonstrate genuine strategic learning rather than
prompt-induced behavior, as evidenced by:

\begin{itemize}
    \item Gradual convergence over multiple iterations
    \item Different equilibria across different cost structures
    \item Consistent results across independent passes
\end{itemize}



\end{document}
