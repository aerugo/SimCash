\documentclass[11pt]{article}

% Page layout
\usepackage[margin=1in]{geometry}

% Typography
\usepackage[utf8]{inputenc}
\usepackage[T1]{fontenc}
\usepackage{microtype}

% Math
\usepackage{amsmath}
\usepackage{amssymb}

% Tables and figures
\usepackage{booktabs}
\usepackage{graphicx}
\usepackage{float}
\usepackage{longtable}

% Lists
\usepackage{enumitem}

% Colors for notice box
\usepackage{xcolor}

% Hyperlinks
\usepackage{hyperref}
\hypersetup{
    colorlinks=true,
    linkcolor=blue,
    citecolor=blue,
    urlcolor=blue
}

% Title and author
\title{Discovering Equilibrium-like Behavior with LLM Agents: A Payment Systems Case Study}
\author{Hugi Aegisberg}
\date{\today}

\begin{document}

\maketitle

\vspace{1em}
\begin{center}
\colorbox{red!90}{\parbox{0.92\textwidth}{\color{white}\centering
\textbf{\Large DO NOT CIRCULATE}\\[0.5em]
\normalsize
This is a working document and is not intended for distribution.\\[0.3em]
This paper and the accompanying SimCash codebase were developed with assistance from\\
\textbf{Claude 4.5 Opus} (Anthropic) and \textbf{GPT-5.2} (OpenAI) as coding and\\
writing tools. All experimental design, analysis, and interpretation are the author's own.
}}
\end{center}

\vspace{0.5em}
\begin{center}
\colorbox{green!70!black}{\parbox{0.92\textwidth}{\color{white}
\textbf{About This Document}\\[0.3em]
This is a \textbf{research proposal} presenting methodology and preliminary findings
to potential collaborators. All tables, figures, and statistics are programmatically
generated from experiment databases (DuckDB $\to$ DataProvider $\to$ LaTeX/charts),
eliminating manual transcription. The accompanying text is written by an AI assistant
(Claude) following author guidance on structure and conclusions.\\[0.3em]
SimCash is a hybrid Rust/Python payment system simulator with deterministic replay,
configurable policies, and multiple settlement mechanisms (RTGS, queues, LSM). The
experiment runner uses LLM agents to iteratively optimize policies through natural
language reasoning, enabling research into multi-agent coordination in financial infrastructure.
}}
\end{center}
\vspace{1em}


\begin{abstract}
Can Large Language Models discover equilibrium-like behavior through strategic reasoning alone?
We explore this question using payment system liquidity management---a domain where
banks must balance the cost of holding reserves against settlement delays, and where
game-theoretic equilibria are well-characterized but difficult to find without explicit
modeling.

We present SimCash, a framework where LLM agents optimize liquidity policies through
natural language deliberation under information isolation: each agent observes only
its own costs and transaction history, never counterparty strategies. Through 9
independent runs across 3 scenarios adapted from Castro et al., agents
reliably converge to stable policy profiles (100\% success in 9
preliminary runs, mean 22.1 iterations). However, outcome selection exhibits
path-dependence: in deterministic symmetric games, agents consistently converge to \textit{asymmetric}
free-rider outcomes, while stochastic environments produce near-symmetric equilibria---with the
identity of the free-rider (when one emerges) determined by early exploration rather than cost structure.

These preliminary findings suggest that LLM-based policy optimization can discover
equilibrium-like behavior without explicit game-theoretic modeling---though we do not
formally verify the Nash condition (no unilateral deviation check). They also reveal
that sequential best-response dynamics in multi-agent LLM systems may systematically
favor asymmetric outcomes. Our small sample (9 runs) requires validation
through expanded experimentation before drawing strong conclusions.
\end{abstract}



\section{Introduction}

Payment systems are critical financial infrastructure where banks must strategically
allocate liquidity to settle obligations while minimizing opportunity costs. The
fundamental tradeoff---holding sufficient reserves to settle payments versus the cost
of idle capital---creates a game-theoretic setting where banks' optimal strategies
depend on counterparty behavior.

Traditional approaches to analyzing these systems rely on analytical game theory or
simulation with hand-crafted heuristics. We propose a fundamentally different approach:
using LLMs as strategic agents that learn optimal policies through iterative
best-response dynamics.

\subsection{Contributions}

\begin{enumerate}
    \item \textbf{SimCash Framework}: A hybrid Rust-Python simulator with LLM-based
    policy optimization
    \item \textbf{Empirical Comparison}: Comparison with Castro et al.'s theoretical
    predictions, showing partial alignment and systematic deviations
    \item \textbf{Reproducibility Analysis}: 9 independent runs demonstrating consistent
    convergence
    \item \textbf{Bootstrap Evaluation}: Methodology for handling stochastic payment
    arrivals
\end{enumerate}



\section{The SimCash Framework}
\label{sec:methods}

\subsection{Simulation Engine}

SimCash uses a discrete-time simulation where:
\begin{itemize}
    \item Time proceeds in \textbf{ticks} (atomic time units)
    \item Banks hold \textbf{balances} in settlement accounts
    \item \textbf{Transactions} arrive with amounts, counterparties, and deadlines
    \item Settlement follows RTGS (Real-Time Gross Settlement) rules
\end{itemize}

\subsection{Cost Function}

Agent costs comprise:
\begin{itemize}
    \item \textbf{Liquidity opportunity cost}: Proportional to allocated reserves
    \item \textbf{Delay penalty}: Accumulated per tick for pending transactions
    \item \textbf{Deadline penalty}: Incurred when transactions become overdue
    \item \textbf{End-of-day penalty}: Large cost for unsettled transactions at day end
    \item \textbf{Overdraft cost}: Fee for negative balance (basis points per day)
\end{itemize}

\subsection{LLM Policy Optimization}

The key innovation is using LLMs to propose policy parameters. At each iteration:

\begin{enumerate}
    \item \textbf{Context Construction}: Agent receives its own policy, filtered simulation trace, and cost history (see Section~\ref{sec:prompt_anatomy})
    \item \textbf{LLM Proposal}: Agent proposes new \texttt{initial\_liquidity\_fraction} parameter
    \item \textbf{Evaluation}: Run simulation(s) with proposed policy
    \item \textbf{Update}: Apply mode-specific acceptance rule (see below)
    \item \textbf{Convergence Check}: Stable \texttt{initial\_liquidity\_fraction} (temporal) or multi-criteria cost stability (bootstrap) over 5 iterations
\end{enumerate}

\subsection{Optimization Prompt Anatomy}
\label{sec:prompt_anatomy}

A critical aspect of our framework is the \textbf{strict information isolation} between agents.
Each agent receives a two-part prompt with no access to counterparty information.

\subsubsection{System Prompt (Shared)}

The system prompt is identical for all agents and provides domain context:
\begin{itemize}
    \item RTGS mechanics and queuing behavior
    \item Cost structure: overdraft, delay, deadline, and EOD penalties
    \item Policy tree architecture: JSON schema for valid policies
    \item Optimization guidance: e.g., ``lower liquidity reduces holding costs but increases delay risk; find the balance that minimizes total cost''
\end{itemize}

\subsubsection{User Prompt (Agent-Specific)}

The user prompt is constructed individually for each agent and contains \textbf{only} information
about that agent's own experience:

\begin{enumerate}
    \item \textbf{Performance metrics from past iterations}: Agent's own mean cost, standard deviation, settlement rate
    \item \textbf{Current policy}: Agent's own \texttt{initial\_liquidity\_fraction} parameter
    \item \textbf{Cost breakdown}: Agent's own costs by type (delay, overdraft, penalties)
    \item \textbf{Simulation trace}: Filtered event log showing \textbf{only}:
    \begin{itemize}
        \item Outgoing transactions FROM this agent
        \item Incoming payments TO this agent
        \item Agent's own policy decisions (Submit, Hold, etc.)
        \item Agent's own balance changes (for settlements it initiated)
    \end{itemize}
    \item \textbf{Iteration history}: Agent's own cost trajectory across iterations
\end{enumerate}

\subsubsection{Information Isolation}

The prompt explicitly excludes all counterparty information:
\begin{itemize}
    \item \textbf{No counterparty balances}: Agents cannot observe opponent's reserves
    \item \textbf{No counterparty policies}: Agents cannot see opponent's liquidity fraction
    \item \textbf{No counterparty costs}: Agents cannot observe opponent's cost breakdown
    \item \textbf{No third-party events}: Transactions not involving this agent are filtered
\end{itemize}

This isolation is enforced programmatically by the \texttt{filter\_events\_for\_agent()} function.
The only ``signal'' about counterparty behavior comes from \textit{incoming payments}---a realistic
level of transparency in actual RTGS systems where participants observe settlement messages but not
others' internal liquidity positions.

Crucially, agents receive \textbf{transaction events from the current iteration} alongside
\textbf{performance metrics from past iterations}, but are never informed that the environment
is stationary. The agent is not told that all iterations use identical transaction schedules
(Experiments 1 and 3) or identical stochastic parameters (Experiment 2). From the agent's
perspective, each iteration could involve a different payment environment---any regularity
must be inferred from observed patterns rather than assumed from explicit knowledge of the
experimental design.

\subsection{Evaluation Modes}

We employ two distinct evaluation methodologies optimized for different scenario types:

\subsubsection{Deterministic-Temporal Mode (Experiments 1 \& 3)}

For scenarios with fixed payment schedules, we use \textbf{temporal policy stability} to identify stable policy profiles:

\begin{itemize}
    \item \textbf{Single simulation} per iteration with deterministic arrivals
    \item \textbf{Unconditional acceptance}: All LLM-proposed policies are accepted immediately, regardless of cost impact
    \item \textbf{Rationale}: Cost-based rejection would cause oscillation in multi-agent settings where counterparty policies also change each iteration
    \item \textbf{Convergence criterion}: Both agents' \texttt{initial\_liquidity\_fraction} stable (relative change $\leq$ 5\%) for 5 consecutive iterations, indicating policy stability
\end{itemize}

\subsubsection{Bootstrap Mode (Experiment 2)}

For stochastic scenarios, we use \textbf{per-iteration bootstrap resampling} with pre-generated seeds
for deterministic reproducibility.

\paragraph{Seed Hierarchy.}
Seeds are generated deterministically from a single master seed:
\begin{enumerate}
    \item \textbf{Master seed}: Fixed per experiment for reproducibility
    \item \textbf{Iteration seeds}: 50 seeds derived from master (one per iteration per agent)
    \item \textbf{Bootstrap seeds}: 50 seeds derived from each iteration seed (one per sample)
\end{enumerate}
This produces $50 \times 50 = 2{,}500$ unique seeds per agent, ensuring full stochastic exploration
while maintaining paired comparison integrity within iterations.

\paragraph{Per-Iteration Evaluation.}
Each iteration proceeds as follows:
\begin{enumerate}
    \item \textbf{Context simulation}: Run full simulation with the iteration-specific seed,
    generating a unique transaction history for this iteration (different stochastic arrivals
    than other iterations)
    \item \textbf{Bootstrap sampling}: Generate 50 transaction schedules by resampling with
    replacement from this iteration's history, preserving settlement offset distributions
    \item \textbf{Paired comparison}: Evaluate both old and new policy on the \textit{same} 50 samples,
    computing $\delta_i = \text{cost}_{\text{old},i} - \text{cost}_{\text{new},i}$
    \item \textbf{Acceptance}: Accept if $\sum_i \delta_i > 0$ (net improvement across samples)
\end{enumerate}

The paired comparison on identical samples eliminates sample-to-sample variance, enabling detection
of smaller policy improvements than unpaired comparison would allow.

\paragraph{Convergence Criterion.}
Three criteria must ALL be satisfied over a 5-iteration window:
\begin{enumerate}
    \item Coefficient of variation below 3\% (cost stability across iteration means)
    \item Mann-Kendall test $p > 0.05$ (no significant trend---with only 5 iterations, this is a heuristic)
    \item Regret below 10\% (current cost within 10\% of best observed)
\end{enumerate}

\textit{Note: CV is computed over iteration means, not individual bootstrap samples.}

\subsection{Experimental Setup}

We implement three canonical scenarios from Castro et al.\ (2025):

\textbf{Experiment 1: 2-Period Deterministic} (Deterministic-Temporal Mode)
\begin{itemize}
    \item 2 ticks per day
    \item Asymmetric payment demands: $P^A = [0, 0.15]$, $P^B = [0.15, 0.05]$
    \item Bank A sends 0.15$B$ at tick 1; Bank B sends 0.15$B$ at tick 0, 0.05$B$ at tick 1
    \item Expected equilibrium: Asymmetric (A=0\%, B=20\%)
\end{itemize}

\textbf{Experiment 2: 12-Period Stochastic} (Bootstrap Mode)
\begin{itemize}
    \item 12 ticks per day
    \item Poisson arrivals ($\lambda=2.0$/tick), LogNormal amounts ($\mu$=10k, $\sigma$=5k)
    \item Expected equilibrium: Both agents in 10--30\% range
\end{itemize}

\textbf{Experiment 3: 3-Period Symmetric} (Deterministic-Temporal Mode)
\begin{itemize}
    \item 3 ticks per day
    \item Symmetric payment demands: $P^A = P^B = [0.2, 0.2, 0]$
    \item Expected equilibrium: Symmetric ($\sim$20\%)
\end{itemize}

\subsection{Comparison with Castro et al.\ (2025)}

Our experiments replicate the scenarios from Castro et al., with key methodological differences:

\begin{itemize}
    \item \textbf{Optimization method}: Castro et al.\ use REINFORCE (policy gradient with neural networks trained over 50--100 episodes); we use LLM-based policy optimization with natural language reasoning
    \item \textbf{Action representation}: Castro et al.\ discretize $x_0 \in \{0, 0.05, \ldots, 1\}$ (21 values); our LLM proposes continuous values in $[0,1]$
    \item \textbf{Convergence}: Castro et al.\ monitor training loss curves; we use explicit policy stability (temporal) or multi-criteria statistical convergence (bootstrap) detection
    \item \textbf{Multi-agent dynamics}: Castro et al.\ train two neural networks simultaneously with gradient updates; we optimize agents sequentially within each iteration, checking for mutual best-response stability
\end{itemize}

\subsection{LLM Configuration}

\begin{itemize}
    \item Model: \texttt{openai:gpt-5.2}
    \item Reasoning effort: \texttt{high}
    \item Temperature: 0.5
    \item Max iterations: 50 per pass
\end{itemize}

Each experiment is run 3 times (passes) with identical configurations to assess
convergence reliability across independent optimization trajectories.



\section{Results}
\label{sec:results}

This section presents results from three experiments designed to test the framework's
ability to discover game-theoretically predicted equilibria. Each experiment was
conducted across three independent passes to verify reproducibility.

\subsection{Convergence Summary}

Table~\ref{tab:convergence_stats} summarizes convergence behavior across all experiments.
All passes achieved convergence, with mean iterations ranging from 7.0
(Experiment 3) to 49.0 (Experiment 2).


\begin{table}[htbp]
    \centering
    \caption{Convergence statistics across all experiments}
    \label{tab:convergence_stats}
    \begin{tabular}{lrrrr}
        \hline
        Experiment & Mean Iters & Min & Max & Conv. Rate \\
        \hline
        EXP1 & 10.3 & 8 & 12 & 100.0\% \\
        EXP2 & 49.0 & 49 & 49 & 100.0\% \\
        EXP3 & 7.0 & 7 & 7 & 100.0\% \\
        \hline
    \end{tabular}
\end{table}


\subsection{Experiment 1: Asymmetric Equilibrium}

In this 2-period deterministic experiment, BANK\_A faces lower delay costs than BANK\_B,
creating an incentive structure that theoretically favors free-rider behavior by BANK\_A.


\begin{table}[htbp]
    \centering
    \caption{Experiment 1: Iteration-by-iteration results (Pass 1)}
    \label{tab:exp1_results}
    \begin{tabular}{llrr}
        \hline
        Iteration & Agent & Cost & Liquidity \\
        \hline
        Baseline & BANK\_A & \$50.00 & 50.0\% \\
        Baseline & BANK\_B & \$50.00 & 50.0\% \\
        0 & BANK\_A & \$50.00 & 50.0\% \\
        0 & BANK\_B & \$50.00 & 50.0\% \\
        1 & BANK\_A & \$20.00 & 20.0\% \\
        1 & BANK\_B & \$30.00 & 30.0\% \\
        2 & BANK\_A & \$10.00 & 10.0\% \\
        2 & BANK\_B & \$20.00 & 20.0\% \\
        3 & BANK\_A & \$5.00 & 5.0\% \\
        3 & BANK\_B & \$28.00 & 18.0\% \\
        4 & BANK\_A & \$0.00 & 0.0\% \\
        4 & BANK\_B & \$20.00 & 20.0\% \\
        5 & BANK\_A & \$0.10 & 0.1\% \\
        5 & BANK\_B & \$27.00 & 17.0\% \\
        6 & BANK\_A & \$0.10 & 0.1\% \\
        6 & BANK\_B & \$27.00 & 17.0\% \\
        7 & BANK\_A & \$0.10 & 0.1\% \\
        7 & BANK\_B & \$27.00 & 17.0\% \\
        8 & BANK\_A & \$0.10 & 0.1\% \\
        8 & BANK\_B & \$27.00 & 17.0\% \\
        \hline
    \end{tabular}
\end{table}



\begin{figure}[htbp]
    \centering
    \includegraphics[width=0.9\textwidth]{charts/exp1_pass1_combined.png}
    \caption{Experiment 1: Convergence of both agents toward asymmetric stable outcome}
    \label{fig:exp1_convergence}
\end{figure}


The agents converged after 8 iterations in Pass 1 to an asymmetric stable outcome:
\begin{itemize}
    \item BANK\_A achieved \$0.10 cost with 0.1\% liquidity allocation
    \item BANK\_B achieved \$27.00 cost with 17.0\% liquidity allocation
\end{itemize}

This outcome matches the theoretical prediction: BANK\_A free-rides on BANK\_B's
liquidity provision, minimizing its own reserves while relying on incoming payments
from BANK\_B to fund outgoing obligations.

Table~\ref{tab:exp1_summary} summarizes convergence across all three passes.
Notably, \textbf{Pass 3 exhibited coordination failure}: BANK\_B adopted a
zero-liquidity strategy, but unlike Passes 1--2 where BANK\_A successfully free-rode,
here BANK\_A's low liquidity (1.8\%) was insufficient to compensate. Both agents
incurred high costs (\$31.78 and \$70.00 respectively), with total cost nearly 4$\times$
that of the efficient outcome. This demonstrates that the learning dynamics can converge to
multiple stable outcomes with substantially different efficiency properties---and that LLM agents
do not always find the Pareto-optimal outcome.


\begin{table}[htbp]
    \centering
    \caption{Experiment 1: Summary across all passes}
    \label{tab:exp1_summary}
    \begin{tabular}{ccrrrrrr}
        \hline
        Pass & Iterations & BANK\_A Liq. & BANK\_B Liq. & BANK\_A Cost & BANK\_B Cost & Total Cost \\
        \hline
        1 & 8 & 0.1\% & 17.0\% & \$0.10 & \$27.00 & \$27.10 \\
        2 & 12 & 0.0\% & 17.9\% & \$0.00 & \$27.90 & \$27.90 \\
        3 & 11 & 1.8\% & 0.0\% & \$31.78 & \$70.00 & \$101.78 \\
        \hline
    \end{tabular}
\end{table}


\subsection{Experiment 2: Stochastic Environment}

Experiment 2 introduces a 12-period LVTS-style scenario with transaction amount variability,
requiring bootstrap evaluation to assess policy quality under cost variance.

All three passes achieved convergence at iteration 49 (the maximum allowed). The strict
bootstrap convergence criteria---requiring CV $<$ 3\%, no significant trend, and regret $<$ 10\%
over a 5-iteration window---demanded extended observation to confidently identify stable policies
in this stochastic environment. We present Pass 2 as the exemplar run.


\begin{longtable}{llrr}
    \caption{Experiment 2: Iteration-by-iteration results (Pass 2)} \label{tab:exp2_results} \\
    \hline
    Iteration & Agent & Cost & Liquidity \\
    \hline
    \endfirsthead

    \multicolumn{4}{c}{\tablename\ \thetable{} -- continued from previous page} \\
    \hline
    Iteration & Agent & Cost & Liquidity \\
    \hline
    \endhead

    \hline
    \multicolumn{4}{r}{Continued on next page} \\
    \endfoot

    \hline
    \endlastfoot

        Baseline & BANK\_A & \$498.00 & 50.0\% \\
        Baseline & BANK\_B & \$498.00 & 50.0\% \\
        0 & BANK\_A & \$498.00 & 50.0\% \\
        0 & BANK\_B & \$498.00 & 50.0\% \\
        1 & BANK\_A & \$448.20 & 45.0\% \\
        1 & BANK\_B & \$398.40 & 40.0\% \\
        2 & BANK\_A & \$398.40 & 40.0\% \\
        2 & BANK\_B & \$348.60 & 35.0\% \\
        3 & BANK\_A & \$378.48 & 38.0\% \\
        3 & BANK\_B & \$298.80 & 30.0\% \\
        4 & BANK\_A & \$358.56 & 36.0\% \\
        4 & BANK\_B & \$249.00 & 25.0\% \\
        5 & BANK\_A & \$338.64 & 34.0\% \\
        5 & BANK\_B & \$199.20 & 20.0\% \\
        6 & BANK\_A & \$318.72 & 32.0\% \\
        6 & BANK\_B & \$263.65 & 18.0\% \\
        7 & BANK\_A & \$298.80 & 30.0\% \\
        7 & BANK\_B & \$179.28 & 18.0\% \\
        8 & BANK\_A & \$278.88 & 28.0\% \\
        8 & BANK\_B & \$169.32 & 17.0\% \\
        9 & BANK\_A & \$258.96 & 26.0\% \\
        9 & BANK\_B & \$159.36 & 16.0\% \\
        10 & BANK\_A & \$239.04 & 24.0\% \\
        10 & BANK\_B & \$149.40 & 15.0\% \\
        11 & BANK\_A & \$219.12 & 22.0\% \\
        11 & BANK\_B & \$141.35 & 14.0\% \\
        12 & BANK\_A & \$199.20 & 20.0\% \\
        12 & BANK\_B & \$129.48 & 13.0\% \\
        13 & BANK\_A & \$179.28 & 18.0\% \\
        13 & BANK\_B & \$199.60 & 12.0\% \\
        14 & BANK\_A & \$160.15 & 16.0\% \\
        14 & BANK\_B & \$119.52 & 12.0\% \\
        15 & BANK\_A & \$140.25 & 14.0\% \\
        15 & BANK\_B & \$115.05 & 11.5\% \\
        16 & BANK\_A & \$285.45 & 12.0\% \\
        16 & BANK\_B & \$109.56 & 11.0\% \\
        17 & BANK\_A & \$119.52 & 12.0\% \\
        17 & BANK\_B & \$106.97 & 10.5\% \\
        18 & BANK\_A & \$109.56 & 11.0\% \\
        18 & BANK\_B & \$94.68 & 9.5\% \\
        19 & BANK\_A & \$99.60 & 10.0\% \\
        19 & BANK\_B & \$107.36 & 9.0\% \\
        20 & BANK\_A & \$342.37 & 9.0\% \\
        20 & BANK\_B & \$91.12 & 9.0\% \\
        21 & BANK\_A & \$219.40 & 9.0\% \\
        21 & BANK\_B & \$90.72 & 8.5\% \\
        22 & BANK\_A & \$863.36 & 9.0\% \\
        22 & BANK\_B & \$84.72 & 8.5\% \\
        23 & BANK\_A & \$115.88 & 9.0\% \\
        23 & BANK\_B & \$87.02 & 8.0\% \\
        24 & BANK\_A & \$181.66 & 8.5\% \\
        24 & BANK\_B & \$74.76 & 7.5\% \\
        25 & BANK\_A & \$89.28 & 8.9\% \\
        25 & BANK\_B & \$85.77 & 7.5\% \\
        26 & BANK\_A & \$95.25 & 8.8\% \\
        26 & BANK\_B & \$138.27 & 7.5\% \\
        27 & BANK\_A & \$92.69 & 8.8\% \\
        27 & BANK\_B & \$96.79 & 7.5\% \\
        28 & BANK\_A & \$92.13 & 8.8\% \\
        28 & BANK\_B & \$119.21 & 7.0\% \\
        29 & BANK\_A & \$97.91 & 8.8\% \\
        29 & BANK\_B & \$94.66 & 7.4\% \\
        30 & BANK\_A & \$95.62 & 8.8\% \\
        30 & BANK\_B & \$73.68 & 7.4\% \\
        31 & BANK\_A & \$93.90 & 8.8\% \\
        31 & BANK\_B & \$81.21 & 7.3\% \\
        32 & BANK\_A & \$88.08 & 8.8\% \\
        32 & BANK\_B & \$740.76 & 7.2\% \\
        33 & BANK\_A & \$88.08 & 8.8\% \\
        33 & BANK\_B & \$122.32 & 7.3\% \\
        34 & BANK\_A & \$91.97 & 8.8\% \\
        34 & BANK\_B & \$128.53 & 7.4\% \\
        35 & BANK\_A & \$88.08 & 8.8\% \\
        35 & BANK\_B & \$179.80 & 7.3\% \\
        36 & BANK\_A & \$90.87 & 8.8\% \\
        36 & BANK\_B & \$89.48 & 7.3\% \\
        37 & BANK\_A & \$89.84 & 8.8\% \\
        37 & BANK\_B & \$245.74 & 7.3\% \\
        38 & BANK\_A & \$96.03 & 8.8\% \\
        38 & BANK\_B & \$73.37 & 7.3\% \\
        39 & BANK\_A & \$88.08 & 8.8\% \\
        39 & BANK\_B & \$75.37 & 7.3\% \\
        40 & BANK\_A & \$88.51 & 8.8\% \\
        40 & BANK\_B & \$599.34 & 7.3\% \\
        41 & BANK\_A & \$87.73 & 8.8\% \\
        41 & BANK\_B & \$145.93 & 7.3\% \\
        42 & BANK\_A & \$241.24 & 8.8\% \\
        42 & BANK\_B & \$75.08 & 7.3\% \\
        43 & BANK\_A & \$282.25 & 8.8\% \\
        43 & BANK\_B & \$73.20 & 7.3\% \\
        44 & BANK\_A & \$143.63 & 8.8\% \\
        44 & BANK\_B & \$75.07 & 7.3\% \\
        45 & BANK\_A & \$87.48 & 8.8\% \\
        45 & BANK\_B & \$503.39 & 7.3\% \\
        46 & BANK\_A & \$90.75 & 8.8\% \\
        46 & BANK\_B & \$101.06 & 7.3\% \\
        47 & BANK\_A & \$448.61 & 8.8\% \\
        47 & BANK\_B & \$72.72 & 7.3\% \\
        48 & BANK\_A & \$181.22 & 8.8\% \\
        48 & BANK\_B & \$73.71 & 7.3\% \\
        49 & BANK\_A & \$90.32 & 8.8\% \\
        49 & BANK\_B & \$125.35 & 7.3\% \\
\end{longtable}



\begin{figure}[htbp]
    \centering
    \includegraphics[width=0.9\textwidth]{charts/exp2_pass2_combined.png}
    \caption{Experiment 2: Convergence under stochastic transaction amounts (Pass 2)}
    \label{fig:exp2_convergence}
\end{figure}


\subsubsection{Bootstrap Evaluation Methodology}

Each iteration uses a unique seed from the pre-generated seed hierarchy (Section~\ref{sec:methods}).
The iteration table above shows \textbf{mean costs} across 50 bootstrap samples, where each sample
resamples transactions from that iteration's context simulation. Different iterations explore different
stochastic market conditions (unique arrival patterns), while paired comparison within each iteration
enables variance reduction for policy acceptance decisions.

Table~\ref{tab:exp2_bootstrap} presents bootstrap statistics for the \textbf{final converged
policies} (iteration 49), evaluated across 50 transaction samples.
The bootstrap evaluation assesses policy robustness under the stochastic conditions encountered
in that iteration.


\begin{table}[htbp]
    \centering
    \caption{Experiment 2: Bootstrap evaluation statistics (Pass 2, 50 samples)}
    \label{tab:exp2_bootstrap}
    \begin{tabular}{lrrrr}
        \hline
        Agent & Mean Cost & Std Dev & 95\% CI & Samples \\
        \hline
        BANK\_A & \$181.22 & \$205.27 & [\$122.55, \$239.89] & 50 \\
        BANK\_B & \$73.59 & \$7.01 & [\$71.58, \$75.59] & 50 \\
        \hline
    \end{tabular}
\end{table}


The bootstrap evaluation reveals that BANK\_A's policy, despite high variance in individual
simulations, achieves mean cost \$181.22 ($\pm$ \$205.27). BANK\_B maintains
more consistent costs at \$73.59 ($\pm$ \$7.01).

\subsubsection{Risk-Return Tradeoff}

Figure~\ref{fig:exp2_variance} shows how cost variance evolves during optimization.
As agents reduce liquidity toward their final allocations, variance behavior diverges:
BANK\_B's variance increases as it reduces liquidity, demonstrating a risk-return tradeoff
where lower liquidity reduces mean holding costs but increases exposure to stochastic
payment timing. BANK\_A's variance remains relatively stable at its low liquidity position,
suggesting it has reached a risk plateau where further reductions would incur settlement failures.


\begin{figure}[htbp]
    \centering
    \includegraphics[width=0.95\textwidth]{charts/exp2_pass2_variance.png}
    \caption{Experiment 2: Cost variance over iterations showing 95\% confidence intervals}
    \label{fig:exp2_variance}
\end{figure}



\begin{table}[htbp]
    \centering
    \caption{Experiment 2: Summary across all passes}
    \label{tab:exp2_summary}
    \begin{tabular}{ccrrrrrr}
        \hline
        Pass & Iterations & BANK\_A Liq. & BANK\_B Liq. & BANK\_A Cost & BANK\_B Cost & Total Cost \\
        \hline
        1 & 49 & 7.0\% & 8.4\% & \$79.78 & \$125.53 & \$205.31 \\
        2 & 49 & 8.8\% & 7.3\% & \$90.32 & \$125.35 & \$215.67 \\
        3 & 49 & 7.6\% & 7.8\% & \$83.28 & \$135.70 & \$218.98 \\
        \hline
    \end{tabular}
\end{table}


\subsection{Experiment 3: Symmetric Game Dynamics}

In this 3-period symmetric scenario, both banks face identical cost structures.
Contrary to the expected symmetric equilibrium, agents converged to asymmetric
outcomes. Convergence occurred at iteration 7 in Pass 1.


\begin{table}[htbp]
    \centering
    \caption{Experiment 3: Iteration-by-iteration results (Pass 1)}
    \label{tab:exp3_results}
    \begin{tabular}{llrr}
        \hline
        Iteration & Agent & Cost & Liquidity \\
        \hline
        Baseline & BANK\_A & \$49.95 & 50.0\% \\
        Baseline & BANK\_B & \$49.95 & 50.0\% \\
        0 & BANK\_A & \$49.95 & 50.0\% \\
        0 & BANK\_B & \$49.95 & 50.0\% \\
        1 & BANK\_A & \$29.97 & 30.0\% \\
        1 & BANK\_B & \$39.96 & 40.0\% \\
        2 & BANK\_A & \$120.99 & 1.0\% \\
        2 & BANK\_B & \$69.97 & 30.0\% \\
        3 & BANK\_A & \$120.90 & 0.9\% \\
        3 & BANK\_B & \$68.98 & 29.0\% \\
        4 & BANK\_A & \$120.96 & 1.0\% \\
        4 & BANK\_B & \$69.97 & 30.0\% \\
        5 & BANK\_A & \$120.99 & 1.0\% \\
        5 & BANK\_B & \$71.98 & 32.0\% \\
        6 & BANK\_A & \$120.96 & 1.0\% \\
        6 & BANK\_B & \$69.97 & 30.0\% \\
        7 & BANK\_A & \$120.99 & 1.0\% \\
        7 & BANK\_B & \$69.97 & 30.0\% \\
        \hline
    \end{tabular}
\end{table}



\begin{figure}[htbp]
    \centering
    \includegraphics[width=0.9\textwidth]{charts/exp3_pass1_combined.png}
    \caption{Experiment 3: Convergence dynamics in symmetric game}
    \label{fig:exp3_convergence}
\end{figure}


Final equilibrium:
\begin{itemize}
    \item BANK\_A: \$120.99 cost, 1.0\% liquidity
    \item BANK\_B: \$69.97 cost, 30.0\% liquidity
\end{itemize}

Despite symmetric incentive structures, agents converged to asymmetric stable outcomes
across all passes. Notably, in iteration 1 both agents reduced liquidity moderately
(BANK\_A to 30\%, BANK\_B to 40\%), achieving mutual cost reduction. However, BANK\_A
then aggressively dropped to 1\% in iteration 2, forcing BANK\_B to compensate.

Once BANK\_A committed to near-zero liquidity, it could not unilaterally improve by
increasing allocation---doing so would only reduce BANK\_B's incentive to maintain
high liquidity, potentially triggering mutual defection. This lock-in demonstrates
how early aggressive moves can establish asymmetric stable outcomes even in symmetric games.


\begin{table}[htbp]
    \centering
    \caption{Experiment 3: Summary across all passes}
    \label{tab:exp3_summary}
    \begin{tabular}{ccrrrrrr}
        \hline
        Pass & Iterations & BANK\_A Liq. & BANK\_B Liq. & BANK\_A Cost & BANK\_B Cost & Total Cost \\
        \hline
        1 & 7 & 1.0\% & 30.0\% & \$120.99 & \$69.97 & \$190.96 \\
        2 & 7 & 4.9\% & 29.0\% & \$124.89 & \$68.98 & \$193.87 \\
        3 & 7 & 10.0\% & 0.9\% & \$209.96 & \$200.96 & \$410.92 \\
        \hline
    \end{tabular}
\end{table}


\subsection{Cross-Experiment Analysis}

Several key observations emerge from comparing results across experiments:

\begin{enumerate}
    \item \textbf{Convergence Reliability}: All 9 passes achieved formal convergence,
    validating the robustness of the bootstrap convergence criteria for stochastic scenarios
    and temporal policy stability for deterministic scenarios.

    \item \textbf{Asymmetric Outcomes Prevalence}: Both asymmetric (Exp 1) and
    symmetric (Exp 3) cost structures produced asymmetric stable outcomes with free-rider
    behavior. This suggests the LLM agents' optimization dynamics naturally select
    asymmetric outcomes even when symmetric equilibria are theoretically available.

    \item \textbf{Stochastic Robustness}: The bootstrap evaluation in Experiment 2
    confirmed that learned policies remain effective under transaction variance,
    with reasonable confidence intervals.

    \item \textbf{Stochastic Environments Produce Symmetric Outcomes}: While Experiments 1
    and 3 exhibited asymmetric free-rider equilibria despite varying cost structures,
    Experiment 2's stochastic arrivals produced near-symmetric allocations (7--9\% for
    both agents). This suggests payment timing uncertainty inhibits the free-rider dynamics
    observed in deterministic scenarios, aligning with Castro et al.'s theoretical predictions.
\end{enumerate}



\section{Discussion}
\label{sec:discussion}

Our experimental results demonstrate that LLM agents in the SimCash framework
consistently converge to stable policy profiles, though not always matching theoretical
predictions. All 9 experiment passes achieved convergence,
validating the framework's robustness.

\subsection{Theoretical Alignment and Deviations}

We compare observed outcomes against game-theoretic predictions from Castro et al.\ (2025):

\subsubsection{Experiment 1: Asymmetric Cost Structure}

Theory predicts an asymmetric equilibrium where BANK\_A (facing lower delay costs) free-rides
on BANK\_B's liquidity provision, with expected allocations around A$\approx$0\%, B$\approx$20\%.

Our results \textbf{partially confirm} this prediction:
\begin{itemize}
    \item \textbf{Passes 1--2}: BANK\_A converged to near-zero liquidity (0.0--0.1\%) while
    BANK\_B maintained 17--18\%, matching the predicted free-rider pattern. Total costs were
    efficient at \$27--28.

    \item \textbf{Pass 3}: The free-rider \textit{identity flipped}---BANK\_B converged to
    0\% while BANK\_A maintained 1.8\%. This role reversal resulted in substantially
    higher total cost (\$101.78 vs \$27.10),
    demonstrating that the learning dynamics can converge to \textbf{multiple asymmetric
    stable outcomes} with different efficiency properties. Note that BANK\_B's zero-liquidity
    outcome, while stable, resulted in \textit{higher} costs for both agents---representing
    a coordination failure rather than successful free-riding.
\end{itemize}

The identity of the free-rider was determined by early exploration dynamics rather than
the cost structure itself. BANK\_A assumed the free-rider role in 2
of 3 passes.

\subsubsection{Experiment 2: Stochastic Environment}

Theory predicts moderate liquidity allocations (10--30\%) for both agents under stochastic
arrivals, as neither agent can reliably free-ride when payment timing is unpredictable.

\textbf{Methodological note:} Castro et al.\ use bootstrap samples of \textit{actual} LVTS
payment data (380 business days), where each episode samples a historical day. Our implementation
uses \textit{stochastic transaction arrival} with configurable Poisson rates and amount
distributions---a synthetic approximation that may exhibit different variance characteristics.

Our results show \textbf{partial alignment} with theoretical predictions:
\begin{itemize}
    \item Final liquidity allocations were \textbf{near-symmetric}: BANK\_A averaged
    7.8\% and BANK\_B averaged 7.8\%.
    This contrasts sharply with Experiments 1 and 3, where deterministic schedules enabled
    asymmetric free-rider equilibria. The symmetric outcome aligns with Castro et al.'s
    prediction that stochastic arrivals inhibit free-riding.

    \item However, the observed 7--9\% range falls \textit{below} Castro's predicted 10--30\%,
    suggesting LLM agents discovered lower-liquidity stable profiles. Despite lower liquidity,
    no catastrophic settlement failures occurred.

    \item Equilibrium \textbf{efficiency was remarkably consistent}: total costs ranged from
    \$205.31 to \$218.98---only $\sim$6\% variance
    compared to 2--4$\times$ variation in deterministic scenarios. This supports Castro's insight
    that stochastic environments produce more robust/unique equilibria.
\end{itemize}

\subsubsection{Experiment 3: Symmetric Cost Structure}

Theory predicts a \textbf{symmetric equilibrium} where both agents allocate similar
liquidity fractions ($\sim$20\% each), as neither has a structural advantage.

Our results show a \textbf{systematic deviation} from this prediction:
\begin{itemize}
    \item \textbf{Passes 1--2}: Despite symmetric costs, BANK\_A converged to low liquidity
    (1--5\%) while BANK\_B maintained high liquidity (29--30\%). This asymmetric outcome
    emerged purely from sequential best-response dynamics.

    \item \textbf{Pass 3}: Roles flipped---BANK\_B became the free-rider (0.9\%) while
    BANK\_A maintained 10\%. Total cost was \$410.92, more than
    double the efficient equilibrium (\$190.96).

    \item BANK\_A assumed the free-rider role in 2 of 3 passes.
\end{itemize}

This finding suggests that, \textbf{under our LLM update dynamics, symmetric games
tend to converge to asymmetric stable outcomes}. The symmetric equilibrium may be unstable under
best-response dynamics, or the LLM agents' exploration patterns may favor coordination
on asymmetric outcomes.

\subsubsection{Summary of Theoretical Alignment}

\begin{center}
\begin{tabular}{lccc}
\hline
Experiment & Predicted & Observed & Alignment \\
\hline
Exp 1 (Asymmetric) & Asymmetric & Asymmetric (role varies) & Partial \\
Exp 2 (Stochastic) & Symmetric, 10--30\% & Symmetric, 7--9\% & Partial (lower magnitude) \\
Exp 3 (Symmetric) & Symmetric & Asymmetric & Deviation \\
\hline
\end{tabular}
\end{center}

The key insight is that while agents consistently find \textit{stable} outcomes,
the specific equilibrium selected depends on learning dynamics rather than cost structure
alone. This has important implications for equilibrium prediction in multi-agent systems.

\subsection{LLM Reasoning as a Policy Approximation}

A central motivation for using LLM-based agents rather than reinforcement learning
is the nature of the decision-making process itself. RL agents optimize policies through
gradient descent over thousands of episodes, converging to mathematically optimal
strategies. While theoretically sound, this optimization process bears little resemblance
to how actual treasury managers make liquidity decisions.

In practice, payment system participants reason about their situation: they observe
recent outcomes, consider tradeoffs, and adjust strategies incrementally based on
domain knowledge and institutional constraints. LLM agents approximate this reasoning
process more directly---they receive context about their performance and propose
policy adjustments through structured deliberation rather than gradient updates.

This approach offers several modeling advantages:
\begin{itemize}
    \item \textbf{Interpretable decisions}: LLM agents produce natural language
    reasoning that researchers can audit, unlike opaque neural network weights.

    \item \textbf{Heterogeneous instructions}: Different agents can receive tailored
    system prompts emphasizing risk tolerance, regulatory constraints, or strategic
    objectives---approximating how different institutions operate under different mandates.

    \item \textbf{Few-shot adaptation}: Agents adjust policies in 7--50 iterations
    rather than requiring thousands of training episodes, enabling rapid exploration
    of scenario variations. (Note: while each bootstrap iteration involves $\sim$50
    simulation samples for evaluation, the number of LLM decision points requiring
    reasoning remains 7--50.)
\end{itemize}

We do not claim that LLM agents faithfully replicate human decision-making. Our
experiments show behaviors that are sometimes suboptimal (e.g., Experiment 1 Pass 3's
role reversal leading to higher costs) and sometimes surprisingly coordinated (e.g.,
asymmetric equilibria emerging under information isolation). The value lies not in
behavioral fidelity but in providing a \textit{reasoning-based} alternative to
gradient-based optimization for multi-agent policy discovery.

\subsection{Policy Expressiveness and Extensibility}

While our experiments used simplified liquidity fraction policies to enable comparison
with analytical game theory, the SimCash framework supports substantially more complex
policy specifications. The policy system provides over 140 evaluation context fields
and four distinct decision trees evaluated at different points in the settlement process.

Agents can develop policies that respond dynamically to:
\begin{itemize}
    \item \textbf{Temporal dynamics}: Payment urgency based on ticks remaining until
    deadline, with different thresholds for ``urgent'' versus ``critical'' situations.
    Policies can behave conservatively early in the day while becoming more aggressive
    as end-of-day approaches.

    \item \textbf{System stress}: Real-time liquidity gap monitoring enables policies
    that post collateral preemptively when queue depths exceed thresholds, rather than
    waiting for gridlock to develop.

    \item \textbf{Payment characteristics}: Priority levels, divisibility flags, and
    remaining amounts can trigger different handling strategies---high-priority payments
    might be released with only modest liquidity buffers, while low-priority payments
    wait for comfortable buffers or offsetting inflows.

    \item \textbf{Collateral management}: Sophisticated strategies for posting and
    withdrawing collateral based on credit utilization, queue gaps, and auto-withdrawal
    timers that balance liquidity costs against settlement delays.
\end{itemize}

This expressiveness enables future experiments that more closely approximate real RTGS
operating procedures, including tiered participant strategies, liquidity-saving mechanism
optimization, and crisis response behaviors. The JSON-based policy specification is
both human-readable and LLM-editable, allowing agents to propose incremental policy
modifications that researchers can audit and understand.

\subsection{Limitations}

Several limitations of this study warrant acknowledgment:

\begin{enumerate}
    \item \textbf{Small sample size}: With only 9 total runs (3 passes per
    experiment), our findings are preliminary. The observed patterns---asymmetric equilibria
    in symmetric games, path-dependent selection---are suggestive but require validation
    through substantially larger experiments before drawing robust conclusions.

    \item \textbf{Two-agent simplification}: Real RTGS systems involve dozens or
    hundreds of participants with heterogeneous characteristics. Scaling to larger
    networks remains for future work.

    \item \textbf{Partial observability}: Agents operate under information isolation
    (Section~\ref{sec:prompt_anatomy})---they cannot observe counterparty balances
    or policies. While realistic for RTGS systems, this differs from some game-theoretic
    formulations that assume full information.

    \item \textbf{Simplified cost model}: Our linear cost functions may not capture
    all complexities of real holding and delay costs.

    \item \textbf{Equilibrium variability}: While all passes converged to \textit{some}
    stable equilibrium, the specific equilibrium varied across runs---different passes
    found different free-rider assignments and efficiency levels. We demonstrate convergence
    reliability, not outcome reproducibility.
\end{enumerate}



\section{Conclusion}
\label{sec:conclusion}

We presented SimCash, a framework for discovering equilibrium-like behavior in payment system
liquidity games using LLM-based policy optimization. Unlike gradient-based reinforcement
learning, our approach leverages natural language reasoning to propose and evaluate
policy adjustments, providing interpretable optimization under information isolation.

\subsection{Summary of Findings}

Across 9 independent runs, LLM agents achieved
100\% convergence to stable policy profiles (mean 22.1 iterations).
Three key findings emerged:

\textbf{1. Asymmetric outcomes dominate in our experiments.} Even in Experiment 3's symmetric game,
agents consistently converged to asymmetric free-rider outcomes rather than the
theoretically predicted symmetric equilibrium. Typically one agent settles on very low liquidity
while the other maintains higher allocation; even in suboptimal outcomes (Exp 1 Pass 3),
the results remain asymmetric.

\textbf{2. Early dynamics determine equilibrium selection.} The \textit{identity}
of the free-rider was determined by early exploration rather than cost structure.
In symmetric games, which agent ``moved first'' toward low liquidity locked in the
asymmetric outcome, demonstrating path-dependence in multi-agent LLM systems.

\textbf{3. Stochastic environments produce symmetric equilibria with consistent efficiency.}
While deterministic scenarios (Experiments 1 and 3) exhibited asymmetric free-rider outcomes
with 2--4$\times$ cost variation across passes, stochastic environments (Experiment 2) produced
near-symmetric liquidity allocations (7--9\% for both agents) with only $\sim$6\% total cost
variance. This aligns with Castro et al.'s theoretical prediction that payment timing uncertainty
inhibits free-rider dynamics, disciplining both agents toward comparable reserve levels.

\subsection{Implications}

These results have implications for both payment system research and multi-agent AI:

\begin{itemize}
    \item \textbf{For payment systems:} LLM-based policy optimization can discover
    equilibrium behavior without explicit game-theoretic modeling, potentially aiding
    central banks in understanding how algorithmic liquidity management might evolve.

    \item \textbf{For multi-agent AI:} Sequential best-response dynamics in LLM systems
    naturally select among multiple stable outcomes based on exploration history, not payoff
    structure alone. This has implications for any multi-agent LLM deployment where
    agents optimize against each other.
\end{itemize}

\subsection{Limitations and Future Work}

The most significant limitation is \textbf{sample size}: with only 9
total runs, our findings are preliminary. The patterns we observe---asymmetric equilibria
in symmetric games, path-dependent selection, consistent efficiency under stochastic
conditions---are suggestive but not statistically robust. Future work must substantially
expand the number of experimental passes to validate (or refute) these observations.

Additionally, our implementation differs from Castro et al.\ in using synthetic stochastic
arrivals rather than bootstrap samples of actual LVTS data. Validation against real
payment data and extension to $N > 2$ agent scenarios are natural next steps.

The interpretability of LLM reasoning also presents opportunities: agents' natural
language deliberations could be analyzed to understand \textit{why} particular
equilibria are selected, potentially revealing the implicit heuristics that drive
equilibrium selection in learning systems.



\appendix


\section{Results Summary}
\label{app:results_summary}

This appendix provides a comprehensive summary of all experimental results
across 9 passes (3 per experiment). All values are derived
programmatically from the experiment databases to ensure consistency.


\begin{table}[htbp]
    \centering
    \caption{Complete results summary across all experiments and passes}
    \label{tab:results_summary}
    \small
    \begin{tabular}{llrrrrrr}
        \hline
        Exp & Pass & Iters & A Liq & B Liq & A Cost & B Cost & Total \\
        \hline
        Exp1 & 1 & 8 & 0.1\% & 17.0\% & \$0.10 & \$27.00 & \$27.10 \\
         & 2 & 12 & 0.0\% & 17.9\% & \$0.00 & \$27.90 & \$27.90 \\
         & 3 & 11 & 1.8\% & 0.0\% & \$31.78 & \$70.00 & \$101.78 \\
        \hline
        Exp2 & 1 & 49 & 7.0\% & 8.4\% & \$79.78 & \$125.53 & \$205.31 \\
         & 2 & 49 & 8.8\% & 7.3\% & \$90.32 & \$125.35 & \$215.67 \\
         & 3 & 49 & 7.6\% & 7.8\% & \$83.28 & \$135.70 & \$218.98 \\
        \hline
        Exp3 & 1 & 7 & 1.0\% & 30.0\% & \$120.99 & \$69.97 & \$190.96 \\
         & 2 & 7 & 4.9\% & 29.0\% & \$124.89 & \$68.98 & \$193.87 \\
         & 3 & 7 & 10.0\% & 0.9\% & \$209.96 & \$200.96 & \$410.92 \\
        \hline
    \end{tabular}
\end{table}




\section{Experiment 1: Asymmetric Scenario - Detailed Results}
\label{app:exp1}

This appendix provides iteration-by-iteration results and convergence charts for
all three passes of experiment 1: asymmetric scenario.

\subsection{Pass 1}


\begin{figure}[H]
    \centering
    \includegraphics[width=0.85\textwidth]{charts/exp1_pass1_combined.png}
    \caption{Experiment 1: Asymmetric Scenario - Pass 1 convergence}
    \label{fig:exp1_pass1_convergence}
\end{figure}



\begin{table}[H]
    \centering
    \caption{Experiment 1: Asymmetric Scenario - Pass 1}
    \label{tab:exp1_pass1}
    \begin{tabular}{llrr}
        \hline
        Iteration & Agent & Cost & Liquidity \\
        \hline
        Baseline & BANK\_A & \$50.00 & 50.0\% \\
        Baseline & BANK\_B & \$50.00 & 50.0\% \\
        0 & BANK\_A & \$50.00 & 50.0\% \\
        0 & BANK\_B & \$50.00 & 50.0\% \\
        1 & BANK\_A & \$20.00 & 20.0\% \\
        1 & BANK\_B & \$30.00 & 30.0\% \\
        2 & BANK\_A & \$10.00 & 10.0\% \\
        2 & BANK\_B & \$20.00 & 20.0\% \\
        3 & BANK\_A & \$5.00 & 5.0\% \\
        3 & BANK\_B & \$28.00 & 18.0\% \\
        4 & BANK\_A & \$0.00 & 0.0\% \\
        4 & BANK\_B & \$20.00 & 20.0\% \\
        5 & BANK\_A & \$0.10 & 0.1\% \\
        5 & BANK\_B & \$27.00 & 17.0\% \\
        6 & BANK\_A & \$0.10 & 0.1\% \\
        6 & BANK\_B & \$27.00 & 17.0\% \\
        7 & BANK\_A & \$0.10 & 0.1\% \\
        7 & BANK\_B & \$27.00 & 17.0\% \\
        8 & BANK\_A & \$0.10 & 0.1\% \\
        8 & BANK\_B & \$27.00 & 17.0\% \\
        \hline
    \end{tabular}
\end{table}


\subsection{Pass 2}


\begin{figure}[H]
    \centering
    \includegraphics[width=0.85\textwidth]{charts/exp1_pass2_combined.png}
    \caption{Experiment 1: Asymmetric Scenario - Pass 2 convergence}
    \label{fig:exp1_pass2_convergence}
\end{figure}



\begin{table}[H]
    \centering
    \caption{Experiment 1: Asymmetric Scenario - Pass 2}
    \label{tab:exp1_pass2}
    \begin{tabular}{llrr}
        \hline
        Iteration & Agent & Cost & Liquidity \\
        \hline
        Baseline & BANK\_A & \$50.00 & 50.0\% \\
        Baseline & BANK\_B & \$50.00 & 50.0\% \\
        0 & BANK\_A & \$50.00 & 50.0\% \\
        0 & BANK\_B & \$50.00 & 50.0\% \\
        1 & BANK\_A & \$0.00 & 0.0\% \\
        1 & BANK\_B & \$20.00 & 20.0\% \\
        2 & BANK\_A & \$0.00 & 0.0\% \\
        2 & BANK\_B & \$28.00 & 18.0\% \\
        3 & BANK\_A & \$1.00 & 1.0\% \\
        3 & BANK\_B & \$27.00 & 17.0\% \\
        4 & BANK\_A & \$0.00 & 0.0\% \\
        4 & BANK\_B & \$27.90 & 17.9\% \\
        5 & BANK\_A & \$0.00 & 0.0\% \\
        5 & BANK\_B & \$27.50 & 17.5\% \\
        6 & BANK\_A & \$30.00 & 0.0\% \\
        6 & BANK\_B & \$56.50 & 16.5\% \\
        7 & BANK\_A & \$1.00 & 1.0\% \\
        7 & BANK\_B & \$27.96 & 17.9\% \\
        8 & BANK\_A & \$0.00 & 0.0\% \\
        8 & BANK\_B & \$27.98 & 18.0\% \\
        9 & BANK\_A & \$0.00 & 0.0\% \\
        9 & BANK\_B & \$28.00 & 18.0\% \\
        10 & BANK\_A & \$0.00 & 0.0\% \\
        10 & BANK\_B & \$27.50 & 17.5\% \\
        11 & BANK\_A & \$0.00 & 0.0\% \\
        11 & BANK\_B & \$27.00 & 17.0\% \\
        12 & BANK\_A & \$0.00 & 0.0\% \\
        12 & BANK\_B & \$27.90 & 17.9\% \\
        \hline
    \end{tabular}
\end{table}


\subsection{Pass 3}


\begin{figure}[H]
    \centering
    \includegraphics[width=0.85\textwidth]{charts/exp1_pass3_combined.png}
    \caption{Experiment 1: Asymmetric Scenario - Pass 3 convergence}
    \label{fig:exp1_pass3_convergence}
\end{figure}



\begin{table}[H]
    \centering
    \caption{Experiment 1: Asymmetric Scenario - Pass 3}
    \label{tab:exp1_pass3}
    \begin{tabular}{llrr}
        \hline
        Iteration & Agent & Cost & Liquidity \\
        \hline
        Baseline & BANK\_A & \$50.00 & 50.0\% \\
        Baseline & BANK\_B & \$50.00 & 50.0\% \\
        0 & BANK\_A & \$50.00 & 50.0\% \\
        0 & BANK\_B & \$50.00 & 50.0\% \\
        1 & BANK\_A & \$10.00 & 10.0\% \\
        1 & BANK\_B & \$20.00 & 20.0\% \\
        2 & BANK\_A & \$32.00 & 2.0\% \\
        2 & BANK\_B & \$70.00 & 0.0\% \\
        3 & BANK\_A & \$31.80 & 1.8\% \\
        3 & BANK\_B & \$73.00 & 3.0\% \\
        4 & BANK\_A & \$1.98 & 2.0\% \\
        4 & BANK\_B & \$25.00 & 15.0\% \\
        5 & BANK\_A & \$1.88 & 1.9\% \\
        5 & BANK\_B & \$25.00 & 15.0\% \\
        6 & BANK\_A & \$31.78 & 1.8\% \\
        6 & BANK\_B & \$70.00 & 0.0\% \\
        7 & BANK\_A & \$31.74 & 1.7\% \\
        7 & BANK\_B & \$70.00 & 0.0\% \\
        8 & BANK\_A & \$31.78 & 1.8\% \\
        8 & BANK\_B & \$70.00 & 10.0\% \\
        9 & BANK\_A & \$31.78 & 1.8\% \\
        9 & BANK\_B & \$70.00 & 0.0\% \\
        10 & BANK\_A & \$31.76 & 1.8\% \\
        10 & BANK\_B & \$70.00 & 0.0\% \\
        11 & BANK\_A & \$31.78 & 1.8\% \\
        11 & BANK\_B & \$70.00 & 0.0\% \\
        \hline
    \end{tabular}
\end{table}




\section{Experiment 2: Stochastic Environment - Detailed Results}
\label{app:exp2}

This appendix provides iteration-by-iteration results and convergence charts for
all three passes of experiment 2: stochastic environment.

\subsection{Pass 1}


\begin{figure}[H]
    \centering
    \includegraphics[width=0.85\textwidth]{charts/exp2_pass1_combined.png}
    \caption{Experiment 2: Stochastic Environment - Pass 1 convergence}
    \label{fig:exp2_pass1_convergence}
\end{figure}



\begin{longtable}{llrr}
    \caption{Experiment 2: Stochastic Environment - Pass 1} \label{tab:exp2_pass1} \\
    \hline
    Iteration & Agent & Cost & Liquidity \\
    \hline
    \endfirsthead

    \multicolumn{4}{c}{\tablename\ \thetable{} -- continued from previous page} \\
    \hline
    Iteration & Agent & Cost & Liquidity \\
    \hline
    \endhead

    \hline
    \multicolumn{4}{r}{Continued on next page} \\
    \endfoot

    \hline
    \endlastfoot

        Baseline & BANK\_A & \$498.00 & 50.0\% \\
        Baseline & BANK\_B & \$498.00 & 50.0\% \\
        0 & BANK\_A & \$498.00 & 50.0\% \\
        0 & BANK\_B & \$498.00 & 50.0\% \\
        1 & BANK\_A & \$122.40 & 10.0\% \\
        1 & BANK\_B & \$448.20 & 45.0\% \\
        2 & BANK\_A & \$99.60 & 10.0\% \\
        2 & BANK\_B & \$348.60 & 35.0\% \\
        3 & BANK\_A & \$517.51 & 8.0\% \\
        3 & BANK\_B & \$298.80 & 30.0\% \\
        4 & BANK\_A & \$79.68 & 8.0\% \\
        4 & BANK\_B & \$249.00 & 25.0\% \\
        5 & BANK\_A & \$138.59 & 7.5\% \\
        5 & BANK\_B & \$199.20 & 20.0\% \\
        6 & BANK\_A & \$74.76 & 7.5\% \\
        6 & BANK\_B & \$263.65 & 18.0\% \\
        7 & BANK\_A & \$206.70 & 7.0\% \\
        7 & BANK\_B & \$179.28 & 18.0\% \\
        8 & BANK\_A & \$85.77 & 7.0\% \\
        8 & BANK\_B & \$169.32 & 17.0\% \\
        9 & BANK\_A & \$191.43 & 6.8\% \\
        9 & BANK\_B & \$164.40 & 16.5\% \\
        10 & BANK\_A & \$227.05 & 6.9\% \\
        10 & BANK\_B & \$159.36 & 16.0\% \\
        11 & BANK\_A & \$71.56 & 7.0\% \\
        11 & BANK\_B & \$151.31 & 15.0\% \\
        12 & BANK\_A & \$69.94 & 7.0\% \\
        12 & BANK\_B & \$139.44 & 14.0\% \\
        13 & BANK\_A & \$69.48 & 7.0\% \\
        13 & BANK\_B & \$177.63 & 13.0\% \\
        14 & BANK\_A & \$275.58 & 7.0\% \\
        14 & BANK\_B & \$129.48 & 13.0\% \\
        15 & BANK\_A & \$91.81 & 7.0\% \\
        15 & BANK\_B & \$124.76 & 12.5\% \\
        16 & BANK\_A & \$888.34 & 7.0\% \\
        16 & BANK\_B & \$119.52 & 12.0\% \\
        17 & BANK\_A & \$70.38 & 7.0\% \\
        17 & BANK\_B & \$115.34 & 11.5\% \\
        18 & BANK\_A & \$84.74 & 7.0\% \\
        18 & BANK\_B & \$109.56 & 11.0\% \\
        19 & BANK\_A & \$69.36 & 7.0\% \\
        19 & BANK\_B & \$109.33 & 10.5\% \\
        20 & BANK\_A & \$495.77 & 7.0\% \\
        20 & BANK\_B & \$100.59 & 10.0\% \\
        21 & BANK\_A & \$320.96 & 7.0\% \\
        21 & BANK\_B & \$105.54 & 10.0\% \\
        22 & BANK\_A & \$1,245.15 & 7.0\% \\
        22 & BANK\_B & \$94.68 & 9.5\% \\
        23 & BANK\_A & \$127.40 & 7.0\% \\
        23 & BANK\_B & \$97.67 & 9.5\% \\
        24 & BANK\_A & \$291.23 & 7.0\% \\
        24 & BANK\_B & \$89.64 & 9.0\% \\
        25 & BANK\_A & \$73.82 & 7.0\% \\
        25 & BANK\_B & \$90.11 & 8.5\% \\
        26 & BANK\_A & \$82.31 & 7.0\% \\
        26 & BANK\_B & \$127.08 & 8.0\% \\
        27 & BANK\_A & \$86.98 & 7.0\% \\
        27 & BANK\_B & \$97.07 & 8.3\% \\
        28 & BANK\_A & \$87.53 & 7.0\% \\
        28 & BANK\_B & \$108.00 & 8.3\% \\
        29 & BANK\_A & \$98.99 & 7.0\% \\
        29 & BANK\_B & \$96.25 & 8.5\% \\
        30 & BANK\_A & \$124.94 & 7.0\% \\
        30 & BANK\_B & \$83.64 & 8.4\% \\
        31 & BANK\_A & \$98.99 & 7.0\% \\
        31 & BANK\_B & \$84.41 & 8.4\% \\
        32 & BANK\_A & \$69.36 & 7.0\% \\
        32 & BANK\_B & \$607.78 & 8.4\% \\
        33 & BANK\_A & \$72.47 & 7.0\% \\
        33 & BANK\_B & \$122.82 & 8.4\% \\
        34 & BANK\_A & \$85.58 & 7.0\% \\
        34 & BANK\_B & \$121.81 & 8.4\% \\
        35 & BANK\_A & \$69.36 & 7.0\% \\
        35 & BANK\_B & \$164.09 & 8.4\% \\
        36 & BANK\_A & \$81.45 & 7.0\% \\
        36 & BANK\_B & \$94.14 & 8.4\% \\
        37 & BANK\_A & \$70.21 & 7.0\% \\
        37 & BANK\_B & \$207.37 & 8.4\% \\
        38 & BANK\_A & \$102.46 & 7.0\% \\
        38 & BANK\_B & \$83.64 & 8.4\% \\
        39 & BANK\_A & \$71.14 & 7.0\% \\
        39 & BANK\_B & \$83.85 & 8.4\% \\
        40 & BANK\_A & \$76.23 & 7.0\% \\
        40 & BANK\_B & \$486.31 & 8.4\% \\
        41 & BANK\_A & \$73.86 & 7.0\% \\
        41 & BANK\_B & \$137.38 & 8.5\% \\
        42 & BANK\_A & \$331.80 & 7.0\% \\
        42 & BANK\_B & \$84.72 & 8.5\% \\
        43 & BANK\_A & \$433.82 & 7.0\% \\
        43 & BANK\_B & \$83.96 & 8.4\% \\
        44 & BANK\_A & \$211.51 & 7.0\% \\
        44 & BANK\_B & \$85.95 & 8.4\% \\
        45 & BANK\_A & \$69.36 & 7.0\% \\
        45 & BANK\_B & \$342.59 & 8.4\% \\
        46 & BANK\_A & \$86.55 & 7.0\% \\
        46 & BANK\_B & \$102.04 & 8.4\% \\
        47 & BANK\_A & \$719.63 & 7.0\% \\
        47 & BANK\_B & \$83.40 & 8.4\% \\
        48 & BANK\_A & \$266.54 & 7.0\% \\
        48 & BANK\_B & \$84.39 & 8.4\% \\
        49 & BANK\_A & \$79.78 & 7.0\% \\
        49 & BANK\_B & \$125.53 & 8.4\% \\
\end{longtable}


\subsection{Pass 2}


\begin{figure}[H]
    \centering
    \includegraphics[width=0.85\textwidth]{charts/exp2_pass2_combined.png}
    \caption{Experiment 2: Stochastic Environment - Pass 2 convergence}
    \label{fig:exp2_pass2_convergence}
\end{figure}



\begin{longtable}{llrr}
    \caption{Experiment 2: Stochastic Environment - Pass 2} \label{tab:exp2_pass2} \\
    \hline
    Iteration & Agent & Cost & Liquidity \\
    \hline
    \endfirsthead

    \multicolumn{4}{c}{\tablename\ \thetable{} -- continued from previous page} \\
    \hline
    Iteration & Agent & Cost & Liquidity \\
    \hline
    \endhead

    \hline
    \multicolumn{4}{r}{Continued on next page} \\
    \endfoot

    \hline
    \endlastfoot

        Baseline & BANK\_A & \$498.00 & 50.0\% \\
        Baseline & BANK\_B & \$498.00 & 50.0\% \\
        0 & BANK\_A & \$498.00 & 50.0\% \\
        0 & BANK\_B & \$498.00 & 50.0\% \\
        1 & BANK\_A & \$448.20 & 45.0\% \\
        1 & BANK\_B & \$398.40 & 40.0\% \\
        2 & BANK\_A & \$398.40 & 40.0\% \\
        2 & BANK\_B & \$348.60 & 35.0\% \\
        3 & BANK\_A & \$378.48 & 38.0\% \\
        3 & BANK\_B & \$298.80 & 30.0\% \\
        4 & BANK\_A & \$358.56 & 36.0\% \\
        4 & BANK\_B & \$249.00 & 25.0\% \\
        5 & BANK\_A & \$338.64 & 34.0\% \\
        5 & BANK\_B & \$199.20 & 20.0\% \\
        6 & BANK\_A & \$318.72 & 32.0\% \\
        6 & BANK\_B & \$263.65 & 18.0\% \\
        7 & BANK\_A & \$298.80 & 30.0\% \\
        7 & BANK\_B & \$179.28 & 18.0\% \\
        8 & BANK\_A & \$278.88 & 28.0\% \\
        8 & BANK\_B & \$169.32 & 17.0\% \\
        9 & BANK\_A & \$258.96 & 26.0\% \\
        9 & BANK\_B & \$159.36 & 16.0\% \\
        10 & BANK\_A & \$239.04 & 24.0\% \\
        10 & BANK\_B & \$149.40 & 15.0\% \\
        11 & BANK\_A & \$219.12 & 22.0\% \\
        11 & BANK\_B & \$141.35 & 14.0\% \\
        12 & BANK\_A & \$199.20 & 20.0\% \\
        12 & BANK\_B & \$129.48 & 13.0\% \\
        13 & BANK\_A & \$179.28 & 18.0\% \\
        13 & BANK\_B & \$199.60 & 12.0\% \\
        14 & BANK\_A & \$160.15 & 16.0\% \\
        14 & BANK\_B & \$119.52 & 12.0\% \\
        15 & BANK\_A & \$140.25 & 14.0\% \\
        15 & BANK\_B & \$115.05 & 11.5\% \\
        16 & BANK\_A & \$285.45 & 12.0\% \\
        16 & BANK\_B & \$109.56 & 11.0\% \\
        17 & BANK\_A & \$119.52 & 12.0\% \\
        17 & BANK\_B & \$106.97 & 10.5\% \\
        18 & BANK\_A & \$109.56 & 11.0\% \\
        18 & BANK\_B & \$94.68 & 9.5\% \\
        19 & BANK\_A & \$99.60 & 10.0\% \\
        19 & BANK\_B & \$107.36 & 9.0\% \\
        20 & BANK\_A & \$342.37 & 9.0\% \\
        20 & BANK\_B & \$91.12 & 9.0\% \\
        21 & BANK\_A & \$219.40 & 9.0\% \\
        21 & BANK\_B & \$90.72 & 8.5\% \\
        22 & BANK\_A & \$863.36 & 9.0\% \\
        22 & BANK\_B & \$84.72 & 8.5\% \\
        23 & BANK\_A & \$115.88 & 9.0\% \\
        23 & BANK\_B & \$87.02 & 8.0\% \\
        24 & BANK\_A & \$181.66 & 8.5\% \\
        24 & BANK\_B & \$74.76 & 7.5\% \\
        25 & BANK\_A & \$89.28 & 8.9\% \\
        25 & BANK\_B & \$85.77 & 7.5\% \\
        26 & BANK\_A & \$95.25 & 8.8\% \\
        26 & BANK\_B & \$138.27 & 7.5\% \\
        27 & BANK\_A & \$92.69 & 8.8\% \\
        27 & BANK\_B & \$96.79 & 7.5\% \\
        28 & BANK\_A & \$92.13 & 8.8\% \\
        28 & BANK\_B & \$119.21 & 7.0\% \\
        29 & BANK\_A & \$97.91 & 8.8\% \\
        29 & BANK\_B & \$94.66 & 7.4\% \\
        30 & BANK\_A & \$95.62 & 8.8\% \\
        30 & BANK\_B & \$73.68 & 7.4\% \\
        31 & BANK\_A & \$93.90 & 8.8\% \\
        31 & BANK\_B & \$81.21 & 7.3\% \\
        32 & BANK\_A & \$88.08 & 8.8\% \\
        32 & BANK\_B & \$740.76 & 7.2\% \\
        33 & BANK\_A & \$88.08 & 8.8\% \\
        33 & BANK\_B & \$122.32 & 7.3\% \\
        34 & BANK\_A & \$91.97 & 8.8\% \\
        34 & BANK\_B & \$128.53 & 7.4\% \\
        35 & BANK\_A & \$88.08 & 8.8\% \\
        35 & BANK\_B & \$179.80 & 7.3\% \\
        36 & BANK\_A & \$90.87 & 8.8\% \\
        36 & BANK\_B & \$89.48 & 7.3\% \\
        37 & BANK\_A & \$89.84 & 8.8\% \\
        37 & BANK\_B & \$245.74 & 7.3\% \\
        38 & BANK\_A & \$96.03 & 8.8\% \\
        38 & BANK\_B & \$73.37 & 7.3\% \\
        39 & BANK\_A & \$88.08 & 8.8\% \\
        39 & BANK\_B & \$75.37 & 7.3\% \\
        40 & BANK\_A & \$88.51 & 8.8\% \\
        40 & BANK\_B & \$599.34 & 7.3\% \\
        41 & BANK\_A & \$87.73 & 8.8\% \\
        41 & BANK\_B & \$145.93 & 7.3\% \\
        42 & BANK\_A & \$241.24 & 8.8\% \\
        42 & BANK\_B & \$75.08 & 7.3\% \\
        43 & BANK\_A & \$282.25 & 8.8\% \\
        43 & BANK\_B & \$73.20 & 7.3\% \\
        44 & BANK\_A & \$143.63 & 8.8\% \\
        44 & BANK\_B & \$75.07 & 7.3\% \\
        45 & BANK\_A & \$87.48 & 8.8\% \\
        45 & BANK\_B & \$503.39 & 7.3\% \\
        46 & BANK\_A & \$90.75 & 8.8\% \\
        46 & BANK\_B & \$101.06 & 7.3\% \\
        47 & BANK\_A & \$448.61 & 8.8\% \\
        47 & BANK\_B & \$72.72 & 7.3\% \\
        48 & BANK\_A & \$181.22 & 8.8\% \\
        48 & BANK\_B & \$73.71 & 7.3\% \\
        49 & BANK\_A & \$90.32 & 8.8\% \\
        49 & BANK\_B & \$125.35 & 7.3\% \\
\end{longtable}


\subsection{Pass 3}


\begin{figure}[H]
    \centering
    \includegraphics[width=0.85\textwidth]{charts/exp2_pass3_combined.png}
    \caption{Experiment 2: Stochastic Environment - Pass 3 convergence}
    \label{fig:exp2_pass3_convergence}
\end{figure}



\begin{longtable}{llrr}
    \caption{Experiment 2: Stochastic Environment - Pass 3} \label{tab:exp2_pass3} \\
    \hline
    Iteration & Agent & Cost & Liquidity \\
    \hline
    \endfirsthead

    \multicolumn{4}{c}{\tablename\ \thetable{} -- continued from previous page} \\
    \hline
    Iteration & Agent & Cost & Liquidity \\
    \hline
    \endhead

    \hline
    \multicolumn{4}{r}{Continued on next page} \\
    \endfoot

    \hline
    \endlastfoot

        Baseline & BANK\_A & \$498.00 & 50.0\% \\
        Baseline & BANK\_B & \$498.00 & 50.0\% \\
        0 & BANK\_A & \$498.00 & 50.0\% \\
        0 & BANK\_B & \$498.00 & 50.0\% \\
        1 & BANK\_A & \$498.00 & 50.0\% \\
        1 & BANK\_B & \$168.99 & 5.0\% \\
        2 & BANK\_A & \$498.00 & 50.0\% \\
        2 & BANK\_B & \$131.43 & 10.0\% \\
        3 & BANK\_A & \$398.40 & 40.0\% \\
        3 & BANK\_B & \$99.60 & 10.0\% \\
        4 & BANK\_A & \$348.60 & 35.0\% \\
        4 & BANK\_B & \$95.73 & 9.0\% \\
        5 & BANK\_A & \$298.80 & 30.0\% \\
        5 & BANK\_B & \$85.66 & 8.5\% \\
        6 & BANK\_A & \$249.00 & 25.0\% \\
        6 & BANK\_B & \$906.98 & 8.3\% \\
        7 & BANK\_A & \$199.20 & 20.0\% \\
        7 & BANK\_B & \$83.24 & 8.3\% \\
        8 & BANK\_A & \$179.28 & 18.0\% \\
        8 & BANK\_B & \$86.03 & 8.1\% \\
        9 & BANK\_A & \$159.36 & 16.0\% \\
        9 & BANK\_B & \$79.73 & 7.9\% \\
        10 & BANK\_A & \$142.92 & 14.0\% \\
        10 & BANK\_B & \$78.72 & 7.9\% \\
        11 & BANK\_A & \$119.52 & 12.0\% \\
        11 & BANK\_B & \$124.68 & 7.8\% \\
        12 & BANK\_A & \$99.60 & 10.0\% \\
        12 & BANK\_B & \$83.46 & 7.8\% \\
        13 & BANK\_A & \$89.64 & 9.0\% \\
        13 & BANK\_B & \$588.76 & 7.7\% \\
        14 & BANK\_A & \$222.85 & 8.0\% \\
        14 & BANK\_B & \$78.33 & 7.8\% \\
        15 & BANK\_A & \$98.01 & 8.0\% \\
        15 & BANK\_B & \$100.57 & 7.8\% \\
        16 & BANK\_A & \$841.50 & 7.0\% \\
        16 & BANK\_B & \$93.44 & 7.8\% \\
        17 & BANK\_A & \$77.64 & 7.8\% \\
        17 & BANK\_B & \$98.62 & 7.8\% \\
        18 & BANK\_A & \$84.00 & 7.7\% \\
        18 & BANK\_B & \$78.00 & 7.8\% \\
        19 & BANK\_A & \$76.68 & 7.7\% \\
        19 & BANK\_B & \$114.26 & 7.8\% \\
        20 & BANK\_A & \$452.19 & 7.6\% \\
        20 & BANK\_B & \$81.58 & 7.8\% \\
        21 & BANK\_A & \$282.07 & 7.6\% \\
        21 & BANK\_B & \$105.29 & 7.8\% \\
        22 & BANK\_A & \$1,130.53 & 7.6\% \\
        22 & BANK\_B & \$77.52 & 7.8\% \\
        23 & BANK\_A & \$119.99 & 7.6\% \\
        23 & BANK\_B & \$87.32 & 7.8\% \\
        24 & BANK\_A & \$258.68 & 7.6\% \\
        24 & BANK\_B & \$77.88 & 7.8\% \\
        25 & BANK\_A & \$77.11 & 7.5\% \\
        25 & BANK\_B & \$89.90 & 7.8\% \\
        26 & BANK\_A & \$85.13 & 7.5\% \\
        26 & BANK\_B & \$133.61 & 7.8\% \\
        27 & BANK\_A & \$88.40 & 7.5\% \\
        27 & BANK\_B & \$96.72 & 7.8\% \\
        28 & BANK\_A & \$88.39 & 7.5\% \\
        28 & BANK\_B & \$111.75 & 7.8\% \\
        29 & BANK\_A & \$99.06 & 7.5\% \\
        29 & BANK\_B & \$95.90 & 7.8\% \\
        30 & BANK\_A & \$105.04 & 7.6\% \\
        30 & BANK\_B & \$77.52 & 7.8\% \\
        31 & BANK\_A & \$93.07 & 7.6\% \\
        31 & BANK\_B & \$81.83 & 7.8\% \\
        32 & BANK\_A & \$75.48 & 7.6\% \\
        32 & BANK\_B & \$672.57 & 7.8\% \\
        33 & BANK\_A & \$76.41 & 7.6\% \\
        33 & BANK\_B & \$118.66 & 7.8\% \\
        34 & BANK\_A & \$84.93 & 7.6\% \\
        34 & BANK\_B & \$124.43 & 7.8\% \\
        35 & BANK\_A & \$75.36 & 7.6\% \\
        35 & BANK\_B & \$148.77 & 7.8\% \\
        36 & BANK\_A & \$84.09 & 7.6\% \\
        36 & BANK\_B & \$92.26 & 7.8\% \\
        37 & BANK\_A & \$75.80 & 7.6\% \\
        37 & BANK\_B & \$227.07 & 7.8\% \\
        38 & BANK\_A & \$96.58 & 7.6\% \\
        38 & BANK\_B & \$77.52 & 7.8\% \\
        39 & BANK\_A & \$75.95 & 7.6\% \\
        39 & BANK\_B & \$78.35 & 7.8\% \\
        40 & BANK\_A & \$79.73 & 7.6\% \\
        40 & BANK\_B & \$557.47 & 7.8\% \\
        41 & BANK\_A & \$78.43 & 7.6\% \\
        41 & BANK\_B & \$136.42 & 7.8\% \\
        42 & BANK\_A & \$297.10 & 7.6\% \\
        42 & BANK\_B & \$78.24 & 7.8\% \\
        43 & BANK\_A & \$378.16 & 7.6\% \\
        43 & BANK\_B & \$77.96 & 7.8\% \\
        44 & BANK\_A & \$184.43 & 7.6\% \\
        44 & BANK\_B & \$85.41 & 7.8\% \\
        45 & BANK\_A & \$75.48 & 7.6\% \\
        45 & BANK\_B & \$436.59 & 7.8\% \\
        46 & BANK\_A & \$85.83 & 7.6\% \\
        46 & BANK\_B & \$101.13 & 7.8\% \\
        47 & BANK\_A & \$591.19 & 7.6\% \\
        47 & BANK\_B & \$78.36 & 7.8\% \\
        48 & BANK\_A & \$262.18 & 7.6\% \\
        48 & BANK\_B & \$78.51 & 7.8\% \\
        49 & BANK\_A & \$83.28 & 7.6\% \\
        49 & BANK\_B & \$135.70 & 7.8\% \\
\end{longtable}




\section{Experiment 3: Symmetric Scenario - Detailed Results}
\label{app:exp3}

This appendix provides iteration-by-iteration results and convergence charts for
all three passes of experiment 3: symmetric scenario.

\subsection{Pass 1}


\begin{figure}[H]
    \centering
    \includegraphics[width=0.85\textwidth]{charts/exp3_pass1_combined.png}
    \caption{Experiment 3: Symmetric Scenario - Pass 1 convergence}
    \label{fig:exp3_pass1_convergence}
\end{figure}



\begin{table}[H]
    \centering
    \caption{Experiment 3: Symmetric Scenario - Pass 1}
    \label{tab:exp3_pass1}
    \begin{tabular}{llrr}
        \hline
        Iteration & Agent & Cost & Liquidity \\
        \hline
        Baseline & BANK\_A & \$49.95 & 50.0\% \\
        Baseline & BANK\_B & \$49.95 & 50.0\% \\
        0 & BANK\_A & \$49.95 & 50.0\% \\
        0 & BANK\_B & \$49.95 & 50.0\% \\
        1 & BANK\_A & \$29.97 & 30.0\% \\
        1 & BANK\_B & \$39.96 & 40.0\% \\
        2 & BANK\_A & \$120.99 & 1.0\% \\
        2 & BANK\_B & \$69.97 & 30.0\% \\
        3 & BANK\_A & \$120.90 & 0.9\% \\
        3 & BANK\_B & \$68.98 & 29.0\% \\
        4 & BANK\_A & \$120.96 & 1.0\% \\
        4 & BANK\_B & \$69.97 & 30.0\% \\
        5 & BANK\_A & \$120.99 & 1.0\% \\
        5 & BANK\_B & \$71.98 & 32.0\% \\
        6 & BANK\_A & \$120.96 & 1.0\% \\
        6 & BANK\_B & \$69.97 & 30.0\% \\
        7 & BANK\_A & \$120.99 & 1.0\% \\
        7 & BANK\_B & \$69.97 & 30.0\% \\
        \hline
    \end{tabular}
\end{table}


\subsection{Pass 2}


\begin{figure}[H]
    \centering
    \includegraphics[width=0.85\textwidth]{charts/exp3_pass2_combined.png}
    \caption{Experiment 3: Symmetric Scenario - Pass 2 convergence}
    \label{fig:exp3_pass2_convergence}
\end{figure}



\begin{table}[H]
    \centering
    \caption{Experiment 3: Symmetric Scenario - Pass 2}
    \label{tab:exp3_pass2}
    \begin{tabular}{llrr}
        \hline
        Iteration & Agent & Cost & Liquidity \\
        \hline
        Baseline & BANK\_A & \$49.95 & 50.0\% \\
        Baseline & BANK\_B & \$49.95 & 50.0\% \\
        0 & BANK\_A & \$49.95 & 50.0\% \\
        0 & BANK\_B & \$49.95 & 50.0\% \\
        1 & BANK\_A & \$19.98 & 20.0\% \\
        1 & BANK\_B & \$39.96 & 40.0\% \\
        2 & BANK\_A & \$125.01 & 5.0\% \\
        2 & BANK\_B & \$69.97 & 30.0\% \\
        3 & BANK\_A & \$123.99 & 4.0\% \\
        3 & BANK\_B & \$67.96 & 28.0\% \\
        4 & BANK\_A & \$124.50 & 4.5\% \\
        4 & BANK\_B & \$69.46 & 29.5\% \\
        5 & BANK\_A & \$124.80 & 4.8\% \\
        5 & BANK\_B & \$69.88 & 29.9\% \\
        6 & BANK\_A & \$124.89 & 4.9\% \\
        6 & BANK\_B & \$68.98 & 29.0\% \\
        7 & BANK\_A & \$124.89 & 4.9\% \\
        7 & BANK\_B & \$68.98 & 29.0\% \\
        \hline
    \end{tabular}
\end{table}


\subsection{Pass 3}


\begin{figure}[H]
    \centering
    \includegraphics[width=0.85\textwidth]{charts/exp3_pass3_combined.png}
    \caption{Experiment 3: Symmetric Scenario - Pass 3 convergence}
    \label{fig:exp3_pass3_convergence}
\end{figure}



\begin{table}[H]
    \centering
    \caption{Experiment 3: Symmetric Scenario - Pass 3}
    \label{tab:exp3_pass3}
    \begin{tabular}{llrr}
        \hline
        Iteration & Agent & Cost & Liquidity \\
        \hline
        Baseline & BANK\_A & \$49.95 & 50.0\% \\
        Baseline & BANK\_B & \$49.95 & 50.0\% \\
        0 & BANK\_A & \$49.95 & 50.0\% \\
        0 & BANK\_B & \$49.95 & 50.0\% \\
        1 & BANK\_A & \$19.98 & 20.0\% \\
        1 & BANK\_B & \$19.98 & 20.0\% \\
        2 & BANK\_A & \$209.99 & 10.0\% \\
        2 & BANK\_B & \$200.99 & 1.0\% \\
        3 & BANK\_A & \$209.00 & 9.0\% \\
        3 & BANK\_B & \$200.51 & 0.5\% \\
        4 & BANK\_A & \$209.99 & 10.0\% \\
        4 & BANK\_B & \$200.90 & 0.9\% \\
        5 & BANK\_A & \$207.98 & 8.0\% \\
        5 & BANK\_B & \$200.99 & 1.0\% \\
        6 & BANK\_A & \$209.90 & 9.9\% \\
        6 & BANK\_B & \$200.96 & 0.9\% \\
        7 & BANK\_A & \$209.96 & 10.0\% \\
        7 & BANK\_B & \$200.96 & 0.9\% \\
        \hline
    \end{tabular}
\end{table}




\end{document}
