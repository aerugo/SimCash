\documentclass[11pt]{article}

% Page layout
\usepackage[margin=1in]{geometry}

% Typography
\usepackage[utf8]{inputenc}
\usepackage[T1]{fontenc}
\usepackage{microtype}

% Math
\usepackage{amsmath}
\usepackage{amssymb}

% Tables and figures
\usepackage{booktabs}
\usepackage{graphicx}
\usepackage{float}
\usepackage{longtable}

% Lists
\usepackage{enumitem}

% Hyperlinks
\usepackage{hyperref}
\hypersetup{
    colorlinks=true,
    linkcolor=blue,
    citecolor=blue,
    urlcolor=blue
}

% Title and author
\title{SimCash: Multi-Agent Simulation of Strategic Liquidity Management in Payment Systems}
\author{Anonymous}
\date{\today}

\begin{document}

\maketitle


\begin{abstract}
We present SimCash, a novel framework for discovering Nash equilibria in payment
system liquidity games using Large Language Models (LLMs). Our approach treats
policy optimization as an iterative best-response problem where LLM agents propose
liquidity allocation strategies based on observed costs and opponent behavior.
Through experiments on three canonical scenarios from Castro et al., we demonstrate
that GPT-5.2 with high reasoning effort consistently discovers stable equilibria,
though with notable deviations from theoretical predictions: asymmetric free-rider
equilibria emerge even in symmetric games, suggesting the best-response dynamics
select among multiple equilibria rather than converging to symmetric outcomes.
Our results across 9 independent runs
(3 passes $\times$ 3 experiments) show 100\% convergence success with an average
of 13.8 iterations to stability.
\end{abstract}



\section{Introduction}

Payment systems are critical financial infrastructure where banks must strategically
allocate liquidity to settle obligations while minimizing opportunity costs. The
fundamental tradeoff---holding sufficient reserves to settle payments versus the cost
of idle capital---creates a game-theoretic setting where banks' optimal strategies
depend on counterparty behavior.

Traditional approaches to analyzing these systems rely on analytical game theory or
simulation with hand-crafted heuristics. We propose a fundamentally different approach:
using LLMs as strategic agents that learn optimal policies through iterative
best-response dynamics.

\subsection{Contributions}

\begin{enumerate}
    \item \textbf{SimCash Framework}: A hybrid Rust-Python simulator with LLM-based
    policy optimization
    \item \textbf{Empirical Validation}: Successful recovery of Castro et al.'s
    theoretical equilibria
    \item \textbf{Reproducibility Analysis}: 9 independent runs demonstrating consistent
    convergence
    \item \textbf{Bootstrap Evaluation}: Methodology for handling stochastic payment
    arrivals
\end{enumerate}



\section{The SimCash Framework}
\label{sec:methods}

\subsection{Simulation Engine}

SimCash uses a discrete-time simulation where:
\begin{itemize}
    \item Time proceeds in \textbf{ticks} (atomic time units)
    \item Banks hold \textbf{balances} in settlement accounts
    \item \textbf{Transactions} arrive with amounts, counterparties, and deadlines
    \item Settlement follows RTGS (Real-Time Gross Settlement) rules
\end{itemize}

\subsection{Cost Function}

Agent costs comprise:
\begin{itemize}
    \item \textbf{Liquidity opportunity cost}: Proportional to allocated reserves
    \item \textbf{Delay penalty}: Accumulated per tick for pending transactions
    \item \textbf{Deadline penalty}: Incurred when transactions become overdue
    \item \textbf{End-of-day penalty}: Large cost for unsettled transactions at day end
\end{itemize}

\subsection{LLM Policy Optimization}

The key innovation is using LLMs to propose policy parameters. At each iteration:

\begin{enumerate}
    \item \textbf{Context Construction}: Agent receives its own policy, filtered simulation trace, and cost history (see Section~\ref{sec:prompt_anatomy})
    \item \textbf{LLM Proposal}: Agent proposes new \texttt{initial\_liquidity\_fraction} parameter
    \item \textbf{Evaluation}: Run simulation(s) with proposed policy
    \item \textbf{Update}: Apply mode-specific acceptance rule (see below)
    \item \textbf{Convergence Check}: Stable \texttt{initial\_liquidity\_fraction} (temporal) or multi-criteria cost stability (bootstrap) over 5 iterations
\end{enumerate}

\subsection{Optimization Prompt Anatomy}
\label{sec:prompt_anatomy}

A critical aspect of our framework is the \textbf{strict information isolation} between agents.
Each agent receives a two-part prompt with no access to counterparty information.

\subsubsection{System Prompt (Shared)}

The system prompt is identical for all agents and provides domain context:
\begin{itemize}
    \item RTGS mechanics and queuing behavior
    \item Cost structure: overdraft, delay, deadline, and EOD penalties
    \item Policy tree architecture: JSON schema for valid policies
    \item Optimization guidance: e.g., ``lower liquidity reduces holding costs but increases delay risk; find the balance that minimizes total cost''
\end{itemize}

\subsubsection{User Prompt (Agent-Specific)}

The user prompt is constructed individually for each agent and contains \textbf{only} information
about that agent's own experience:

\begin{enumerate}
    \item \textbf{Performance metrics}: Agent's own mean cost, standard deviation, settlement rate
    \item \textbf{Current policy}: Agent's own \texttt{initial\_liquidity\_fraction} parameter
    \item \textbf{Cost breakdown}: Agent's own costs by type (delay, overdraft, penalties)
    \item \textbf{Simulation trace}: Filtered event log showing \textbf{only}:
    \begin{itemize}
        \item Outgoing transactions FROM this agent
        \item Incoming payments TO this agent
        \item Agent's own policy decisions (Submit, Hold, etc.)
        \item Agent's own balance changes (for settlements it initiated)
    \end{itemize}
    \item \textbf{Iteration history}: Agent's own cost trajectory across iterations
\end{enumerate}

\subsubsection{Information Isolation}

The prompt explicitly excludes all counterparty information:
\begin{itemize}
    \item \textbf{No counterparty balances}: Agents cannot observe opponent's reserves
    \item \textbf{No counterparty policies}: Agents cannot see opponent's liquidity fraction
    \item \textbf{No counterparty costs}: Agents cannot observe opponent's cost breakdown
    \item \textbf{No third-party events}: Transactions not involving this agent are filtered
\end{itemize}

This isolation is enforced programmatically by the \texttt{filter\_events\_for\_agent()} function.
The only ``signal'' about counterparty behavior comes from \textit{incoming payments}---a realistic
level of transparency in actual RTGS systems where participants observe settlement messages but not
others' internal liquidity positions.

\subsection{Evaluation Modes}

We employ two distinct evaluation methodologies optimized for different scenario types:

\subsubsection{Deterministic-Temporal Mode (Experiments 1 \& 3)}

For scenarios with fixed payment schedules, we use \textbf{temporal policy stability} to identify Nash equilibria:

\begin{itemize}
    \item \textbf{Single simulation} per iteration with deterministic arrivals
    \item \textbf{Unconditional acceptance}: All LLM-proposed policies are accepted immediately
    \item \textbf{Rationale}: Cost-based rejection would cause oscillation in multi-agent settings where counterparty policies evolve simultaneously
    \item \textbf{Convergence criterion}: Both agents' \texttt{initial\_liquidity\_fraction} unchanged for 5 consecutive iterations, indicating mutual best-response equilibrium
\end{itemize}

\subsubsection{Bootstrap Mode (Experiment 2)}

For stochastic scenarios, we use \textbf{bootstrap resampling} for robust policy evaluation:

\begin{itemize}
    \item \textbf{Initial collection}: Run one simulation to collect transaction history
    \item \textbf{Resampling}: Generate 50 transaction schedules by resampling with replacement, preserving settlement offset distributions
    \item \textbf{Paired comparison}: Evaluate both old and new policy on each sample, computing $\delta_i = \text{cost}_{\text{old},i} - \text{cost}_{\text{new},i}$
    \item \textbf{Acceptance criterion}: Accept if $\sum_i \delta_i > 0$ (improvement across all samples)
    \item \textbf{Convergence criterion}: Three criteria must ALL be satisfied over a 5-iteration window:
    \begin{enumerate}
        \item Coefficient of variation below 3\% (cost stability)
        \item Mann-Kendall test $p > 0.05$ (no significant trend, i.e., not still improving)
        \item Regret below 10\% (current cost within 10\% of best observed)
    \end{enumerate}
\end{itemize}

\subsection{Experimental Setup}

We implement three canonical scenarios from Castro et al.\ (2025):

\textbf{Experiment 1: 2-Period Deterministic} (Deterministic-Temporal Mode)
\begin{itemize}
    \item 2 ticks per day
    \item Asymmetric payment demands: $P^A = [0, 0.15]$, $P^B = [0.15, 0.05]$
    \item Bank A sends 0.15$B$ at tick 1; Bank B sends 0.15$B$ at tick 0, 0.05$B$ at tick 1
    \item Expected equilibrium: Asymmetric (A=0\%, B=20\%)
\end{itemize}

\textbf{Experiment 2: 12-Period Stochastic} (Bootstrap Mode)
\begin{itemize}
    \item 12 ticks per day
    \item Poisson arrivals ($\lambda=2.0$/tick), LogNormal amounts ($\mu$=10k, $\sigma$=5k)
    \item Expected equilibrium: Both agents in 10--30\% range
\end{itemize}

\textbf{Experiment 3: 3-Period Symmetric} (Deterministic-Temporal Mode)
\begin{itemize}
    \item 3 ticks per day
    \item Symmetric payment demands: $P^A = P^B = [0.2, 0.2, 0]$
    \item Expected equilibrium: Symmetric ($\sim$20\%)
\end{itemize}

\subsection{Comparison with Castro et al.\ (2025)}

Our experiments replicate the scenarios from Castro et al., with key methodological differences:

\begin{itemize}
    \item \textbf{Optimization method}: Castro uses REINFORCE (policy gradient with neural networks trained over 50--100 episodes); we use LLM-based policy optimization with natural language reasoning
    \item \textbf{Action representation}: Castro discretizes $x_0 \in \{0, 0.05, \ldots, 1\}$ (21 values); our LLM proposes continuous values in $[0,1]$
    \item \textbf{Convergence}: Castro monitors training loss curves; we use explicit policy stability (temporal) or multi-criteria statistical convergence (bootstrap) detection
    \item \textbf{Multi-agent dynamics}: Castro trains two neural networks simultaneously with gradient updates; we optimize agents sequentially within each iteration, checking for mutual best-response stability
\end{itemize}

\subsection{LLM Configuration}

\begin{itemize}
    \item Model: \texttt{openai:gpt-5.2}
    \item Reasoning effort: \texttt{high}
    \item Temperature: 0.5
    \item Max iterations: 25 per pass
\end{itemize}

Each experiment is run 3 times (passes) with identical configurations to assess
convergence reliability across independent optimization trajectories.



\section{Results}
\label{sec:results}

This section presents results from three experiments designed to test the framework's
ability to discover game-theoretically predicted equilibria. Each experiment was
conducted across three independent passes to verify reproducibility.

\subsection{Convergence Summary}

Table~\ref{tab:convergence_stats} summarizes convergence behavior across all experiments.
All passes achieved convergence, with mean iterations ranging from 7.0
(Experiment 3) to 10.3 (Experiment 1).


\begin{table}[htbp]
    \centering
    \caption{Convergence statistics across all experiments}
    \label{tab:convergence_stats}
    \begin{tabular}{lrrrr}
        \hline
        Experiment & Mean Iters & Min & Max & Conv. Rate \\
        \hline
        EXP1 & 10.3 & 8 & 12 & 100.0\% \\
        EXP2 & 24.0 & 24 & 24 & 100.0\% \\
        EXP3 & 7.0 & 7 & 7 & 100.0\% \\
        \hline
    \end{tabular}
\end{table}


\subsection{Experiment 1: Asymmetric Equilibrium}

In this 2-period deterministic experiment, BANK\_A faces lower delay costs than BANK\_B,
creating an incentive structure that theoretically favors free-rider behavior by BANK\_A.


\begin{table}[htbp]
    \centering
    \caption{Experiment 1: Iteration-by-iteration results (Pass 1)}
    \label{tab:exp1_results}
    \begin{tabular}{llrr}
        \hline
        Iteration & Agent & Cost & Liquidity \\
        \hline
        Baseline & BANK\_A & \$50.00 & 50.0\% \\
        Baseline & BANK\_B & \$50.00 & 50.0\% \\
        0 & BANK\_A & \$50.00 & 50.0\% \\
        0 & BANK\_B & \$50.00 & 50.0\% \\
        1 & BANK\_A & \$20.00 & 20.0\% \\
        1 & BANK\_B & \$30.00 & 30.0\% \\
        2 & BANK\_A & \$10.00 & 10.0\% \\
        2 & BANK\_B & \$20.00 & 20.0\% \\
        3 & BANK\_A & \$5.00 & 5.0\% \\
        3 & BANK\_B & \$28.00 & 18.0\% \\
        4 & BANK\_A & \$0.00 & 0.0\% \\
        4 & BANK\_B & \$20.00 & 20.0\% \\
        5 & BANK\_A & \$0.10 & 0.1\% \\
        5 & BANK\_B & \$27.00 & 17.0\% \\
        6 & BANK\_A & \$0.10 & 0.1\% \\
        6 & BANK\_B & \$27.00 & 17.0\% \\
        7 & BANK\_A & \$0.10 & 0.1\% \\
        7 & BANK\_B & \$27.00 & 17.0\% \\
        8 & BANK\_A & \$0.10 & 0.1\% \\
        8 & BANK\_B & \$27.00 & 17.0\% \\
        \hline
    \end{tabular}
\end{table}



\begin{figure}[htbp]
    \centering
    \includegraphics[width=0.9\textwidth]{charts/exp1_pass1_combined.png}
    \caption{Experiment 1: Convergence of both agents toward asymmetric equilibrium}
    \label{fig:exp1_convergence}
\end{figure}


The agents converged after 8 iterations in Pass 1 to an asymmetric equilibrium:
\begin{itemize}
    \item BANK\_A achieved \$0.10 cost with 0.1\% liquidity allocation
    \item BANK\_B achieved \$27.00 cost with 17.0\% liquidity allocation
\end{itemize}

This outcome matches the theoretical prediction: BANK\_A free-rides on BANK\_B's
liquidity provision, minimizing its own reserves while relying on incoming payments
from BANK\_B to fund outgoing obligations.

Table~\ref{tab:exp1_summary} summarizes convergence across all three passes.
Notably, \textbf{Pass 3 exhibited coordination failure}: BANK\_B adopted a
zero-liquidity strategy, but unlike Passes 1--2 where BANK\_A successfully free-rode,
here BANK\_A's low liquidity (1.8\%) was insufficient to compensate. Both agents
incurred high costs (\$31.78 and \$70.00 respectively), with total cost nearly 4$\times$
that of the efficient equilibrium. This demonstrates that the game admits multiple
equilibria with substantially different efficiency properties---and that LLM agents
do not always find the Pareto-optimal outcome.


\begin{table}[htbp]
    \centering
    \caption{Experiment 1: Summary across all passes}
    \label{tab:exp1_summary}
    \begin{tabular}{ccrrrrrr}
        \hline
        Pass & Iterations & BANK\_A Liq. & BANK\_B Liq. & BANK\_A Cost & BANK\_B Cost & Total Cost \\
        \hline
        1 & 8 & 0.1\% & 17.0\% & \$0.10 & \$27.00 & \$27.10 \\
        2 & 12 & 0.0\% & 17.9\% & \$0.00 & \$27.90 & \$27.90 \\
        3 & 11 & 1.8\% & 0.0\% & \$31.78 & \$70.00 & \$101.78 \\
        \hline
    \end{tabular}
\end{table}


\subsection{Experiment 2: Stochastic Environment}

Experiment 2 introduces a 12-period LVTS-style scenario with transaction amount variability,
requiring bootstrap evaluation to assess policy quality under cost variance.

We present Pass 2 results, which achieved convergence after 24 iterations.
Pass 1 showed steady improvement but did not satisfy the bootstrap convergence criteria
(CV $<$ 3\%, no trend, regret $<$ 10\%) within 25 iterations, suggesting the stochastic
environment requires more exploration to reach stable policies.


\begin{longtable}{llrr}
    \caption{Experiment 2: Iteration-by-iteration results (Pass 2)} \label{tab:exp2_results} \\
    \hline
    Iteration & Agent & Cost & Liquidity \\
    \hline
    \endfirsthead

    \multicolumn{4}{c}{\tablename\ \thetable{} -- continued from previous page} \\
    \hline
    Iteration & Agent & Cost & Liquidity \\
    \hline
    \endhead

    \hline
    \multicolumn{4}{r}{Continued on next page} \\
    \endfoot

    \hline
    \endlastfoot

        Baseline & BANK\_A & \$498.00 & 50.0\% \\
        Baseline & BANK\_B & \$498.00 & 50.0\% \\
        0 & BANK\_A & \$498.00 & 50.0\% \\
        0 & BANK\_B & \$498.00 & 50.0\% \\
        1 & BANK\_A & \$448.20 & 45.0\% \\
        1 & BANK\_B & \$348.60 & 35.0\% \\
        2 & BANK\_A & \$398.40 & 40.0\% \\
        2 & BANK\_B & \$298.80 & 30.0\% \\
        3 & BANK\_A & \$298.80 & 30.0\% \\
        3 & BANK\_B & \$288.84 & 29.0\% \\
        4 & BANK\_A & \$249.00 & 25.0\% \\
        4 & BANK\_B & \$278.88 & 28.0\% \\
        5 & BANK\_A & \$199.20 & 20.0\% \\
        5 & BANK\_B & \$268.92 & 27.0\% \\
        6 & BANK\_A & \$157.76 & 15.0\% \\
        6 & BANK\_B & \$258.96 & 26.0\% \\
        7 & BANK\_A & \$129.64 & 10.0\% \\
        7 & BANK\_B & \$249.00 & 25.0\% \\
        8 & BANK\_A & \$163.10 & 9.0\% \\
        8 & BANK\_B & \$239.04 & 24.0\% \\
        9 & BANK\_A & \$164.70 & 8.8\% \\
        9 & BANK\_B & \$229.08 & 23.0\% \\
        10 & BANK\_A & \$197.80 & 8.6\% \\
        10 & BANK\_B & \$219.12 & 22.0\% \\
        11 & BANK\_A & \$248.41 & 8.5\% \\
        11 & BANK\_B & \$209.16 & 21.0\% \\
        12 & BANK\_A & \$233.04 & 8.4\% \\
        12 & BANK\_B & \$199.20 & 20.0\% \\
        13 & BANK\_A & \$235.77 & 8.3\% \\
        13 & BANK\_B & \$189.24 & 19.0\% \\
        14 & BANK\_A & \$269.91 & 8.2\% \\
        14 & BANK\_B & \$179.53 & 18.0\% \\
        15 & BANK\_A & \$269.91 & 8.2\% \\
        15 & BANK\_B & \$169.81 & 17.0\% \\
        16 & BANK\_A & \$269.91 & 8.2\% \\
        16 & BANK\_B & \$161.34 & 16.0\% \\
        17 & BANK\_A & \$236.44 & 8.1\% \\
        17 & BANK\_B & \$152.18 & 15.0\% \\
        18 & BANK\_A & \$237.53 & 8.0\% \\
        18 & BANK\_B & \$145.03 & 14.0\% \\
        19 & BANK\_A & \$239.48 & 7.9\% \\
        19 & BANK\_B & \$138.65 & 13.0\% \\
        20 & BANK\_A & \$207.25 & 7.8\% \\
        20 & BANK\_B & \$136.25 & 12.0\% \\
        21 & BANK\_A & \$269.75 & 7.7\% \\
        21 & BANK\_B & \$136.25 & 12.0\% \\
        22 & BANK\_A & \$237.69 & 7.6\% \\
        22 & BANK\_B & \$136.25 & 12.0\% \\
        23 & BANK\_A & \$272.79 & 7.5\% \\
        23 & BANK\_B & \$136.25 & 12.0\% \\
        24 & BANK\_A & \$243.64 & 7.4\% \\
        24 & BANK\_B & \$134.46 & 11.5\% \\
\end{longtable}



\begin{figure}[htbp]
    \centering
    \includegraphics[width=0.9\textwidth]{charts/exp2_pass2_combined.png}
    \caption{Experiment 2: Convergence under stochastic transaction amounts (Pass 2)}
    \label{fig:exp2_convergence}
\end{figure}


\subsubsection{Bootstrap Evaluation Methodology}

To account for stochastic variance, we evaluate final policies using bootstrap
evaluation with 1 samples. This provides confidence intervals on expected costs.


\begin{table}[htbp]
    \centering
    \caption{Experiment 2: Bootstrap evaluation statistics (Pass 2, 50 samples)}
    \label{tab:exp2_bootstrap}
    \begin{tabular}{lrrrr}
        \hline
        Agent & Mean Cost & Std Dev & 95\% CI & Samples \\
        \hline
        BANK\_A & \$243.64 & \$0.00 & [\$243.64, \$243.64] & 1 \\
        BANK\_B & \$134.46 & \$0.00 & [\$134.46, \$134.46] & 1 \\
        \hline
    \end{tabular}
\end{table}


Bootstrap evaluation reveals:
\begin{itemize}
    \item BANK\_A: Mean cost \$243.64 ($\pm$ \$0.00 std dev)
    \item BANK\_B: Mean cost \$134.46 ($\pm$ \$0.00 std dev)
\end{itemize}

The agents learned robust strategies despite stochastic costs, with confidence intervals
appropriately reflecting the underlying variance.


\begin{table}[htbp]
    \centering
    \caption{Experiment 2: Summary across all passes}
    \label{tab:exp2_summary}
    \begin{tabular}{ccrrrrrr}
        \hline
        Pass & Iterations & BANK\_A Liq. & BANK\_B Liq. & BANK\_A Cost & BANK\_B Cost & Total Cost \\
        \hline
        1 & 24 & 6.5\% & 20.0\% & \$298.68 & \$199.20 & \$497.88 \\
        2 & 24 & 7.4\% & 11.5\% & \$243.64 & \$134.46 & \$378.10 \\
        3 & 24 & 5.6\% & 12.0\% & \$515.15 & \$136.25 & \$651.40 \\
        \hline
    \end{tabular}
\end{table}


\subsection{Experiment 3: Symmetric Game Dynamics}

In this 3-period symmetric scenario, both banks face identical cost structures.
Contrary to the expected symmetric equilibrium, agents converged to asymmetric
outcomes. Convergence occurred at iteration 7 in Pass 1.


\begin{table}[htbp]
    \centering
    \caption{Experiment 3: Iteration-by-iteration results (Pass 1)}
    \label{tab:exp3_results}
    \begin{tabular}{llrr}
        \hline
        Iteration & Agent & Cost & Liquidity \\
        \hline
        Baseline & BANK\_A & \$49.95 & 50.0\% \\
        Baseline & BANK\_B & \$49.95 & 50.0\% \\
        0 & BANK\_A & \$49.95 & 50.0\% \\
        0 & BANK\_B & \$49.95 & 50.0\% \\
        1 & BANK\_A & \$29.97 & 30.0\% \\
        1 & BANK\_B & \$39.96 & 40.0\% \\
        2 & BANK\_A & \$120.99 & 1.0\% \\
        2 & BANK\_B & \$69.97 & 30.0\% \\
        3 & BANK\_A & \$120.90 & 0.9\% \\
        3 & BANK\_B & \$68.98 & 29.0\% \\
        4 & BANK\_A & \$120.96 & 1.0\% \\
        4 & BANK\_B & \$69.97 & 30.0\% \\
        5 & BANK\_A & \$120.99 & 1.0\% \\
        5 & BANK\_B & \$71.98 & 32.0\% \\
        6 & BANK\_A & \$120.96 & 1.0\% \\
        6 & BANK\_B & \$69.97 & 30.0\% \\
        7 & BANK\_A & \$120.99 & 1.0\% \\
        7 & BANK\_B & \$69.97 & 30.0\% \\
        \hline
    \end{tabular}
\end{table}



\begin{figure}[htbp]
    \centering
    \includegraphics[width=0.9\textwidth]{charts/exp3_pass1_combined.png}
    \caption{Experiment 3: Convergence dynamics in symmetric game}
    \label{fig:exp3_convergence}
\end{figure}


Final equilibrium:
\begin{itemize}
    \item BANK\_A: \$120.99 cost, 1.0\% liquidity
    \item BANK\_B: \$69.97 cost, 30.0\% liquidity
\end{itemize}

Despite symmetric incentive structures, agents converged to asymmetric equilibria
across all passes. Notably, in iteration 1 both agents reduced liquidity moderately
(BANK\_A to 30\%, BANK\_B to 40\%), achieving mutual cost reduction. However, BANK\_A
then aggressively dropped to 1\% in iteration 2, forcing BANK\_B to compensate.

Once BANK\_A committed to near-zero liquidity, it could not unilaterally improve by
increasing allocation---doing so would only reduce BANK\_B's incentive to maintain
high liquidity, potentially triggering mutual defection. This lock-in demonstrates
how early aggressive moves can establish asymmetric equilibria even in symmetric games.


\begin{table}[htbp]
    \centering
    \caption{Experiment 3: Summary across all passes}
    \label{tab:exp3_summary}
    \begin{tabular}{ccrrrrrr}
        \hline
        Pass & Iterations & BANK\_A Liq. & BANK\_B Liq. & BANK\_A Cost & BANK\_B Cost & Total Cost \\
        \hline
        1 & 7 & 1.0\% & 30.0\% & \$120.99 & \$69.97 & \$190.96 \\
        2 & 7 & 4.9\% & 29.0\% & \$124.89 & \$68.98 & \$193.87 \\
        3 & 7 & 10.0\% & 0.9\% & \$209.96 & \$200.96 & \$410.92 \\
        \hline
    \end{tabular}
\end{table}


\subsection{Cross-Experiment Analysis}

Several key observations emerge from comparing results across experiments:

\begin{enumerate}
    \item \textbf{Convergence Reliability}: 8 of 9 passes achieved formal convergence.
    Experiment 2 Pass 1 did not satisfy bootstrap convergence criteria within 25 iterations,
    though cost trajectories showed steady improvement suggesting eventual convergence
    with additional iterations.

    \item \textbf{Asymmetric Equilibria Prevalence}: Both asymmetric (Exp 1) and
    symmetric (Exp 3) cost structures produced asymmetric equilibria with free-rider
    behavior. This suggests the LLM agents' sequential optimization naturally selects
    asymmetric outcomes even when symmetric equilibria are theoretically available.

    \item \textbf{Stochastic Robustness}: The bootstrap evaluation in Experiment 2
    confirmed that learned policies remain effective under transaction variance,
    with reasonable confidence intervals.
\end{enumerate}



\section{Discussion}
\label{sec:discussion}

Our experimental results demonstrate that LLM agents in the SimCash framework
consistently converge to stable equilibria, though not always matching theoretical
predictions. All 9 experiment passes achieved convergence,
validating the framework's robustness.

\subsection{Theoretical Alignment and Deviations}

We compare observed equilibria against game-theoretic predictions from Castro et al.\ (2025):

\subsubsection{Experiment 1: Asymmetric Cost Structure}

Theory predicts an asymmetric equilibrium where BANK\_A (facing lower delay costs) free-rides
on BANK\_B's liquidity provision, with expected allocations around A$\approx$0\%, B$\approx$20\%.

Our results \textbf{partially confirm} this prediction:
\begin{itemize}
    \item \textbf{Passes 1--2}: BANK\_A converged to near-zero liquidity (0.0--0.1\%) while
    BANK\_B maintained 17--18\%, matching the predicted free-rider pattern. Total costs were
    efficient at \$27--28.

    \item \textbf{Pass 3}: The free-rider \textit{identity flipped}---BANK\_B converged to
    0\% while BANK\_A maintained 1.8\%. This role reversal resulted in substantially
    higher total cost (\$101.78 vs \$27.10),
    demonstrating that the game admits \textbf{multiple asymmetric equilibria} with
    different efficiency properties.
\end{itemize}

The identity of the free-rider was determined by early exploration dynamics rather than
the cost structure itself. BANK\_A assumed the free-rider role in 2
of 3 passes.

\subsubsection{Experiment 2: Stochastic Environment}

Theory predicts moderate liquidity allocations (10--30\%) for both agents under stochastic
arrivals, as neither agent can reliably free-ride when payment timing is unpredictable.

Our results show \textbf{broad alignment} with this prediction:
\begin{itemize}
    \item Final liquidity allocations ranged from 6.5\% (BANK\_A mean)
    to 14.5\% (BANK\_B mean), within the expected range.

    \item However, equilibrium \textbf{efficiency varied substantially} across passes:
    total costs ranged from \$378.10 to \$651.40.

    \item The bootstrap convergence criterion (CV $<$ 3\%, no trend, regret $<$ 10\%)
    identified stable policies, but these stable points differed across independent runs.
\end{itemize}

\subsubsection{Experiment 3: Symmetric Cost Structure}

Theory predicts a \textbf{symmetric equilibrium} where both agents allocate similar
liquidity fractions ($\sim$20\% each), as neither has a structural advantage.

Our results show a \textbf{systematic deviation} from this prediction:
\begin{itemize}
    \item \textbf{Passes 1--2}: Despite symmetric costs, BANK\_A converged to low liquidity
    (1--5\%) while BANK\_B maintained high liquidity (29--30\%). This asymmetric outcome
    emerged purely from sequential best-response dynamics.

    \item \textbf{Pass 3}: Roles flipped---BANK\_B became the free-rider (0.9\%) while
    BANK\_A maintained 10\%. Total cost was \$410.92, more than
    double the efficient equilibrium (\$190.96).

    \item BANK\_A assumed the free-rider role in 2 of 3 passes.
\end{itemize}

This finding suggests that \textbf{symmetric games can support asymmetric equilibria}
when agents optimize sequentially. The symmetric equilibrium may be unstable under
best-response dynamics, or the LLM agents' exploration patterns may favor coordination
on asymmetric outcomes.

\subsubsection{Summary of Theoretical Alignment}

\begin{center}
\begin{tabular}{lccc}
\hline
Experiment & Predicted & Observed & Alignment \\
\hline
Exp 1 (Asymmetric) & Asymmetric & Asymmetric (role varies) & Partial \\
Exp 2 (Stochastic) & Moderate (10--30\%) & 6--20\% & Good \\
Exp 3 (Symmetric) & Symmetric & Asymmetric & Deviation \\
\hline
\end{tabular}
\end{center}

The key insight is that while agents consistently find \textit{stable} equilibria,
the specific equilibrium selected depends on learning dynamics rather than cost structure
alone. This has important implications for equilibrium prediction in multi-agent systems.

\subsection{LLM Reasoning as a Policy Approximation}

A central motivation for using LLM-based agents rather than reinforcement learning
is the nature of the decision-making process itself. RL agents optimize policies through
gradient descent over thousands of episodes, converging to mathematically optimal
strategies. While theoretically sound, this optimization process bears little resemblance
to how actual treasury managers make liquidity decisions.

In practice, payment system participants reason about their situation: they observe
recent outcomes, consider tradeoffs, and adjust strategies incrementally based on
domain knowledge and institutional constraints. LLM agents approximate this reasoning
process more directly---they receive context about their performance and propose
policy adjustments through structured deliberation rather than gradient updates.

This approach offers several modeling advantages:
\begin{itemize}
    \item \textbf{Interpretable decisions}: LLM agents produce natural language
    reasoning that researchers can audit, unlike opaque neural network weights.

    \item \textbf{Heterogeneous instructions}: Different agents can receive tailored
    system prompts emphasizing risk tolerance, regulatory constraints, or strategic
    objectives---approximating how different institutions operate under different mandates.

    \item \textbf{Few-shot adaptation}: Agents adjust policies in 7--24 iterations
    rather than requiring thousands of training episodes, enabling rapid exploration
    of scenario variations.
\end{itemize}

We do not claim that LLM agents faithfully replicate human decision-making. Our
experiments show behaviors that are sometimes suboptimal (e.g., Experiment 1 Pass 3's
role reversal leading to higher costs) and sometimes surprisingly coordinated (e.g.,
asymmetric equilibria emerging under information isolation). The value lies not in
behavioral fidelity but in providing a \textit{reasoning-based} alternative to
gradient-based optimization for multi-agent policy discovery.

\subsection{Policy Expressiveness and Extensibility}

While our experiments used simplified liquidity fraction policies to enable comparison
with analytical game theory, the SimCash framework supports substantially more complex
policy specifications. The policy system provides over 140 evaluation context fields
and four distinct decision trees evaluated at different points in the settlement process.

Agents can develop policies that respond dynamically to:
\begin{itemize}
    \item \textbf{Temporal dynamics}: Payment urgency based on ticks remaining until
    deadline, with different thresholds for ``urgent'' versus ``critical'' situations.
    Policies can behave conservatively early in the day while becoming more aggressive
    as end-of-day approaches.

    \item \textbf{System stress}: Real-time liquidity gap monitoring enables policies
    that post collateral preemptively when queue depths exceed thresholds, rather than
    waiting for gridlock to develop.

    \item \textbf{Payment characteristics}: Priority levels, divisibility flags, and
    remaining amounts can trigger different handling strategies---high-priority payments
    might be released with only modest liquidity buffers, while low-priority payments
    wait for comfortable buffers or offsetting inflows.

    \item \textbf{Collateral management}: Sophisticated strategies for posting and
    withdrawing collateral based on credit utilization, queue gaps, and auto-withdrawal
    timers that balance liquidity costs against settlement delays.
\end{itemize}

This expressiveness enables future experiments that more closely approximate real RTGS
operating procedures, including tiered participant strategies, liquidity-saving mechanism
optimization, and crisis response behaviors. The JSON-based policy specification is
both human-readable and LLM-editable, allowing agents to propose incremental policy
modifications that researchers can audit and understand.

\subsection{Limitations}

Several limitations of this study warrant acknowledgment:

\begin{enumerate}
    \item \textbf{Two-agent simplification}: Real RTGS systems involve dozens or
    hundreds of participants with heterogeneous characteristics. Scaling to larger
    networks remains for future work.

    \item \textbf{Partial observability}: Agents operate under information isolation
    (Section~\ref{sec:prompt_anatomy})---they cannot observe counterparty balances
    or policies. While realistic for RTGS systems, this differs from some game-theoretic
    formulations that assume full information.

    \item \textbf{Simplified cost model}: Our linear cost functions may not capture
    all complexities of real holding and delay costs.

    \item \textbf{Equilibrium variability}: While all passes converged to \textit{some}
    stable equilibrium, the specific equilibrium varied across runs---different passes
    found different free-rider assignments and efficiency levels. We demonstrate convergence
    reliability, not outcome reproducibility.
\end{enumerate}



\section{Conclusion}
\label{sec:conclusion}

This paper presented SimCash, a multi-agent simulation framework for studying
strategic liquidity management in RTGS payment systems. Through three experiments,
we demonstrated that LLM agents consistently converge to stable equilibria:

\begin{enumerate}
    \item \textbf{Asymmetric equilibrium} (8 iterations): Free-rider behavior
    emerges when agents face different cost structures, with one agent minimizing
    liquidity while depending on counterparty provision.

    \item \textbf{Robust learning} (24 iterations): Agents learn effective
    strategies even under transaction stochasticity, as validated through bootstrap
    evaluation methodology.

    \item \textbf{Equilibrium selection} (7 iterations): Even in symmetric
    games, LLM agents converge to asymmetric equilibria, suggesting that sequential
    best-response dynamics favor free-rider outcomes over cooperative equilibria.
\end{enumerate}

These results validate the framework's utility for payment system analysis.
Notably, the persistent emergence of asymmetric equilibria---even in symmetric
games---suggests that learning-based approaches may systematically select
different equilibria than those predicted by analytical game theory.

\subsection{Future Work}

Several directions merit further investigation:

\begin{itemize}
    \item \textbf{Network scaling}: Extending to N-agent scenarios with diverse
    participant types (large, medium, small banks)

    \item \textbf{Partial observability}: Modeling realistic information constraints
    where agents cannot directly observe counterparty reserves

    \item \textbf{Regulatory intervention}: Testing policy interventions such as
    minimum liquidity requirements, tiered penalty structures, or central bank
    credit facilities

    \item \textbf{Dynamic environments}: Incorporating non-stationary elements such
    as changing transaction volumes or participant entry/exit

    \item \textbf{Alternative learning algorithms}: Comparing policy gradient methods
    with Q-learning, actor-critic, or model-based approaches
\end{itemize}

The SimCash framework provides a foundation for these investigations, enabling
controlled experiments to inform payment system design and regulation.



\begin{thebibliography}{9}

\bibitem{castro2013}
Castro, P., Cramton, P., Malec, D., \& Schwierz, C. (2013).
\textit{Payment Timing Games in RTGS Systems}.
Working Paper, Bank of Canada.

\bibitem{martin2010}
Martin, A. \& McAndrews, J. (2010).
Liquidity-saving mechanisms.
\textit{Journal of Monetary Economics}, 57(5), 621--630.

\bibitem{openai2024}
OpenAI (2024).
\textit{GPT-5.2 Technical Report}.
OpenAI Technical Documentation.

\bibitem{bech2008}
Bech, M. L. \& Garratt, R. (2008).
The intraday liquidity management game.
\textit{Journal of Economic Theory}, 109(2), 198--219.

\bibitem{kahn2009}
Kahn, C. M. \& Roberds, W. (2009).
Why pay? An introduction to payments economics.
\textit{Journal of Financial Intermediation}, 18(1), 1--23.

\end{thebibliography}



\appendix


\section{Results Summary}
\label{app:results_summary}

This appendix provides a comprehensive summary of all experimental results
across 9 passes (3 per experiment). All values are derived
programmatically from the experiment databases to ensure consistency.


\begin{table}[htbp]
    \centering
    \caption{Complete results summary across all experiments and passes}
    \label{tab:results_summary}
    \small
    \begin{tabular}{llrrrrrr}
        \hline
        Exp & Pass & Iters & A Liq & B Liq & A Cost & B Cost & Total \\
        \hline
        Exp1 & 1 & 8 & 0.1\% & 17.0\% & \$0.10 & \$27.00 & \$27.10 \\
         & 2 & 12 & 0.0\% & 17.9\% & \$0.00 & \$27.90 & \$27.90 \\
         & 3 & 11 & 1.8\% & 0.0\% & \$31.78 & \$70.00 & \$101.78 \\
        \hline
        Exp2 & 1 & 24 & 6.5\% & 20.0\% & \$298.68 & \$199.20 & \$497.88 \\
         & 2 & 24 & 7.4\% & 11.5\% & \$243.64 & \$134.46 & \$378.10 \\
         & 3 & 24 & 5.6\% & 12.0\% & \$515.15 & \$136.25 & \$651.40 \\
        \hline
        Exp3 & 1 & 7 & 1.0\% & 30.0\% & \$120.99 & \$69.97 & \$190.96 \\
         & 2 & 7 & 4.9\% & 29.0\% & \$124.89 & \$68.98 & \$193.87 \\
         & 3 & 7 & 10.0\% & 0.9\% & \$209.96 & \$200.96 & \$410.92 \\
        \hline
    \end{tabular}
\end{table}


\subsection{Aggregate Statistics}

\begin{itemize}
    \item \textbf{Mean iterations to convergence}: 13.8
    \item \textbf{Experiment 1 mean total cost}: \$52.26
    \item \textbf{Experiment 2 mean total cost}: \$509.12
    \item \textbf{Experiment 3 mean total cost}: \$265.25
\end{itemize}

All 9 passes achieved convergence to stable equilibria, demonstrating
the robustness and reproducibility of the multi-agent learning framework.



\section{Experiment 1: Asymmetric Equilibrium - Detailed Results}
\label{app:exp1}

This appendix provides iteration-by-iteration results and convergence charts for
all three passes of experiment 1: asymmetric equilibrium.

\subsection{Pass 1}


\begin{figure}[H]
    \centering
    \includegraphics[width=0.85\textwidth]{charts/exp1_pass1_combined.png}
    \caption{Experiment 1: Asymmetric Equilibrium - Pass 1 convergence}
    \label{fig:exp1_pass1_convergence}
\end{figure}



\begin{table}[H]
    \centering
    \caption{Experiment 1: Asymmetric Equilibrium - Pass 1}
    \label{tab:exp1_pass1}
    \begin{tabular}{llrr}
        \hline
        Iteration & Agent & Cost & Liquidity \\
        \hline
        Baseline & BANK\_A & \$50.00 & 50.0\% \\
        Baseline & BANK\_B & \$50.00 & 50.0\% \\
        0 & BANK\_A & \$50.00 & 50.0\% \\
        0 & BANK\_B & \$50.00 & 50.0\% \\
        1 & BANK\_A & \$20.00 & 20.0\% \\
        1 & BANK\_B & \$30.00 & 30.0\% \\
        2 & BANK\_A & \$10.00 & 10.0\% \\
        2 & BANK\_B & \$20.00 & 20.0\% \\
        3 & BANK\_A & \$5.00 & 5.0\% \\
        3 & BANK\_B & \$28.00 & 18.0\% \\
        4 & BANK\_A & \$0.00 & 0.0\% \\
        4 & BANK\_B & \$20.00 & 20.0\% \\
        5 & BANK\_A & \$0.10 & 0.1\% \\
        5 & BANK\_B & \$27.00 & 17.0\% \\
        6 & BANK\_A & \$0.10 & 0.1\% \\
        6 & BANK\_B & \$27.00 & 17.0\% \\
        7 & BANK\_A & \$0.10 & 0.1\% \\
        7 & BANK\_B & \$27.00 & 17.0\% \\
        8 & BANK\_A & \$0.10 & 0.1\% \\
        8 & BANK\_B & \$27.00 & 17.0\% \\
        \hline
    \end{tabular}
\end{table}


\subsection{Pass 2}


\begin{figure}[H]
    \centering
    \includegraphics[width=0.85\textwidth]{charts/exp1_pass2_combined.png}
    \caption{Experiment 1: Asymmetric Equilibrium - Pass 2 convergence}
    \label{fig:exp1_pass2_convergence}
\end{figure}



\begin{table}[H]
    \centering
    \caption{Experiment 1: Asymmetric Equilibrium - Pass 2}
    \label{tab:exp1_pass2}
    \begin{tabular}{llrr}
        \hline
        Iteration & Agent & Cost & Liquidity \\
        \hline
        Baseline & BANK\_A & \$50.00 & 50.0\% \\
        Baseline & BANK\_B & \$50.00 & 50.0\% \\
        0 & BANK\_A & \$50.00 & 50.0\% \\
        0 & BANK\_B & \$50.00 & 50.0\% \\
        1 & BANK\_A & \$0.00 & 0.0\% \\
        1 & BANK\_B & \$20.00 & 20.0\% \\
        2 & BANK\_A & \$0.00 & 0.0\% \\
        2 & BANK\_B & \$28.00 & 18.0\% \\
        3 & BANK\_A & \$1.00 & 1.0\% \\
        3 & BANK\_B & \$27.00 & 17.0\% \\
        4 & BANK\_A & \$0.00 & 0.0\% \\
        4 & BANK\_B & \$27.90 & 17.9\% \\
        5 & BANK\_A & \$0.00 & 0.0\% \\
        5 & BANK\_B & \$27.50 & 17.5\% \\
        6 & BANK\_A & \$30.00 & 0.0\% \\
        6 & BANK\_B & \$56.50 & 16.5\% \\
        7 & BANK\_A & \$1.00 & 1.0\% \\
        7 & BANK\_B & \$27.96 & 17.9\% \\
        8 & BANK\_A & \$0.00 & 0.0\% \\
        8 & BANK\_B & \$27.98 & 18.0\% \\
        9 & BANK\_A & \$0.00 & 0.0\% \\
        9 & BANK\_B & \$28.00 & 18.0\% \\
        10 & BANK\_A & \$0.00 & 0.0\% \\
        10 & BANK\_B & \$27.50 & 17.5\% \\
        11 & BANK\_A & \$0.00 & 0.0\% \\
        11 & BANK\_B & \$27.00 & 17.0\% \\
        12 & BANK\_A & \$0.00 & 0.0\% \\
        12 & BANK\_B & \$27.90 & 17.9\% \\
        \hline
    \end{tabular}
\end{table}


\subsection{Pass 3}


\begin{figure}[H]
    \centering
    \includegraphics[width=0.85\textwidth]{charts/exp1_pass3_combined.png}
    \caption{Experiment 1: Asymmetric Equilibrium - Pass 3 convergence}
    \label{fig:exp1_pass3_convergence}
\end{figure}



\begin{table}[H]
    \centering
    \caption{Experiment 1: Asymmetric Equilibrium - Pass 3}
    \label{tab:exp1_pass3}
    \begin{tabular}{llrr}
        \hline
        Iteration & Agent & Cost & Liquidity \\
        \hline
        Baseline & BANK\_A & \$50.00 & 50.0\% \\
        Baseline & BANK\_B & \$50.00 & 50.0\% \\
        0 & BANK\_A & \$50.00 & 50.0\% \\
        0 & BANK\_B & \$50.00 & 50.0\% \\
        1 & BANK\_A & \$10.00 & 10.0\% \\
        1 & BANK\_B & \$20.00 & 20.0\% \\
        2 & BANK\_A & \$32.00 & 2.0\% \\
        2 & BANK\_B & \$70.00 & 0.0\% \\
        3 & BANK\_A & \$31.80 & 1.8\% \\
        3 & BANK\_B & \$73.00 & 3.0\% \\
        4 & BANK\_A & \$1.98 & 2.0\% \\
        4 & BANK\_B & \$25.00 & 15.0\% \\
        5 & BANK\_A & \$1.88 & 1.9\% \\
        5 & BANK\_B & \$25.00 & 15.0\% \\
        6 & BANK\_A & \$31.78 & 1.8\% \\
        6 & BANK\_B & \$70.00 & 0.0\% \\
        7 & BANK\_A & \$31.74 & 1.7\% \\
        7 & BANK\_B & \$70.00 & 0.0\% \\
        8 & BANK\_A & \$31.78 & 1.8\% \\
        8 & BANK\_B & \$70.00 & 10.0\% \\
        9 & BANK\_A & \$31.78 & 1.8\% \\
        9 & BANK\_B & \$70.00 & 0.0\% \\
        10 & BANK\_A & \$31.76 & 1.8\% \\
        10 & BANK\_B & \$70.00 & 0.0\% \\
        11 & BANK\_A & \$31.78 & 1.8\% \\
        11 & BANK\_B & \$70.00 & 0.0\% \\
        \hline
    \end{tabular}
\end{table}




\section{Experiment 2: Stochastic Environment - Detailed Results}
\label{app:exp2}

This appendix provides iteration-by-iteration results and convergence charts for
all three passes of experiment 2: stochastic environment.

\subsection{Pass 1}


\begin{figure}[H]
    \centering
    \includegraphics[width=0.85\textwidth]{charts/exp2_pass1_combined.png}
    \caption{Experiment 2: Stochastic Environment - Pass 1 convergence}
    \label{fig:exp2_pass1_convergence}
\end{figure}



\begin{longtable}{llrr}
    \caption{Experiment 2: Stochastic Environment - Pass 1} \label{tab:exp2_pass1} \\
    \hline
    Iteration & Agent & Cost & Liquidity \\
    \hline
    \endfirsthead

    \multicolumn{4}{c}{\tablename\ \thetable{} -- continued from previous page} \\
    \hline
    Iteration & Agent & Cost & Liquidity \\
    \hline
    \endhead

    \hline
    \multicolumn{4}{r}{Continued on next page} \\
    \endfoot

    \hline
    \endlastfoot

        Baseline & BANK\_A & \$498.00 & 50.0\% \\
        Baseline & BANK\_B & \$498.00 & 50.0\% \\
        0 & BANK\_A & \$498.00 & 50.0\% \\
        0 & BANK\_B & \$498.00 & 50.0\% \\
        1 & BANK\_A & \$398.40 & 40.0\% \\
        1 & BANK\_B & \$448.20 & 45.0\% \\
        2 & BANK\_A & \$328.68 & 33.0\% \\
        2 & BANK\_B & \$418.32 & 42.0\% \\
        3 & BANK\_A & \$298.80 & 30.0\% \\
        3 & BANK\_B & \$408.36 & 41.0\% \\
        4 & BANK\_A & \$278.88 & 28.0\% \\
        4 & BANK\_B & \$398.40 & 40.0\% \\
        5 & BANK\_A & \$258.96 & 26.0\% \\
        5 & BANK\_B & \$388.44 & 39.0\% \\
        6 & BANK\_A & \$239.04 & 24.0\% \\
        6 & BANK\_B & \$378.48 & 38.0\% \\
        7 & BANK\_A & \$219.12 & 22.0\% \\
        7 & BANK\_B & \$368.52 & 37.0\% \\
        8 & BANK\_A & \$209.16 & 21.0\% \\
        8 & BANK\_B & \$358.56 & 36.0\% \\
        9 & BANK\_A & \$199.20 & 20.0\% \\
        9 & BANK\_B & \$348.60 & 35.0\% \\
        10 & BANK\_A & \$189.47 & 19.0\% \\
        10 & BANK\_B & \$338.64 & 34.0\% \\
        11 & BANK\_A & \$183.81 & 18.0\% \\
        11 & BANK\_B & \$328.68 & 33.0\% \\
        12 & BANK\_A & \$174.20 & 17.0\% \\
        12 & BANK\_B & \$318.72 & 32.0\% \\
        13 & BANK\_A & \$164.95 & 16.0\% \\
        13 & BANK\_B & \$308.76 & 31.0\% \\
        14 & BANK\_A & \$157.76 & 15.0\% \\
        14 & BANK\_B & \$298.80 & 30.0\% \\
        15 & BANK\_A & \$149.45 & 14.0\% \\
        15 & BANK\_B & \$288.84 & 29.0\% \\
        16 & BANK\_A & \$143.29 & 13.0\% \\
        16 & BANK\_B & \$278.88 & 28.0\% \\
        17 & BANK\_A & \$134.78 & 12.0\% \\
        17 & BANK\_B & \$268.92 & 27.0\% \\
        18 & BANK\_A & \$131.78 & 11.0\% \\
        18 & BANK\_B & \$258.96 & 26.0\% \\
        19 & BANK\_A & \$129.64 & 10.0\% \\
        19 & BANK\_B & \$249.00 & 25.0\% \\
        20 & BANK\_A & \$163.10 & 9.0\% \\
        20 & BANK\_B & \$239.04 & 24.0\% \\
        21 & BANK\_A & \$237.53 & 8.0\% \\
        21 & BANK\_B & \$229.08 & 23.0\% \\
        22 & BANK\_A & \$237.53 & 8.0\% \\
        22 & BANK\_B & \$219.12 & 22.0\% \\
        23 & BANK\_A & \$317.48 & 7.0\% \\
        23 & BANK\_B & \$209.16 & 21.0\% \\
        24 & BANK\_A & \$298.68 & 6.5\% \\
        24 & BANK\_B & \$199.20 & 20.0\% \\
\end{longtable}


\subsection{Pass 2}


\begin{figure}[H]
    \centering
    \includegraphics[width=0.85\textwidth]{charts/exp2_pass2_combined.png}
    \caption{Experiment 2: Stochastic Environment - Pass 2 convergence}
    \label{fig:exp2_pass2_convergence}
\end{figure}



\begin{longtable}{llrr}
    \caption{Experiment 2: Stochastic Environment - Pass 2} \label{tab:exp2_pass2} \\
    \hline
    Iteration & Agent & Cost & Liquidity \\
    \hline
    \endfirsthead

    \multicolumn{4}{c}{\tablename\ \thetable{} -- continued from previous page} \\
    \hline
    Iteration & Agent & Cost & Liquidity \\
    \hline
    \endhead

    \hline
    \multicolumn{4}{r}{Continued on next page} \\
    \endfoot

    \hline
    \endlastfoot

        Baseline & BANK\_A & \$498.00 & 50.0\% \\
        Baseline & BANK\_B & \$498.00 & 50.0\% \\
        0 & BANK\_A & \$498.00 & 50.0\% \\
        0 & BANK\_B & \$498.00 & 50.0\% \\
        1 & BANK\_A & \$448.20 & 45.0\% \\
        1 & BANK\_B & \$348.60 & 35.0\% \\
        2 & BANK\_A & \$398.40 & 40.0\% \\
        2 & BANK\_B & \$298.80 & 30.0\% \\
        3 & BANK\_A & \$298.80 & 30.0\% \\
        3 & BANK\_B & \$288.84 & 29.0\% \\
        4 & BANK\_A & \$249.00 & 25.0\% \\
        4 & BANK\_B & \$278.88 & 28.0\% \\
        5 & BANK\_A & \$199.20 & 20.0\% \\
        5 & BANK\_B & \$268.92 & 27.0\% \\
        6 & BANK\_A & \$157.76 & 15.0\% \\
        6 & BANK\_B & \$258.96 & 26.0\% \\
        7 & BANK\_A & \$129.64 & 10.0\% \\
        7 & BANK\_B & \$249.00 & 25.0\% \\
        8 & BANK\_A & \$163.10 & 9.0\% \\
        8 & BANK\_B & \$239.04 & 24.0\% \\
        9 & BANK\_A & \$164.70 & 8.8\% \\
        9 & BANK\_B & \$229.08 & 23.0\% \\
        10 & BANK\_A & \$197.80 & 8.6\% \\
        10 & BANK\_B & \$219.12 & 22.0\% \\
        11 & BANK\_A & \$248.41 & 8.5\% \\
        11 & BANK\_B & \$209.16 & 21.0\% \\
        12 & BANK\_A & \$233.04 & 8.4\% \\
        12 & BANK\_B & \$199.20 & 20.0\% \\
        13 & BANK\_A & \$235.77 & 8.3\% \\
        13 & BANK\_B & \$189.24 & 19.0\% \\
        14 & BANK\_A & \$269.91 & 8.2\% \\
        14 & BANK\_B & \$179.53 & 18.0\% \\
        15 & BANK\_A & \$269.91 & 8.2\% \\
        15 & BANK\_B & \$169.81 & 17.0\% \\
        16 & BANK\_A & \$269.91 & 8.2\% \\
        16 & BANK\_B & \$161.34 & 16.0\% \\
        17 & BANK\_A & \$236.44 & 8.1\% \\
        17 & BANK\_B & \$152.18 & 15.0\% \\
        18 & BANK\_A & \$237.53 & 8.0\% \\
        18 & BANK\_B & \$145.03 & 14.0\% \\
        19 & BANK\_A & \$239.48 & 7.9\% \\
        19 & BANK\_B & \$138.65 & 13.0\% \\
        20 & BANK\_A & \$207.25 & 7.8\% \\
        20 & BANK\_B & \$136.25 & 12.0\% \\
        21 & BANK\_A & \$269.75 & 7.7\% \\
        21 & BANK\_B & \$136.25 & 12.0\% \\
        22 & BANK\_A & \$237.69 & 7.6\% \\
        22 & BANK\_B & \$136.25 & 12.0\% \\
        23 & BANK\_A & \$272.79 & 7.5\% \\
        23 & BANK\_B & \$136.25 & 12.0\% \\
        24 & BANK\_A & \$243.64 & 7.4\% \\
        24 & BANK\_B & \$134.46 & 11.5\% \\
\end{longtable}


\subsection{Pass 3}


\begin{figure}[H]
    \centering
    \includegraphics[width=0.85\textwidth]{charts/exp2_pass3_combined.png}
    \caption{Experiment 2: Stochastic Environment - Pass 3 convergence}
    \label{fig:exp2_pass3_convergence}
\end{figure}



\begin{longtable}{llrr}
    \caption{Experiment 2: Stochastic Environment - Pass 3} \label{tab:exp2_pass3} \\
    \hline
    Iteration & Agent & Cost & Liquidity \\
    \hline
    \endfirsthead

    \multicolumn{4}{c}{\tablename\ \thetable{} -- continued from previous page} \\
    \hline
    Iteration & Agent & Cost & Liquidity \\
    \hline
    \endhead

    \hline
    \multicolumn{4}{r}{Continued on next page} \\
    \endfoot

    \hline
    \endlastfoot

        Baseline & BANK\_A & \$498.00 & 50.0\% \\
        Baseline & BANK\_B & \$498.00 & 50.0\% \\
        0 & BANK\_A & \$498.00 & 50.0\% \\
        0 & BANK\_B & \$498.00 & 50.0\% \\
        1 & BANK\_A & \$398.40 & 40.0\% \\
        1 & BANK\_B & \$152.18 & 15.0\% \\
        2 & BANK\_A & \$199.20 & 20.0\% \\
        2 & BANK\_B & \$136.25 & 12.0\% \\
        3 & BANK\_A & \$183.81 & 18.0\% \\
        3 & BANK\_B & \$136.25 & 12.0\% \\
        4 & BANK\_A & \$174.20 & 17.0\% \\
        4 & BANK\_B & \$136.25 & 12.0\% \\
        5 & BANK\_A & \$164.95 & 16.0\% \\
        5 & BANK\_B & \$136.25 & 12.0\% \\
        6 & BANK\_A & \$157.76 & 15.0\% \\
        6 & BANK\_B & \$136.25 & 12.0\% \\
        7 & BANK\_A & \$149.45 & 14.0\% \\
        7 & BANK\_B & \$136.25 & 12.0\% \\
        8 & BANK\_A & \$143.29 & 13.0\% \\
        8 & BANK\_B & \$136.25 & 12.0\% \\
        9 & BANK\_A & \$134.78 & 12.0\% \\
        9 & BANK\_B & \$136.25 & 12.0\% \\
        10 & BANK\_A & \$131.78 & 11.0\% \\
        10 & BANK\_B & \$136.25 & 12.0\% \\
        11 & BANK\_A & \$129.64 & 10.0\% \\
        11 & BANK\_B & \$136.25 & 12.0\% \\
        12 & BANK\_A & \$163.10 & 9.0\% \\
        12 & BANK\_B & \$136.25 & 12.0\% \\
        13 & BANK\_A & \$237.53 & 8.0\% \\
        13 & BANK\_B & \$136.25 & 12.0\% \\
        14 & BANK\_A & \$317.48 & 7.0\% \\
        14 & BANK\_B & \$136.25 & 12.0\% \\
        15 & BANK\_A & \$298.68 & 6.5\% \\
        15 & BANK\_B & \$136.25 & 12.0\% \\
        16 & BANK\_A & \$326.00 & 6.0\% \\
        16 & BANK\_B & \$136.25 & 12.0\% \\
        17 & BANK\_A & \$516.04 & 5.5\% \\
        17 & BANK\_B & \$136.25 & 12.0\% \\
        18 & BANK\_A & \$516.04 & 5.5\% \\
        18 & BANK\_B & \$136.25 & 12.0\% \\
        19 & BANK\_A & \$516.04 & 5.5\% \\
        19 & BANK\_B & \$136.25 & 12.0\% \\
        20 & BANK\_A & \$516.04 & 5.5\% \\
        20 & BANK\_B & \$136.25 & 12.0\% \\
        21 & BANK\_A & \$490.81 & 5.6\% \\
        21 & BANK\_B & \$136.25 & 12.0\% \\
        22 & BANK\_A & \$515.29 & 5.6\% \\
        22 & BANK\_B & \$136.25 & 12.0\% \\
        23 & BANK\_A & \$515.15 & 5.6\% \\
        23 & BANK\_B & \$136.25 & 12.0\% \\
        24 & BANK\_A & \$515.15 & 5.6\% \\
        24 & BANK\_B & \$136.25 & 12.0\% \\
\end{longtable}




\section{Experiment 3: Symmetric Equilibrium - Detailed Results}
\label{app:exp3}

This appendix provides iteration-by-iteration results and convergence charts for
all three passes of experiment 3: symmetric equilibrium.

\subsection{Pass 1}


\begin{figure}[H]
    \centering
    \includegraphics[width=0.85\textwidth]{charts/exp3_pass1_combined.png}
    \caption{Experiment 3: Symmetric Equilibrium - Pass 1 convergence}
    \label{fig:exp3_pass1_convergence}
\end{figure}



\begin{table}[H]
    \centering
    \caption{Experiment 3: Symmetric Equilibrium - Pass 1}
    \label{tab:exp3_pass1}
    \begin{tabular}{llrr}
        \hline
        Iteration & Agent & Cost & Liquidity \\
        \hline
        Baseline & BANK\_A & \$49.95 & 50.0\% \\
        Baseline & BANK\_B & \$49.95 & 50.0\% \\
        0 & BANK\_A & \$49.95 & 50.0\% \\
        0 & BANK\_B & \$49.95 & 50.0\% \\
        1 & BANK\_A & \$29.97 & 30.0\% \\
        1 & BANK\_B & \$39.96 & 40.0\% \\
        2 & BANK\_A & \$120.99 & 1.0\% \\
        2 & BANK\_B & \$69.97 & 30.0\% \\
        3 & BANK\_A & \$120.90 & 0.9\% \\
        3 & BANK\_B & \$68.98 & 29.0\% \\
        4 & BANK\_A & \$120.96 & 1.0\% \\
        4 & BANK\_B & \$69.97 & 30.0\% \\
        5 & BANK\_A & \$120.99 & 1.0\% \\
        5 & BANK\_B & \$71.98 & 32.0\% \\
        6 & BANK\_A & \$120.96 & 1.0\% \\
        6 & BANK\_B & \$69.97 & 30.0\% \\
        7 & BANK\_A & \$120.99 & 1.0\% \\
        7 & BANK\_B & \$69.97 & 30.0\% \\
        \hline
    \end{tabular}
\end{table}


\subsection{Pass 2}


\begin{figure}[H]
    \centering
    \includegraphics[width=0.85\textwidth]{charts/exp3_pass2_combined.png}
    \caption{Experiment 3: Symmetric Equilibrium - Pass 2 convergence}
    \label{fig:exp3_pass2_convergence}
\end{figure}



\begin{table}[H]
    \centering
    \caption{Experiment 3: Symmetric Equilibrium - Pass 2}
    \label{tab:exp3_pass2}
    \begin{tabular}{llrr}
        \hline
        Iteration & Agent & Cost & Liquidity \\
        \hline
        Baseline & BANK\_A & \$49.95 & 50.0\% \\
        Baseline & BANK\_B & \$49.95 & 50.0\% \\
        0 & BANK\_A & \$49.95 & 50.0\% \\
        0 & BANK\_B & \$49.95 & 50.0\% \\
        1 & BANK\_A & \$19.98 & 20.0\% \\
        1 & BANK\_B & \$39.96 & 40.0\% \\
        2 & BANK\_A & \$125.01 & 5.0\% \\
        2 & BANK\_B & \$69.97 & 30.0\% \\
        3 & BANK\_A & \$123.99 & 4.0\% \\
        3 & BANK\_B & \$67.96 & 28.0\% \\
        4 & BANK\_A & \$124.50 & 4.5\% \\
        4 & BANK\_B & \$69.46 & 29.5\% \\
        5 & BANK\_A & \$124.80 & 4.8\% \\
        5 & BANK\_B & \$69.88 & 29.9\% \\
        6 & BANK\_A & \$124.89 & 4.9\% \\
        6 & BANK\_B & \$68.98 & 29.0\% \\
        7 & BANK\_A & \$124.89 & 4.9\% \\
        7 & BANK\_B & \$68.98 & 29.0\% \\
        \hline
    \end{tabular}
\end{table}


\subsection{Pass 3}


\begin{figure}[H]
    \centering
    \includegraphics[width=0.85\textwidth]{charts/exp3_pass3_combined.png}
    \caption{Experiment 3: Symmetric Equilibrium - Pass 3 convergence}
    \label{fig:exp3_pass3_convergence}
\end{figure}



\begin{table}[H]
    \centering
    \caption{Experiment 3: Symmetric Equilibrium - Pass 3}
    \label{tab:exp3_pass3}
    \begin{tabular}{llrr}
        \hline
        Iteration & Agent & Cost & Liquidity \\
        \hline
        Baseline & BANK\_A & \$49.95 & 50.0\% \\
        Baseline & BANK\_B & \$49.95 & 50.0\% \\
        0 & BANK\_A & \$49.95 & 50.0\% \\
        0 & BANK\_B & \$49.95 & 50.0\% \\
        1 & BANK\_A & \$19.98 & 20.0\% \\
        1 & BANK\_B & \$19.98 & 20.0\% \\
        2 & BANK\_A & \$209.99 & 10.0\% \\
        2 & BANK\_B & \$200.99 & 1.0\% \\
        3 & BANK\_A & \$209.00 & 9.0\% \\
        3 & BANK\_B & \$200.51 & 0.5\% \\
        4 & BANK\_A & \$209.99 & 10.0\% \\
        4 & BANK\_B & \$200.90 & 0.9\% \\
        5 & BANK\_A & \$207.98 & 8.0\% \\
        5 & BANK\_B & \$200.99 & 1.0\% \\
        6 & BANK\_A & \$209.90 & 9.9\% \\
        6 & BANK\_B & \$200.96 & 0.9\% \\
        7 & BANK\_A & \$209.96 & 10.0\% \\
        7 & BANK\_B & \$200.96 & 0.9\% \\
        \hline
    \end{tabular}
\end{table}




\section{LLM Prompt Audit}
\label{app:prompt_audit}

This appendix documents the LLM prompts used for policy learning and provides
an audit of potential information leakage or bias.

\subsection{Prompt Structure}

Each agent receives a \textbf{system prompt} (identical for all agents) and a
\textbf{user prompt} (agent-specific, filtered).

\subsubsection{System Prompt (Shared)}

The system prompt provides domain context without agent-specific information:

\begin{enumerate}
    \item \textbf{Domain explanation}: RTGS mechanics, queuing, LSM netting
    \item \textbf{Cost structure}: Overdraft, delay, deadline, and EOD penalties
    \item \textbf{Policy tree architecture}: JSON schema for valid policies
    \item \textbf{Optimization guidance}: General strategy for cost minimization
    \item \textbf{Validation checklist}: Common errors to avoid
\end{enumerate}

\subsubsection{User Prompt (Agent-Specific)}

The user prompt provides filtered information for the optimizing agent only:

\begin{enumerate}
    \item \textbf{Performance metrics}: Agent's own mean cost, standard deviation, settlement rate
    \item \textbf{Current policy}: Agent's own policy parameters (e.g., \texttt{initial\_liquidity\_fraction})
    \item \textbf{Cost breakdown}: Agent's own costs by type (delay, overdraft, penalties)
    \item \textbf{Simulation trace}: Filtered event log showing ONLY:
    \begin{itemize}
        \item Outgoing transactions FROM this agent
        \item Incoming payments TO this agent
        \item Agent's own policy decisions (Submit, Hold, etc.)
        \item Agent's own balance changes (for outgoing settlements only)
    \end{itemize}
    \item \textbf{Iteration history}: Agent's own cost trajectory across iterations
    \item \textbf{Parameter trajectories}: How agent's parameters evolved
\end{enumerate}

\subsection{Information Boundaries}

\textbf{Critical invariant}: An agent optimizing policy may ONLY observe:
\begin{itemize}
    \item Events where they are the sender (outgoing transactions)
    \item Events where they are the receiver (incoming liquidity)
    \item Their own state changes (balance, collateral, costs)
\end{itemize}

\subsubsection{What Agents CANNOT See}

\begin{itemize}
    \item \textbf{Counterparty balances}: No visibility into opponent's reserves
    \item \textbf{Counterparty policies}: No access to opponent's decision trees
    \item \textbf{Counterparty costs}: No visibility into opponent's cost breakdown
    \item \textbf{Counterparty reasoning}: No access to opponent's LLM responses
    \item \textbf{Third-party transactions}: Events not involving this agent are filtered
\end{itemize}

This strict isolation is enforced by the \texttt{filter\_events\_for\_agent()} function
which processes the raw simulation event stream before prompt construction.

\subsection{Prompt Sanitization}

All prompts are sanitized to remove:

\begin{itemize}
    \item References to ``optimal'' or ``theoretical'' equilibria
    \item Hints about expected asymmetric vs symmetric outcomes
    \item Explicit game-theoretic terminology (Nash, Pareto, etc.)
    \item Training data leakage from prior experiments
\end{itemize}

\subsection{Audit Conclusions}

Based on our review:

\begin{enumerate}
    \item \textbf{No information leakage}: Agents discover equilibria through
    observed costs, not prompt hints. Counterparty information is strictly filtered.

    \item \textbf{Fair competition}: Both agents receive identically structured
    prompts with symmetric information access. Neither agent has visibility into
    the other's balance, policy, or reasoning.

    \item \textbf{Reproducibility}: The same prompts with identical seeds produce
    identical learning trajectories.

    \item \textbf{Strategic opacity}: Agents cannot observe counterparty reserves
    or pending decisions. The only ``signal'' about counterparty behavior comes
    from incoming payments, which is realistic RTGS transparency.
\end{enumerate}

The experiment results demonstrate genuine strategic learning rather than
prompt-induced behavior, as evidenced by:

\begin{itemize}
    \item Gradual convergence over multiple iterations
    \item Different equilibria across different cost structures
    \item Consistent results across independent passes
\end{itemize}



\end{document}
